% Options for packages loaded elsewhere
\PassOptionsToPackage{unicode}{hyperref}
\PassOptionsToPackage{hyphens}{url}
%
\documentclass[
  12pt,
  twocolumn]{book}
\usepackage{lmodern}
\usepackage{amssymb,amsmath}
\usepackage{ifxetex,ifluatex}
\ifnum 0\ifxetex 1\fi\ifluatex 1\fi=0 % if pdftex
  \usepackage[T1]{fontenc}
  \usepackage[utf8]{inputenc}
  \usepackage{textcomp} % provide euro and other symbols
\else % if luatex or xetex
  \usepackage{unicode-math}
  \defaultfontfeatures{Scale=MatchLowercase}
  \defaultfontfeatures[\rmfamily]{Ligatures=TeX,Scale=1}
\fi
% Use upquote if available, for straight quotes in verbatim environments
\IfFileExists{upquote.sty}{\usepackage{upquote}}{}
\IfFileExists{microtype.sty}{% use microtype if available
  \usepackage[]{microtype}
  \UseMicrotypeSet[protrusion]{basicmath} % disable protrusion for tt fonts
}{}
\makeatletter
\@ifundefined{KOMAClassName}{% if non-KOMA class
  \IfFileExists{parskip.sty}{%
    \usepackage{parskip}
  }{% else
    \setlength{\parindent}{0pt}
    \setlength{\parskip}{6pt plus 2pt minus 1pt}}
}{% if KOMA class
  \KOMAoptions{parskip=half}}
\makeatother
\usepackage{xcolor}
\IfFileExists{xurl.sty}{\usepackage{xurl}}{} % add URL line breaks if available
\IfFileExists{bookmark.sty}{\usepackage{bookmark}}{\usepackage{hyperref}}
\hypersetup{
  hidelinks,
  pdfcreator={LaTeX via pandoc}}
\urlstyle{same} % disable monospaced font for URLs
\usepackage[margin=1in]{geometry}
\usepackage{graphicx,grffile}
\makeatletter
\def\maxwidth{\ifdim\Gin@nat@width>\linewidth\linewidth\else\Gin@nat@width\fi}
\def\maxheight{\ifdim\Gin@nat@height>\textheight\textheight\else\Gin@nat@height\fi}
\makeatother
% Scale images if necessary, so that they will not overflow the page
% margins by default, and it is still possible to overwrite the defaults
% using explicit options in \includegraphics[width, height, ...]{}
\setkeys{Gin}{width=\maxwidth,height=\maxheight,keepaspectratio}
% Set default figure placement to htbp
\makeatletter
\def\fps@figure{htbp}
\makeatother
\setlength{\emergencystretch}{3em} % prevent overfull lines
\providecommand{\tightlist}{%
  \setlength{\itemsep}{0pt}\setlength{\parskip}{0pt}}
\setcounter{secnumdepth}{-\maxdimen} % remove section numbering
\setlength{\parskip}{1em}
\usepackage{float}
\usepackage[]{natbib}
\bibliographystyle{plainnat}

\author{}
\date{\vspace{-2.5em}}

\begin{document}
\frontmatter

\mainmatter
chapter\{Data collection\}

\section{Background on survey design}

As explained by \citet{wiley2004}, a survey is a means of obtaining
quantitative information regarding opinions and experiences of the
respondents in order to explore the views of the target population as a
whole. In this book, a survey is noted as a ``systematic'' method of
collecting data, where the author states that the word ``systematic'' is
deliberately used in order to separate surveys from other methods of
information collection. ``systematic'' is defined by the Collins English
Dictionary as something that
\textit{"is done according to a fixed plan, in a thorough and efficient way"}
\citep{collins-systematic}, and this reflects the manner in which
surveys are created in accordance with a given system, where methods for
distribution, implementation and analysis are defined under a
pre-determined structure. The survey will be delivered to potential
respondents in the target population, who will then be asked to complete
a series of standardised questions, or questions for which the question
ordering and wording is identical for every respondent, unless different
formats are to be used to research purposes. It is once again discussed
by \citet{wiley2004} that standardised questioning was not always the
norm; most interviewers would more likely have a list of objectives, and
each interviewer would formulate and word questions based around these.
It was discovered that question wording can have a drastic effect on
respondents' answers.

Whether or not the survey is `thorough' and `efficient' depends heavily
on the survey structure and design. Designing an effective, systematic
survey involves balancing efficiency with completeness, creating a
survey that can obtain as much information as possible whilst not boring
or fatiguing participants, which can lead to non-response and
measurement errors due to participants skipping questions or selecting
answers at random. A well-designed systematic survey has the capacity to
yield large amounts of both qualitative and quantitative information
regarding the research topic while minimising these errors.

As discussed in {[}CITE:
\url{https://ojs.ub.uni-konstanz.de/srm/article/view/7145}{]}, too long
a survey can result in higher measurement error due to factors such as
waning interest or mental fatigue of respondents, resulting in careless
responding and non-response. This is also further explored in {[}CITE:
\url{https://corescholar.libraries.wright.edu/cgi/viewcontent.cgi?article=3059\&context=etd_all}{]}.
While {[}CITE:
\url{https://ojs.ub.uni-konstanz.de/srm/article/view/7145}{]} does
conclude that a `split survey' design, where each respondent is only
asked to answer a selection of questions from the whole set, is
effective in reducing error while gathering large amount of information,
this will not be employed here. The reasoning for this is that there
will already be a large set of different surveys being sent, and
creating further splits could potentially lead to small sample sizes and
thus inconclusive results. Additionally to this, the paper investigates
how placement of questions in the survey can affect responses,
concluding that questions asked later in the survey are more susceptible
to bias.

There exist a variety of methods for delivering a survey, such as
self-completed questionnaires and interviewer-administered interviews.
Depending on the aims of the study, there will be advantages and
disadvantages to each method. There may also be times when a combined
approach is helpful in gathering the necessary information.

The first method of surveying, a questionnaire, may consist of either
physical paper forms that are mailed or handed out to people within the
target population, or in an online format. As discussed by
\citet{brace2004}, this form of surveying constitutes a method of
indirect communication between the respondent and researcher, in effect
a non-verbal conversation in which the respondent is replying to the
researcher's questions. The non-face-to-face aspect of this method can
be beneficial in terms of anonymity; an anonymous respondent is more
likely to be honest in their answers than a respondent for whom the
identity is known. As a result, an anonymous questionnaire can mitigate
errors that may be caused by respondents fearing judgment of their
answers. It is also possible to administer a large number of these
questionnaires in a short period of time since they are
self-administered, and thus constraints such as the number of
interviewers or time taken to administer the survey has less effect on
the amount of information obtained.

There are, however negatives to this questionnaire method. In his book,
Brace discusses the way in which question wording must be very carefully
thought about when using this method of indirect conversation, for
reasons such as there being no way to correct participant
misunderstanding of questions. Additionally, the fact that the
researcher and participant never come into contact may allow the
researcher to write questions without considering the human nature of
the participants; it is easy to become absorbed in attempting to gather
information and fall into forgetting that long-winded or complicated
questions may bore or confuse respondents, leading to poorer quality
responses. Similarly including too many questions in the questionnaire
may lead to response errors for the same reasons. It is then crucial to
be as clear and concise as possible in question wording, leaving little
room for interpretation. This type of survey is also a very static
medium; it does not allow for much expansion on participants' answers,
with reasoning behind answers unknown unless specifically requested,
which again could add to respondent fatigue and affect quality of
response.

We can attempt to implement some dynamic discussion into a questionnaire
in the form of `open-ended questions', mentioned above as specifically
requesting reasoning behind answers. A questionnaire is composed of two
types of questions; closed-ended questions, for which the respondent
selects their answer from a given set of potential responses, and
open-ended questions, in which the participants are able to write their
answers in a free-form format. Closed-ended questions are very good for
obtaining quantitative data that may be easily categorised and counted,
which is useful for gathering empirical evidence in order to form
objective conclusions regarding the sample population.

Open-ended questions are generally used where more expansion may be
required in addition to the closed-form answer, or if using a
closed-form question would limit the answer range. The Leibniz Institute
for the Social Sciences \citep{leibniz} provides guidance on open-ended
questions, in which the occasions for using open-ended questions are
outlined as:

\begin{itemize}
   \item "knowledge measurement"; with with multiple choice, respondents would have a chance of guessing the correct answer, and thus this would be a sub-optimal way to measure raw knowledge
   \item "Unknown range of possible answers"; multiple choice may be limiting for certain questions, and may cause the researcher to miss important information 
   \item "Avoidance of excessively long lists of response options"; if there is a known range of answers, but this range is very large, it may overwhelm respondents to see all of these as options
   \item "Avoidance of directive questions"; certain questions may have options based on the researcher's own opinions, and thus have the potential to direct the participant in a certain direction, and may not reflect the participants' true views. This links to "unknown range of answers" in that the researcher may incorrectly assume the potential range of answers and thus the given options may not cover the respondents' true opinions.
   \item "Cognitive pretesting", which covers instances such as ensuring the question was understood correctly.
\end{itemize}

To summarise, open-ended questions are useful when either there is not
enough information to set a standardised range of potential responses or
if more information is needed after a closed-ended response.

A method of surveying that is, by design, more dynamic is an interview.
An interview may be structured, semi-structured or structured and each
of these have a different set of features that distinguish them from one
another. Structured interviews, as by the name, are rigid in nature and
comprise of a vocal conversation in which the interviewer has a specific
set of questions from which the discussion does not deviate. The
slightly less rigid semi-structured interview is similar, but slight
deviation from the plan is allowed in order to explore new avenues and
ideas that might not be found with a structured interview, but the
interviewer will still have a set of specific questions for which to
obtain responses. For the most flexible of the three, the unstructured
interview, the interviewer will tend to follow a loose plan of what they
wish to explore rather than a strict question schedule, with the
discussion led by the respondent's answers.

Phone calls and other forms of interview-based survey allow the
interviewer to form a personal connection with the survey participant,
which can be especially helpful for a company's image if the interviewer
is particularly professional or charismatic. Additionally, while the
interviewer will still be limited to asking the pre-set questions, the
format of such a survey can be considered semi-structured and with much
more room for interpretation. This can lend itself to gaining additional
insights that may not have otherwise been gathered from a more
closed-form paper or online survey. Additionally, the more open format
can negate any error as a result of participants misinterpreting
questions due to the interviewer's ability to immediately clarify on any
misunderstandings. This type of survey also provides an instant
response, which is beneficial if there is only a short time frame
available in which to gather information.

However, there are also shortfalls to an interview-based survey method.
For instance, although a charismatic interviewer can positively impact
the image of whoever is conducting the survey, this could also lead to
biases, such as the respondent answering in a way they feel will please
the interviewer. Additionally, the image of the organisation could
potentially be tainted if the interviewer appears rude or
unprofessional, alongside potentially providing bias in the opposite
direction. As well as this, telephone surveys are likely to be
interpreted as a telemarketing scheme, and thus potentially have a
negative impact on the number of willing respondents. The reduced
anonymity of this type of survey may also create bias in the way of
participants avoiding making statements that could be deemed socially
unacceptable, or that they feel they may be judged for, and therefore
may not provide answers accurate to their true line of thought.

The UK Household Longitudinal Study \citep{longitudinal} is an ongoing
study and an example of implementation of a combined use of the above
mentioned surveying methods. Initially, in `wave 1' of the study, a
sample of 40,000 households in the UK were selected to be surveyed on a
yearly basis. The survey involves all members of each selected
household, overall comprising of around 100,000 individuals, and asks
them a range of questions regarding areas such as family life, income,
employment and health. The study consists of a self-administered youth
paper questionnaire given to respondents ages 10-15, and an interview
for those aged 16 and up. This split in age demographic allows some
questions to be omitted from the youth survey, such as those about
income and employment, and some to be added such as about pocket money
habits and `future intentions', as the website states. Giving the youth
respondents a paper questionnaire may help obtain more useful or
relevant answers, as the respondent may be more comfortable with this
than being interviewed by an adult. The youth questionnaire is also
shorter, which could perhaps just be a result of many questions not
being relevant to this demographic, or it could be a conscious decision,
but either way this with help to ensure the young respondent doesn't
lose interest and potentially incur bias in their answers due to either
rushing to finish the survey or not paying attention. The adult survey
also includes a section specific to 16-21 year olds. The surveys contain
a standardised set of core questions asked each year alongside a set
asked every other year. The reasoning behind this is given to be that
this study has a very large scope, asking about many aspects of each
respondents' life, and so it becomes inefficient and counterproductive
to include all questions every year since, as mentioned previously, the
longer a survey is, the more likely a respondent is to get bored or
mentally fatigued. The fact that the adult survey is administered in an
interview also means that there may be limits on the amount of time the
survey can take, as interviewers may have to get through a certain
number of respondents in a day, additionally to the interviewer
potentially also becoming fatigued. If the interviewer is fatigued,
their tone and how they hold themselves may change, and potentially
cause a subconscious bias in how the respondent answers the questions.

\section{Specific goals of survey tool for this study}

While visualisations can be a very useful tool for understanding data,
they also have the potential to be highly misleading. This section of
the study will explore how modifying certain aesthetic features of
visualisations can impact perception and interpretation of data, and how
these modifications can be exploited in order to mislead the observer.
Misleading visualisations may be created in an effort to deliberately
influence the viewers' perceptions, or accidentally as a result of poor
practice and knowledge surrounding data visualisation. In either case,
visualisations have the ability to communicate different messages and
stories depending on how they present the data to the observer. There is
a large amount of research and literature surrounding this topic, both
in terms of providing frameworks for good visualisation practice as well
as looking into how various techniques are used to deceive viewers.
Results from some of these papers will be replicated, as well as used to
form hypotheses which this survey will investigate.

A large amount of the literature exploring misleading tactics in data
visualisation focuses mainly on bar plots and line plots for categorical
and time series data, and so this is what the survey will focus on. The
specific aim of the survey is to test whether altering y-axis scaling,
bar width, bar grouping method and colouring will have an impact on
single data value interpretation and subjective interpretation of
differences in data values.

\section{Survey Design}

The survey design was be inspired by a series of papers, all of which
investigate how different aesthetic and design choices have the
potential to mislead the observer or alter perception.

The 2020 paper ``The Deceptive Potential of Common Design Tactics Used
in Data Visualizations'' \citep{claire-obrian}, as the title suggests,
explores how using different design tactics may mislead the person
seeing the visualisation. Similarly to ``An Empirical Study of Data
Visualisation'', the Claire and O'Brian paper uses a survey to explore
how deceptive visualisation techniques can be employed as well as their
impact on perception of the data. The survey discussed in this paper
presents the participant with four plots; a bar plot, a line plot, a pie
chart and a bubble plot. Additionally to changing aesthetic features of
the plots themselves, the study investigates the use of exaggerated,
leading titles, for example one control plot has the title " Home Sales
Show Increase From 2015 - 2016``, which is altered to''Huge Increase in
Home Sales From 2015 -- 2016!``. The control plots consist of using a
y-axis scaling beginning at 0 for the bar and line plots, a standard pie
chart, and a bubble plot with proportionally sized bubbles, all
alongside the non-exaggerated titles. The altered plots involve
truncating the y-scale for the bar and line plots, making the pie chart
in 3D, and arbitrarily altering the sizes of the bubbles on the bubble
plot. The altered plots are referred to as the''deceptive" plots. The
survey used sets of plots as crossed between deceptive aesthetics and
deceptive titles; two had control aesthetics, one with the control title
and one for the exaggerated title, and two had deceptive aesthetics with
one having the exaggerated titling.

With regard to truncated axes, Claire and O'Brian asked participants to
subjectively judge the difference between two data points using a 6
point scale ranging from ``a little'' to ``a lot''. For both the bar
plot and line plot it was found the the use of a truncated scale
increases the perceived difference between the data points. The use of a
truncated scale is also discussed by \citet{YANG2021}, whereby 5
empirical studies were performed in order to assess the effect of
altering the scale in this way. The first of the 5 studies once again
assessed how large the difference between data points is perceived to be
in the truncated plot as compared to a control, again using a subjective
scale from ``Not at all different'' to ``Extremely different'' on a 7
point scale. This scale differed, however, in the way that a midpoint
label of ``Moderately different'' was provided. The 7 point scale may be
preferable to the 6 point scale as the 7 point has a defined midpoint at
4, whereas the 6 point does not. This study once again concludes that
the differences in data points tended to be perceived as larger than for
the control plot. Alongside these studies, a 2014 blog post
\citep{parikh_2014} discusses axis truncation and its effect on
perceived data point difference for bar plots alongside other aesthetic
features. The first example shows how truncating the y-axis of a bar
plot can over-exaggerate differences in the heights of the bars, perhaps
leading to incorrect observations regarding comparisons of values within
the data.

The paper \citet{stackscale} performs a similar study, but instead
investigates the use of `stack-scale', or `stacked' bar charts and
logarithmic scaling. The aim of the study was to explore whether
stack-scale bar charts are an effective way to visualise large value
data, which is less relevant to our study since we have relatively
low-valued data compared to the paper, but nevertheless provides a
framework for exploring the use of logarithmic scaling and stacked bars
in a respondent study. Participants were shown three plots; a control
with a linear scale, a bar plot using a stack-scale, and one with
logarithmic scaling. The questions asked determined how the different
scaling affected accuracy in reading individual values, interpreting
differences in values and determining which time-step exhibits the
largest difference in values.

Following this, questions included in part 1 the survey focused on
gauging whether altering the y-scale to be truncated or logarithmic has
an effect on user perception of difference in data point values, for
both bar and line plots. These asked the respondents to gauge both
individual values and differences in values, with the former providing
an open answer box in which the may type their answer to allow for
maximum freedom and obtain their true observation, unimpeded by the bias
of having a specific set of numbers to pick from when their true
observation may lie outside this range. The question for gauging
difference perception followed \citet{claire-obrian} and
\citet{YANG2021} in using a numbered scale with numbers representing a
range from not much difference up to a large difference. The
\citet{YANG2021} method of a 7-point scale was employed here. From these
papers, it was hypothesised that the truncated scale would cause
respondents to overestimate differences between data values, and the
logarithmic scale would result in underestimation.

As well as scaling, another aspect of visualisation design that could
potentially mislead the observer is bar width and aspect ratios. When
adding a visualisation into a publication, re-sizing the visualisation
to fit a specific gap may include altering the aspect ratio, in turn
affecting the length to width ratio of the bars in a bar plot. As
explored by Steven Few in a 2016 article for the
\textit{'Visual Business Intelligence Newsletter'} \citep{Few2016},
altering this ratio can affect viewer perception in the way of a
narrower and taller image distorting bars to appear longer, and vice
versa, meaning that perceived differences between bar heights may be
affected. Part 2 of the survey was based around investigating this idea,
alongside how the reading of exact values is affected.

Additionally, stacked bar charts will be investigated, showing a
comparison between using the stacking method as opposed to a grouped bar
plot. An article from the University of Stuttgart
\citep{HuynhHaiDang2017} gives an overview of may types of bar chart,
including stacked and grouped bars. The author remarks that grouped bar
charts may make the comparison of bars in the same category more
difficult, while the stacked bar chart sacrifices ease of comparison of
values in the bars for increased spacial efficiency. A 2018 work from
the journal of \textit{'Visual Informatics'} \citep{INDRATMO2018155}
also provides a discussion on the use of various forms of stacked and
grouped bar charts and their efficacy. The paper notes how a classical
stacked bar chart can be useful for overall comparisons as the height of
the bar represents the value of the item, with the different attributes
depicted as a segmentation of this single bar into different colours.
When discussing grouped bar charts it is mentioned that stacked bar
charts may be less useful when performing attribute comparisons, in
other words comparisons between different categories on the same bar, as
a result of the bar segments being non-aligned. This results in
comparison taking the form of length judgment as opposed to position
judgment. Cleveland and McGill in their 1984 article in the
\textit{'Journal of the American Statistical Association'}
\citep{clevelandmcgill} discuss how judgments based on length are likely
to be less accurate than those based on position. A grouped bar chart is
a way to allow for easy comparison between individual categories, but is
discussed to be less effective in overall comparison. Based on this
research, part 3 of the survey will include questions with the objective
of testing standard stacked against grouped bar charts, alongside
questions relating to the colour palettes used in depicting the
different groups.

The last two parts of the survey focused on the different y-axis
scalings with respect to line plots, but for these, as opposed to the
bar plots, the default was a truncated axis. We can also investigate the
effects of using numerical vs word labels for months.

Although the content of the surveys for this study is not likely to be
controversial or highly personal, anonymity is still important as the
participants could otherwise potentially feel pressure to give a
`correct' answer, given the mathematical nature of the questions.
Anonymity here means that this pressure is potentially reduced and thus
the relevant measurement bias may be mitigated.

The decision to focus on the two plot types was made in part to follow
the existing literature, but also to ensure the survey is not too long.

\%\%\%\%\% Potential sources about survey length \%\%\%\%\% \%
\url{https://journals.sagepub.com/doi/pdf/10.2501/IJMR-2017-039}
-\textgreater{} Study about ideal length of survey in terms of time \%
\url{https://ojs.ub.uni-konstanz.de/srm/article/view/7145}
-\textgreater{} Study about reducing measurement error due to survey
length \% \url{https://corescholar.libraries.wright.edu/etd_all/1918/}
-\textgreater{} About `careless responding' and `insufficient effort
responding' \%
\url{https://journals.sagepub.com/doi/abs/10.1177/089443930101900202}

The set will consist of two separate surveys, which will be identical up
to the visualisation package used. Particularly, one will contain
visualisations made with R's ggplot2, the next with matplotlib from
Python. These surveys will be distributed to the general public by
sharing links on social media platforms such as Facebook. The reasoning
behind creating three separate surveys in a variety of languages is to
ascertain whether the language used influences the interpretation.

\section{The survey}

Potentially will use google sheets and post all three, and ask people to
complete one of them. \%TODO Specify further

\section{\textbf{Demographic Questions}}

The questions below are used to

\begin{itemize}
    \item Please enter your age
    
    \item If you are a university student or past university graduate please specify your area of study. (Drop down box: Science, Technology, Engineering, Maths, Arts, Social Sciences, Humanities, Business, N/A, Other (please specify))

    \item How strongly do you agree with each of the following statements? (Linear scale with 1 - 5, 1=strongly disagree, 5=strongly agree)

    \item\item I have good spatial awareness skills 
    
    \item\item I have good observational skills 
    
    \item\item I have good numerical skills 
    
    \item Are you colourblind? (Checkbox: Yes, No, Prefer not to answer)
    
    \item Do you have any disorders that may affect visual processing? (this could be a general visual processing disorder 
    or dyslexia, dyscalculia etc)
    ((Checkbox: Yes, No, Prefer not to answer))
\end{itemize}

\section{\textbf{Bar Plot Questions}}

American Ninja Warrior\newline

The following bar charts present information regarding how many times 4
obstacles were used over the course of 10 seasons of the TV show
`American Ninja Warrior'. Please note that the answers to this section
are entirely subjective, and that there are no `correct'
answers.\newline

The first three questions refer to this bar chart, bar chart A.

\textbf{Figure}

\%Note: Deliberately design `bad' as well as good plots?

\%\%\%\%\% Misleading aspects of plots: \%\%\%\%\%

\% Axis Scales \% - Sources: \% -
\url{https://www.nature.com/articles/s41559-018-0610-7} \% -
\url{https://web.archive.org/web/20101123050530/http://graphpad.com/faq/file/1487logaxes.pdf}
\% - \url{https://dl.acm.org/doi/fullHtml/10.1145/3231772} \% -
\url{https://journals.sagepub.com/doi/pdf/10.1177/1050651920958392} \% -
\url{https://journals.sagepub.com/doi/pdf/10.1177/1050651920958392} \% -

\%\%\%\%\%

\% TODO Write about choosing more `basic' questions to cut down on
number of questions as well as appealing to audience

\section{Conclusion}

\backmatter
  \bibliography{main.bib}

\end{document}
