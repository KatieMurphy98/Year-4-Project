% Options for packages loaded elsewhere
\PassOptionsToPackage{unicode}{hyperref}
\PassOptionsToPackage{hyphens}{url}
%
\documentclass[
]{article}
\usepackage{lmodern}
\usepackage{amssymb,amsmath}
\usepackage{ifxetex,ifluatex}
\ifnum 0\ifxetex 1\fi\ifluatex 1\fi=0 % if pdftex
  \usepackage[T1]{fontenc}
  \usepackage[utf8]{inputenc}
  \usepackage{textcomp} % provide euro and other symbols
\else % if luatex or xetex
  \usepackage{unicode-math}
  \defaultfontfeatures{Scale=MatchLowercase}
  \defaultfontfeatures[\rmfamily]{Ligatures=TeX,Scale=1}
\fi
% Use upquote if available, for straight quotes in verbatim environments
\IfFileExists{upquote.sty}{\usepackage{upquote}}{}
\IfFileExists{microtype.sty}{% use microtype if available
  \usepackage[]{microtype}
  \UseMicrotypeSet[protrusion]{basicmath} % disable protrusion for tt fonts
}{}
\makeatletter
\@ifundefined{KOMAClassName}{% if non-KOMA class
  \IfFileExists{parskip.sty}{%
    \usepackage{parskip}
  }{% else
    \setlength{\parindent}{0pt}
    \setlength{\parskip}{6pt plus 2pt minus 1pt}}
}{% if KOMA class
  \KOMAoptions{parskip=half}}
\makeatother
\usepackage{xcolor}
\IfFileExists{xurl.sty}{\usepackage{xurl}}{} % add URL line breaks if available
\IfFileExists{bookmark.sty}{\usepackage{bookmark}}{\usepackage{hyperref}}
\hypersetup{
  hidelinks,
  pdfcreator={LaTeX via pandoc}}
\urlstyle{same} % disable monospaced font for URLs
\usepackage[margin=1in]{geometry}
\usepackage{color}
\usepackage{fancyvrb}
\newcommand{\VerbBar}{|}
\newcommand{\VERB}{\Verb[commandchars=\\\{\}]}
\DefineVerbatimEnvironment{Highlighting}{Verbatim}{commandchars=\\\{\}}
% Add ',fontsize=\small' for more characters per line
\usepackage{framed}
\definecolor{shadecolor}{RGB}{248,248,248}
\newenvironment{Shaded}{\begin{snugshade}}{\end{snugshade}}
\newcommand{\AlertTok}[1]{\textcolor[rgb]{0.94,0.16,0.16}{#1}}
\newcommand{\AnnotationTok}[1]{\textcolor[rgb]{0.56,0.35,0.01}{\textbf{\textit{#1}}}}
\newcommand{\AttributeTok}[1]{\textcolor[rgb]{0.77,0.63,0.00}{#1}}
\newcommand{\BaseNTok}[1]{\textcolor[rgb]{0.00,0.00,0.81}{#1}}
\newcommand{\BuiltInTok}[1]{#1}
\newcommand{\CharTok}[1]{\textcolor[rgb]{0.31,0.60,0.02}{#1}}
\newcommand{\CommentTok}[1]{\textcolor[rgb]{0.56,0.35,0.01}{\textit{#1}}}
\newcommand{\CommentVarTok}[1]{\textcolor[rgb]{0.56,0.35,0.01}{\textbf{\textit{#1}}}}
\newcommand{\ConstantTok}[1]{\textcolor[rgb]{0.00,0.00,0.00}{#1}}
\newcommand{\ControlFlowTok}[1]{\textcolor[rgb]{0.13,0.29,0.53}{\textbf{#1}}}
\newcommand{\DataTypeTok}[1]{\textcolor[rgb]{0.13,0.29,0.53}{#1}}
\newcommand{\DecValTok}[1]{\textcolor[rgb]{0.00,0.00,0.81}{#1}}
\newcommand{\DocumentationTok}[1]{\textcolor[rgb]{0.56,0.35,0.01}{\textbf{\textit{#1}}}}
\newcommand{\ErrorTok}[1]{\textcolor[rgb]{0.64,0.00,0.00}{\textbf{#1}}}
\newcommand{\ExtensionTok}[1]{#1}
\newcommand{\FloatTok}[1]{\textcolor[rgb]{0.00,0.00,0.81}{#1}}
\newcommand{\FunctionTok}[1]{\textcolor[rgb]{0.00,0.00,0.00}{#1}}
\newcommand{\ImportTok}[1]{#1}
\newcommand{\InformationTok}[1]{\textcolor[rgb]{0.56,0.35,0.01}{\textbf{\textit{#1}}}}
\newcommand{\KeywordTok}[1]{\textcolor[rgb]{0.13,0.29,0.53}{\textbf{#1}}}
\newcommand{\NormalTok}[1]{#1}
\newcommand{\OperatorTok}[1]{\textcolor[rgb]{0.81,0.36,0.00}{\textbf{#1}}}
\newcommand{\OtherTok}[1]{\textcolor[rgb]{0.56,0.35,0.01}{#1}}
\newcommand{\PreprocessorTok}[1]{\textcolor[rgb]{0.56,0.35,0.01}{\textit{#1}}}
\newcommand{\RegionMarkerTok}[1]{#1}
\newcommand{\SpecialCharTok}[1]{\textcolor[rgb]{0.00,0.00,0.00}{#1}}
\newcommand{\SpecialStringTok}[1]{\textcolor[rgb]{0.31,0.60,0.02}{#1}}
\newcommand{\StringTok}[1]{\textcolor[rgb]{0.31,0.60,0.02}{#1}}
\newcommand{\VariableTok}[1]{\textcolor[rgb]{0.00,0.00,0.00}{#1}}
\newcommand{\VerbatimStringTok}[1]{\textcolor[rgb]{0.31,0.60,0.02}{#1}}
\newcommand{\WarningTok}[1]{\textcolor[rgb]{0.56,0.35,0.01}{\textbf{\textit{#1}}}}
\usepackage{graphicx,grffile}
\makeatletter
\def\maxwidth{\ifdim\Gin@nat@width>\linewidth\linewidth\else\Gin@nat@width\fi}
\def\maxheight{\ifdim\Gin@nat@height>\textheight\textheight\else\Gin@nat@height\fi}
\makeatother
% Scale images if necessary, so that they will not overflow the page
% margins by default, and it is still possible to overwrite the defaults
% using explicit options in \includegraphics[width, height, ...]{}
\setkeys{Gin}{width=\maxwidth,height=\maxheight,keepaspectratio}
% Set default figure placement to htbp
\makeatletter
\def\fps@figure{htbp}
\makeatother
\setlength{\emergencystretch}{3em} % prevent overfull lines
\providecommand{\tightlist}{%
  \setlength{\itemsep}{0pt}\setlength{\parskip}{0pt}}
\setcounter{secnumdepth}{-\maxdimen} % remove section numbering
\usepackage{booktabs}
\usepackage{longtable}
\usepackage{array}
\usepackage{multirow}
\usepackage{wrapfig}
\usepackage{float}
\usepackage{colortbl}
\usepackage{pdflscape}
\usepackage{tabu}
\usepackage{threeparttable}
\usepackage{threeparttablex}
\usepackage[normalem]{ulem}
\usepackage{makecell}
\usepackage{xcolor}

\author{}
\date{\vspace{-2.5em}}

\begin{document}

\chapter{Univariate Analysis}

This chapter will discuss basic univariate analysis of the survey
results, including summary statistics and univariate testing for the
whole population as well as the subsetting for the programming language
used and degree type. Additionally, subsets will be created considering
only the first plot shown for each question, drawing comparisons between
responses for these plots themselves without influence of the others.
The analysis will be performed in R version R version 4.0.2 {[}@R{]}.

In terms of testing, Shapiro-Wilk tests will be applied with the
\texttt{shapiro.test()} function to gauge whether the data sets can be
considered normally distributed and thus whether parametric T-Tests are
suitable for either one-sample or paired comparisons, for the
Shapiro-Wilk test, the alternative hypothesis is that the data is not
normally distributed. Failing he normality condition, a symmetry test
will be administered via the \texttt{symmetry.test()} function from the
package \texttt{lawstat} {[}@lawstat{]}, and providing there is
insufficient evidence to reject the null hypothesis that the data is
symmetric, a Mann-Whitney-Wilcoxon (MWW) test will be used. If there is
sufficient evidence that data proves neither symmetric nor normally
distributed, sign tests will be applied. MWW will also be used for two
sample testing where perhaps a sign test would be most appropriate, but
cannot be used as the samples are of different sizes.

The sample sizes are 70, 38 and 32 for the whole population, R subgroup
and Python subgroup, respectively before removing NA of invalid values.
The sample means and medians will be notated as \(\bar{x}\) and
\(\tilde{x}\), respectively.

See appendix 2 for all statistical testing results and p-values.

\section{American Ninja Warrior - Part 1}

This part of the survey assess the effect of truncated and logarithmic
scaling on bar plots perception and interpretation.

The final question in part 1 of the survey,
\textit{'In your opinion, approximately how many times would you say 'Log Grip' was used, as a percentage of the number of times 'Salmon Ladder' was used?'}
will not be considered as it is similar to the previous questions, and
responses ranged in form, between percentages and decimals, and it can
not just be assumed that all the decimals can be converted to
percentages; for example a value of 0.5 could be the decimal value for
\(50\%\), or the respondent could have meant this as \(0.5\%\).

\subsection{Effect of Y-Axis Truncation}

In general, truncating the y-axis had less of an effect than
anticipated. In question 1,
\textit{"Approximately many times would you say the ‘Salmon Ladder’ was used?"},
for which the true value was 41, the distribution of responses for the
truncated plot (\(\bar{x} = 41.35\)) as compared to that of the control
plot responses (\(\bar{x}=41.21\)) shows a small difference, with the
mean perceived value of the bar being slightly higher for the truncated
plot. The median for both of these is 41, showing that both
distributions are centered around the true value of 41. The control and
truncated plots have contextually fairly small variances of 0.752 and
0.753 respectively, depicting both that there is limited variation in
the responses and most of the observations lie fairly close to the
respective means. The variances are also quite similar, showing that the
distributions appear fairly similar, as emphasised by observing figure
3.1 below.

\begin{verbatim}
## Warning in cbind(control_1, truncated_1, logarithmic_1): number of rows of
## result is not a multiple of vector length (arg 3)
\end{verbatim}

\begin{Shaded}
\begin{Highlighting}[]
\NormalTok{brks <-}\StringTok{ }\KeywordTok{c}\NormalTok{(}\StringTok{"Control"}\NormalTok{, }\StringTok{"Truncated"}\NormalTok{)}
\NormalTok{vals <-}\StringTok{ }\KeywordTok{c}\NormalTok{(}\StringTok{"#1c9e77"}\NormalTok{, }\StringTok{"#d95f02"}\NormalTok{)}
\end{Highlighting}
\end{Shaded}

\begin{Shaded}
\begin{Highlighting}[]
\KeywordTok{ggplot}\NormalTok{() }\OperatorTok{+}
\StringTok{  }\KeywordTok{geom_density}\NormalTok{(}\DataTypeTok{data =} \KeywordTok{as.data.frame}\NormalTok{(control_}\DecValTok{1}\NormalTok{), }\KeywordTok{aes}\NormalTok{(}\DataTypeTok{x=}\NormalTok{control_}\DecValTok{1}\NormalTok{, }\DataTypeTok{col =} \StringTok{"Control"}\NormalTok{))}\OperatorTok{+}
\StringTok{  }\KeywordTok{geom_density}\NormalTok{(}\DataTypeTok{data =} \KeywordTok{as.data.frame}\NormalTok{(truncated_}\DecValTok{1}\NormalTok{), }\KeywordTok{aes}\NormalTok{(}\DataTypeTok{x=}\NormalTok{truncated_}\DecValTok{1}\NormalTok{, }\DataTypeTok{col =} \StringTok{"Truncated"}\NormalTok{))}\OperatorTok{+}
\StringTok{  }\KeywordTok{labs}\NormalTok{(}\DataTypeTok{x=}\StringTok{"Response"}\NormalTok{, }\DataTypeTok{y=}\StringTok{"Density"}\NormalTok{)}\OperatorTok{+}
\StringTok{  }\KeywordTok{scale_colour_manual}\NormalTok{(}\DataTypeTok{name =} \StringTok{" "}\NormalTok{, }\DataTypeTok{breaks =}\NormalTok{ brks, }\DataTypeTok{values =}\NormalTok{ vals)}\OperatorTok{+}
\StringTok{  }\KeywordTok{theme_classic}\NormalTok{()}
\end{Highlighting}
\end{Shaded}

\begin{figure}
\centering
\includegraphics{main_files/figure-latex/unnamed-chunk-3-1.pdf}
\caption{Density plot showing distributions of responses regarding the
control and truncated plots for the question 1}
\end{figure}

Performing a dependent-samples sign test comparing these two sets of
responses confirms that there is no significant difference
(\(p = 0.1877\)) in the response distributions. However, the one sample
sign tests show that there is not sufficient evidence to suggest the
control plot responses differ from the true value of 41
(\(p = 0.1214\)), but there is evidence to accept the hypothesis that
the truncated plot responses differ from the true value
(\(p = 0.0026\)). This shows that, while there is insufficient evidence
from sign testing to suggest a statistically difference in the responses
for the two plots, the location of the truncated plot responses may be
slightly further from the true value than the control, and it is
confirmed by a one sided sign test with an alternative hypothesis that
the true median of truncated responses is greater than 41
(\(p=0.0002\)). This gives evidence that the truncated plot results in a
slight overestimation in reading of the bar height as compared to the
true value of 41. Note that in the responses for the control plot for
question 1, there was a response of ``41/41'', which was taken to be
41.5.

In question 2,
\textit{'Approximately how much more than 'Log Grip' would you say 'Salmon Ladder' was was used?'},
the set of responses for the truncated plot (\(\bar{x} = 5.87\),
\(\tilde{x} = 6\)) is considered significantly different by a
dependent-samples sign test from the control plot responses
(\(\bar{x} = 5.36\), \(\tilde{x} = 5\)). By eye, the average values do
not seem too different between the two plot types, although the p-value
of the sign test (\(p = 0.00019\)) shows that there is in fact a
statistically significant difference. The perceived difference for the
truncated plot being rated higher on average than for the control plot
provides evidence to accept the hypothesis that using a truncated scale
can cause differences in bar height to appear larger, once again this is
confirmed by a one-sided sign test (\(p=9.554e-05\)), with the
alternative hypothesis that the true median of truncated responses is
greater than that of the control responses.

\begin{Shaded}
\begin{Highlighting}[]
\NormalTok{control_}\DecValTok{2}\NormalTok{ <-}\StringTok{ }\NormalTok{ctrl_y_scale}\OperatorTok{$}\NormalTok{con_}\DecValTok{2}
\NormalTok{truncated_}\DecValTok{2}\NormalTok{ <-}\StringTok{ }\NormalTok{trnc_y_scale}\OperatorTok{$}\NormalTok{trn_}\DecValTok{2}
\NormalTok{resp <-}\StringTok{ }\KeywordTok{c}\NormalTok{(control_}\DecValTok{2}\NormalTok{, truncated_}\DecValTok{2}\NormalTok{)}
\NormalTok{type <-}\StringTok{ }\KeywordTok{c}\NormalTok{(}\KeywordTok{rep}\NormalTok{(}\StringTok{'Control'}\NormalTok{, }\DecValTok{70}\NormalTok{), }\KeywordTok{rep}\NormalTok{(}\StringTok{'Truncated'}\NormalTok{, }\DecValTok{70}\NormalTok{)) }
\NormalTok{stats_}\DecValTok{2}\NormalTok{ <-}\StringTok{ }\KeywordTok{data.frame}\NormalTok{(resp, type)}
\end{Highlighting}
\end{Shaded}

\begin{Shaded}
\begin{Highlighting}[]
\KeywordTok{ggplot}\NormalTok{(stats_}\DecValTok{2}\NormalTok{)}\OperatorTok{+}
\StringTok{  }\KeywordTok{geom_bar}\NormalTok{(}\KeywordTok{aes}\NormalTok{(}\DataTypeTok{x=}\NormalTok{resp, }\DataTypeTok{group =}\NormalTok{ type, }\DataTypeTok{fill =}\NormalTok{ type), }\DataTypeTok{position =} \StringTok{'dodge'}\NormalTok{) }\OperatorTok{+}
\StringTok{  }\KeywordTok{ylab}\NormalTok{(}\StringTok{"Number of Respondents"}\NormalTok{)}\OperatorTok{+}
\StringTok{  }\KeywordTok{xlab}\NormalTok{(}\StringTok{"Response"}\NormalTok{)}\OperatorTok{+}
\StringTok{  }\KeywordTok{labs}\NormalTok{(}\DataTypeTok{title =} \StringTok{"Responses selected Over the Whole Population"}\NormalTok{)}\OperatorTok{+}
\StringTok{  }\KeywordTok{scale_fill_brewer}\NormalTok{(}\DataTypeTok{palette=}\StringTok{"Dark2"}\NormalTok{, }\DataTypeTok{labels =} \KeywordTok{c}\NormalTok{(}\StringTok{'Control'}\NormalTok{, }\StringTok{'Truncated'}\NormalTok{))}\OperatorTok{+}
\StringTok{  }\KeywordTok{scale_x_discrete}\NormalTok{(}\DataTypeTok{breaks =} \KeywordTok{c}\NormalTok{(}\StringTok{"1"}\NormalTok{, }\StringTok{"2"}\NormalTok{, }\StringTok{"3"}\NormalTok{, }\StringTok{"4"}\NormalTok{, }\StringTok{"5"}\NormalTok{, }\StringTok{"6"}\NormalTok{, }\StringTok{"7"}\NormalTok{), }\DataTypeTok{labels =} \KeywordTok{c}\NormalTok{(}\StringTok{"1"}\NormalTok{, }\StringTok{"2"}\NormalTok{, }\StringTok{"3"}\NormalTok{, }\StringTok{"4"}\NormalTok{, }\StringTok{"5"}\NormalTok{, }\StringTok{"6"}\NormalTok{, }\StringTok{"7"}\NormalTok{), }\DataTypeTok{limits =} \KeywordTok{c}\NormalTok{(}\StringTok{"1"}\NormalTok{, }\StringTok{"2"}\NormalTok{, }\StringTok{"3"}\NormalTok{, }\StringTok{"4"}\NormalTok{, }\StringTok{"5"}\NormalTok{, }\StringTok{"6"}\NormalTok{, }\StringTok{"7"}\NormalTok{))}\OperatorTok{+}
\StringTok{  }\KeywordTok{theme_classic}\NormalTok{()}
\end{Highlighting}
\end{Shaded}

\begin{figure}
\centering
\includegraphics{main_files/figure-latex/unnamed-chunk-5-1.pdf}
\caption{Bar plot showing distributions of responses regarding the
control and truncated plots for question 2}
\end{figure}

The spread for the truncated and control plot responses are slightly
skewed to the right, depicting that the subjective view on the
difference between the bar heights was that it was in general on the
larger side. Looking at the bar heights, for the responses of 4 and 5
the control plot bars are higher, and vice versa for the truncated plot
response bars. This again emphasises the evidence to support the
hypothesis that truncation leads to larger perceived difference.

Question 3 of part 1,
\textit{'Approximately how much more than 'Quintuple Steps' would you say 'Salmon Ladder' was used?'},
asks a similar question to question 2, but asks respondents to judge the
difference for bars on opposite ends of the plot as opposed to next to
each. Again, the by eye comparison shows not a massive difference
between distributions of responses for the control (\(\bar{x}=3.12\),
\(\tilde{x}=3\)) and truncated (\(\bar{x}=3.12\), \(\tilde{x}=3\))
plots, although the sign test shows that the there is evidence to
suggest that the truncated plot responses are in fact on average greater
than for the control plot (\(p=4.624e-06\)). figure{[}?{]} shows the
distribution of responses.

\begin{Shaded}
\begin{Highlighting}[]
\NormalTok{control_}\DecValTok{3}\NormalTok{ <-}\StringTok{ }\NormalTok{ctrl_y_scale}\OperatorTok{$}\NormalTok{con_}\DecValTok{3}
\NormalTok{truncated_}\DecValTok{3}\NormalTok{ <-}\StringTok{ }\NormalTok{trnc_y_scale}\OperatorTok{$}\NormalTok{trn_}\DecValTok{3}
\NormalTok{resp <-}\StringTok{ }\KeywordTok{c}\NormalTok{(control_}\DecValTok{3}\NormalTok{, truncated_}\DecValTok{3}\NormalTok{)}
\NormalTok{type <-}\StringTok{ }\KeywordTok{c}\NormalTok{(}\KeywordTok{rep}\NormalTok{(}\StringTok{'Control'}\NormalTok{, }\DecValTok{70}\NormalTok{), }\KeywordTok{rep}\NormalTok{(}\StringTok{'Truncated'}\NormalTok{, }\DecValTok{70}\NormalTok{)) }
\NormalTok{stats_}\DecValTok{3}\NormalTok{ <-}\StringTok{ }\KeywordTok{data.frame}\NormalTok{(resp, type)}
\end{Highlighting}
\end{Shaded}

\begin{Shaded}
\begin{Highlighting}[]
\KeywordTok{ggplot}\NormalTok{(stats_}\DecValTok{3}\NormalTok{)}\OperatorTok{+}
\StringTok{  }\KeywordTok{geom_bar}\NormalTok{(}\KeywordTok{aes}\NormalTok{(}\DataTypeTok{x=}\NormalTok{resp, }\DataTypeTok{group =}\NormalTok{ type, }\DataTypeTok{fill =}\NormalTok{ type), }\DataTypeTok{position =} \StringTok{'dodge'}\NormalTok{) }\OperatorTok{+}
\StringTok{  }\KeywordTok{ylab}\NormalTok{(}\StringTok{"Number of Respondents"}\NormalTok{)}\OperatorTok{+}
\StringTok{  }\KeywordTok{xlab}\NormalTok{(}\StringTok{"Response"}\NormalTok{)}\OperatorTok{+}
\StringTok{  }\KeywordTok{labs}\NormalTok{(}\DataTypeTok{title =} \StringTok{"Responses selected Over the Whole Population"}\NormalTok{)}\OperatorTok{+}
\StringTok{  }\KeywordTok{scale_fill_brewer}\NormalTok{(}\DataTypeTok{palette=}\StringTok{"Dark2"}\NormalTok{, }\DataTypeTok{labels =} \KeywordTok{c}\NormalTok{(}\StringTok{'Control'}\NormalTok{, }\StringTok{'Truncated'}\NormalTok{))}\OperatorTok{+}
\StringTok{  }\KeywordTok{scale_x_discrete}\NormalTok{(}\DataTypeTok{breaks =} \KeywordTok{c}\NormalTok{(}\StringTok{"1"}\NormalTok{, }\StringTok{"2"}\NormalTok{, }\StringTok{"3"}\NormalTok{, }\StringTok{"4"}\NormalTok{, }\StringTok{"5"}\NormalTok{, }\StringTok{"6"}\NormalTok{, }\StringTok{"7"}\NormalTok{), }\DataTypeTok{labels =} \KeywordTok{c}\NormalTok{(}\StringTok{"1"}\NormalTok{, }\StringTok{"2"}\NormalTok{, }\StringTok{"3"}\NormalTok{, }\StringTok{"4"}\NormalTok{, }\StringTok{"5"}\NormalTok{, }\StringTok{"6"}\NormalTok{, }\StringTok{"7"}\NormalTok{), }\DataTypeTok{limits =} \KeywordTok{c}\NormalTok{(}\StringTok{"1"}\NormalTok{, }\StringTok{"2"}\NormalTok{, }\StringTok{"3"}\NormalTok{, }\StringTok{"4"}\NormalTok{, }\StringTok{"5"}\NormalTok{, }\StringTok{"6"}\NormalTok{, }\StringTok{"7"}\NormalTok{))}\OperatorTok{+}
\StringTok{  }\KeywordTok{theme_classic}\NormalTok{()}
\end{Highlighting}
\end{Shaded}

\begin{figure}
\centering
\includegraphics{main_files/figure-latex/unnamed-chunk-7-1.pdf}
\caption{Bar plot showing distributions of responses regarding the
control and truncated plots for question 3}
\end{figure}

The response distributions, conversely to question 2, now seem skewed
more to the left. However there is a similarity in the way that for the
lower ratings of 2 and 3, the control plot response bars dominate, and
for the responses of 4 and 5 the opposite is true.

Overall, it seems that the use of truncation has a small but
statistically significant effect on perception of height difference
between bars, with respondents tending to judge the difference as
slightly larger than for the control plot, although this effect is
smaller than initially anticipated, and larger for bars that are further
apart. In terms of reading values from bars, the truncation did not have
a statistically significant effect when comparing the two distributions,
however in one sample testing the truncated plot responses did differ
significantly from the true value.

When considering the language subgroups, note that there is a
discrepancy here between languages in terms of the axis tick breaks and
labeling, with the R plot being incremented in steps of 10 for both the
control and truncated plots and the Python being more granular in steps
of 5 for the control and steps of 2.5 for the truncated.

Consider question 1. Comparing the two language subgroups for the
truncated plot, the distributions for both the R (\(\bar{x} = 41.56\),
\(\tilde{x} = 41\)) and Python (\(\bar{x} = 41.01\), \(\tilde{x} = 41\))
responses to question 1 appear similar in location to those of both each
other and the whole population (\(\bar{x} = 41.35\),
\(\tilde{x} = 41\)).

Comparisons via MWW testing show that the responses related to the
control plot differ statistically significantly between the two language
cohorts (\(p=0.00012\)), and similar for the truncated plot responses
(\(p=0.02163\)), where the tests were performed comparing first the R
and Python responses for the control plot, and then for the truncated.

A sign test shows sufficient evidence that the R subgroup responses
relating to the truncated plot differ from the true value
(\(p = 0.0004\)), whereas there is insufficient evidence when applying a
MWW test to the Python responses (\(p = 0.718\)). Similarly, the R
subgroup's responses in relation to the control plot statistically
significantly differ from the true value (\(p=7.629e-05\)), but the
Python subgroup's do not (\(p=0.1185\)). This could potentially be a
result of the less granulated R plot scaling, due to the reduced
precision.

\begin{Shaded}
\begin{Highlighting}[]
\NormalTok{control_}\DecValTok{1}\NormalTok{_r <-}\StringTok{ }\NormalTok{ctrl_y_scale_r}\OperatorTok{$}\NormalTok{con_}\DecValTok{1}
\NormalTok{control_}\DecValTok{1}\NormalTok{_r[}\KeywordTok{which}\NormalTok{(control_}\DecValTok{1}\NormalTok{_r }\OperatorTok{==}\StringTok{ "41/42"}\NormalTok{)] <-}\StringTok{ }\FloatTok{41.5}
\NormalTok{control_}\DecValTok{1}\NormalTok{_r <-}\StringTok{ }\KeywordTok{na.exclude}\NormalTok{(}\KeywordTok{as.numeric}\NormalTok{(control_}\DecValTok{1}\NormalTok{_r))}

\NormalTok{truncated_}\DecValTok{1}\NormalTok{_r <-}\StringTok{ }\KeywordTok{as.numeric}\NormalTok{(trnc_y_scale_r}\OperatorTok{$}\NormalTok{trn_}\DecValTok{1}\NormalTok{)}

\NormalTok{control_}\DecValTok{1}\NormalTok{_py <-}\StringTok{ }\NormalTok{ctrl_y_scale_py}\OperatorTok{$}\NormalTok{con_}\DecValTok{1}
\NormalTok{control_}\DecValTok{1}\NormalTok{_py[}\KeywordTok{which}\NormalTok{(control_}\DecValTok{1}\NormalTok{_py }\OperatorTok{==}\StringTok{ "41/42"}\NormalTok{)] <-}\StringTok{ }\FloatTok{41.5}
\NormalTok{control_}\DecValTok{1}\NormalTok{_py <-}\StringTok{ }\KeywordTok{na.exclude}\NormalTok{(}\KeywordTok{as.numeric}\NormalTok{(control_}\DecValTok{1}\NormalTok{_py))}

\NormalTok{truncated_}\DecValTok{1}\NormalTok{_py <-}\StringTok{ }\KeywordTok{as.numeric}\NormalTok{(trnc_y_scale_py}\OperatorTok{$}\NormalTok{trn_}\DecValTok{1}\NormalTok{)}

\NormalTok{brks <-}\StringTok{ }\KeywordTok{c}\NormalTok{(}\StringTok{"Control"}\NormalTok{, }\StringTok{"Truncated"}\NormalTok{)}
\NormalTok{vals <-}\StringTok{ }\KeywordTok{c}\NormalTok{(}\StringTok{"#1c9e77"}\NormalTok{, }\StringTok{"#d95f02"}\NormalTok{)}

\NormalTok{p<-}\KeywordTok{ggplot}\NormalTok{() }\OperatorTok{+}
\StringTok{  }\KeywordTok{geom_density}\NormalTok{(}\DataTypeTok{data =} \KeywordTok{as.data.frame}\NormalTok{(control_}\DecValTok{1}\NormalTok{_r), }\KeywordTok{aes}\NormalTok{(}\DataTypeTok{x=}\NormalTok{control_}\DecValTok{1}\NormalTok{_r, }\DataTypeTok{col =} \StringTok{"Control"}\NormalTok{))}\OperatorTok{+}
\StringTok{  }\KeywordTok{geom_density}\NormalTok{(}\DataTypeTok{data =} \KeywordTok{as.data.frame}\NormalTok{(truncated_}\DecValTok{1}\NormalTok{_r), }\KeywordTok{aes}\NormalTok{(}\DataTypeTok{x=}\NormalTok{truncated_}\DecValTok{1}\NormalTok{_r, }\DataTypeTok{col =} \StringTok{"Truncated"}\NormalTok{))}\OperatorTok{+}
\StringTok{  }\KeywordTok{labs}\NormalTok{(}\DataTypeTok{x=}\StringTok{"Response"}\NormalTok{, }\DataTypeTok{y=}\StringTok{"Density"}\NormalTok{)}\OperatorTok{+}
\StringTok{  }\KeywordTok{scale_colour_manual}\NormalTok{(}\DataTypeTok{name =} \StringTok{" "}\NormalTok{, }\DataTypeTok{breaks =}\NormalTok{ brks, }\DataTypeTok{values =}\NormalTok{ vals)}\OperatorTok{+}
\StringTok{  }\KeywordTok{theme}\NormalTok{(}\DataTypeTok{title =} \StringTok{"R Subgroup"}\NormalTok{)}\OperatorTok{+}
\StringTok{  }\KeywordTok{theme_classic}\NormalTok{()}


\NormalTok{q<-}\KeywordTok{ggplot}\NormalTok{() }\OperatorTok{+}
\StringTok{  }\KeywordTok{geom_density}\NormalTok{(}\DataTypeTok{data =} \KeywordTok{as.data.frame}\NormalTok{(control_}\DecValTok{1}\NormalTok{_py), }\KeywordTok{aes}\NormalTok{(}\DataTypeTok{x=}\NormalTok{control_}\DecValTok{1}\NormalTok{_py, }\DataTypeTok{col =} \StringTok{"Control"}\NormalTok{))}\OperatorTok{+}
\StringTok{  }\KeywordTok{geom_density}\NormalTok{(}\DataTypeTok{data =} \KeywordTok{as.data.frame}\NormalTok{(truncated_}\DecValTok{1}\NormalTok{_py), }\KeywordTok{aes}\NormalTok{(}\DataTypeTok{x=}\NormalTok{truncated_}\DecValTok{1}\NormalTok{_py, }\DataTypeTok{col =} \StringTok{"Truncated"}\NormalTok{))}\OperatorTok{+}
\StringTok{  }\KeywordTok{labs}\NormalTok{(}\DataTypeTok{x=}\StringTok{"Response"}\NormalTok{, }\DataTypeTok{y=}\StringTok{"Density"}\NormalTok{)}\OperatorTok{+}
\StringTok{  }\KeywordTok{scale_colour_manual}\NormalTok{(}\DataTypeTok{name =} \StringTok{" "}\NormalTok{, }\DataTypeTok{breaks =}\NormalTok{ brks, }\DataTypeTok{values =}\NormalTok{ vals)}\OperatorTok{+}
\StringTok{  }\KeywordTok{theme}\NormalTok{(}\DataTypeTok{title =} \StringTok{"Python Subgroup"}\NormalTok{)}\OperatorTok{+}
\StringTok{  }\KeywordTok{theme_classic}\NormalTok{()}
\end{Highlighting}
\end{Shaded}

\begin{Shaded}
\begin{Highlighting}[]
\KeywordTok{plot_grid}\NormalTok{(p, q)}
\end{Highlighting}
\end{Shaded}

\begin{figure}
\centering
\includegraphics{main_files/figure-latex/unnamed-chunk-9-1.pdf}
\caption{Density plot showing distributions of responses regarding the
control and truncated plots for the question 1}
\end{figure}

The distributions for the control and truncated plot responses for the R
subgroup are fairly similar to the whole population, although the peaks
for the logarithmic plot responses are marginally lower. The
distribution of the truncated plots is unexpected from lokking at the
numbers, and more `chaotic'. This shows potentially more variation in
the responses.

For question 2 it is similarly seen that the language used does not have
a statistically significant impact on the response for the truncated
plot, with means \(5.500\) and \(5.187\), and medians \(6\) and \(5\)
respectively for R and Python for the control plot, and means \(5.98\)
and \(5.84\) both with median \(6\) for the truncated. Comparative
testing with MWW gives \(p=0.2199\) for the control plot and \(0.9105\)
for the truncated. Thus, the scale granulation or any other differing
aspect of the plots does not seem to have a significant effect. See
figure{[}?{]} for the distributions.

\begin{Shaded}
\begin{Highlighting}[]
\NormalTok{control_}\DecValTok{2}\NormalTok{_r <-}\StringTok{ }\NormalTok{ctrl_y_scale_r}\OperatorTok{$}\NormalTok{con_}\DecValTok{2}
\NormalTok{truncated_}\DecValTok{2}\NormalTok{_r <-}\StringTok{ }\NormalTok{trnc_y_scale_r}\OperatorTok{$}\NormalTok{trn_}\DecValTok{2}
\NormalTok{resp <-}\StringTok{ }\KeywordTok{c}\NormalTok{(control_}\DecValTok{2}\NormalTok{_r, truncated_}\DecValTok{2}\NormalTok{_r)}
\NormalTok{type <-}\StringTok{ }\KeywordTok{c}\NormalTok{(}\KeywordTok{rep}\NormalTok{(}\StringTok{'Control'}\NormalTok{, }\DecValTok{38}\NormalTok{), }\KeywordTok{rep}\NormalTok{(}\StringTok{'Truncated'}\NormalTok{, }\DecValTok{38}\NormalTok{)) }
\NormalTok{stats_}\DecValTok{2}\NormalTok{_r <-}\StringTok{ }\KeywordTok{data.frame}\NormalTok{(resp, type)}

\NormalTok{control_}\DecValTok{2}\NormalTok{_py <-}\StringTok{ }\NormalTok{ctrl_y_scale_py}\OperatorTok{$}\NormalTok{con_}\DecValTok{2}
\NormalTok{truncated_}\DecValTok{2}\NormalTok{_py <-}\StringTok{ }\NormalTok{trnc_y_scale_py}\OperatorTok{$}\NormalTok{trn_}\DecValTok{2}
\NormalTok{resp <-}\StringTok{ }\KeywordTok{c}\NormalTok{(control_}\DecValTok{2}\NormalTok{_py, truncated_}\DecValTok{2}\NormalTok{_py)}
\NormalTok{type <-}\StringTok{ }\KeywordTok{c}\NormalTok{(}\KeywordTok{rep}\NormalTok{(}\StringTok{'Control'}\NormalTok{, }\DecValTok{32}\NormalTok{), }\KeywordTok{rep}\NormalTok{(}\StringTok{'Truncated'}\NormalTok{, }\DecValTok{32}\NormalTok{)) }
\NormalTok{stats_}\DecValTok{2}\NormalTok{_py <-}\StringTok{ }\KeywordTok{data.frame}\NormalTok{(resp, type)}

\NormalTok{p <-}\StringTok{ }\KeywordTok{ggplot}\NormalTok{(stats_}\DecValTok{2}\NormalTok{_r)}\OperatorTok{+}
\StringTok{  }\KeywordTok{geom_bar}\NormalTok{(}\KeywordTok{aes}\NormalTok{(}\DataTypeTok{x=}\NormalTok{resp, }\DataTypeTok{group =}\NormalTok{ type, }\DataTypeTok{fill =}\NormalTok{ type), }\DataTypeTok{position =} \StringTok{'dodge'}\NormalTok{) }\OperatorTok{+}
\StringTok{  }\KeywordTok{ylab}\NormalTok{(}\StringTok{"Number of Respondents"}\NormalTok{)}\OperatorTok{+}
\StringTok{  }\KeywordTok{xlab}\NormalTok{(}\StringTok{"Response"}\NormalTok{)}\OperatorTok{+}
\StringTok{  }\KeywordTok{labs}\NormalTok{(}\DataTypeTok{title =} \StringTok{"R Subgroup"}\NormalTok{)}\OperatorTok{+}
\StringTok{  }\KeywordTok{scale_fill_brewer}\NormalTok{(}\DataTypeTok{palette=}\StringTok{"Dark2"}\NormalTok{, }\DataTypeTok{labels =} \KeywordTok{c}\NormalTok{(}\StringTok{'Control'}\NormalTok{, }\StringTok{'Truncated'}\NormalTok{))}\OperatorTok{+}
\StringTok{  }\KeywordTok{scale_x_discrete}\NormalTok{(}\DataTypeTok{breaks =} \KeywordTok{c}\NormalTok{(}\StringTok{"1"}\NormalTok{, }\StringTok{"2"}\NormalTok{, }\StringTok{"3"}\NormalTok{, }\StringTok{"4"}\NormalTok{, }\StringTok{"5"}\NormalTok{, }\StringTok{"6"}\NormalTok{, }\StringTok{"7"}\NormalTok{), }\DataTypeTok{labels =} \KeywordTok{c}\NormalTok{(}\StringTok{"1"}\NormalTok{, }\StringTok{"2"}\NormalTok{, }\StringTok{"3"}\NormalTok{, }\StringTok{"4"}\NormalTok{, }\StringTok{"5"}\NormalTok{, }\StringTok{"6"}\NormalTok{, }\StringTok{"7"}\NormalTok{), }\DataTypeTok{limits =} \KeywordTok{c}\NormalTok{(}\StringTok{"1"}\NormalTok{, }\StringTok{"2"}\NormalTok{, }\StringTok{"3"}\NormalTok{, }\StringTok{"4"}\NormalTok{, }\StringTok{"5"}\NormalTok{, }\StringTok{"6"}\NormalTok{, }\StringTok{"7"}\NormalTok{))}\OperatorTok{+}
\StringTok{  }\KeywordTok{theme}\NormalTok{(}\DataTypeTok{plot.title =} \KeywordTok{element_text}\NormalTok{(}\DataTypeTok{size=}\DecValTok{10}\NormalTok{))}\OperatorTok{+}
\StringTok{  }\KeywordTok{theme_classic}\NormalTok{()}

\NormalTok{q <-}\StringTok{ }\KeywordTok{ggplot}\NormalTok{(stats_}\DecValTok{2}\NormalTok{_py)}\OperatorTok{+}
\StringTok{  }\KeywordTok{geom_bar}\NormalTok{(}\KeywordTok{aes}\NormalTok{(}\DataTypeTok{x=}\NormalTok{resp, }\DataTypeTok{group =}\NormalTok{ type, }\DataTypeTok{fill =}\NormalTok{ type), }\DataTypeTok{position =} \StringTok{'dodge'}\NormalTok{) }\OperatorTok{+}
\StringTok{  }\KeywordTok{ylab}\NormalTok{(}\StringTok{"Number of Respondents"}\NormalTok{)}\OperatorTok{+}
\StringTok{  }\KeywordTok{xlab}\NormalTok{(}\StringTok{"Response"}\NormalTok{)}\OperatorTok{+}
\StringTok{  }\KeywordTok{labs}\NormalTok{(}\DataTypeTok{title =} \StringTok{"Python Subgroup"}\NormalTok{)}\OperatorTok{+}
\StringTok{  }\KeywordTok{scale_fill_brewer}\NormalTok{(}\DataTypeTok{palette=}\StringTok{"Dark2"}\NormalTok{, }\DataTypeTok{labels =} \KeywordTok{c}\NormalTok{(}\StringTok{'Control'}\NormalTok{, }\StringTok{'Truncated'}\NormalTok{))}\OperatorTok{+}
\StringTok{  }\KeywordTok{scale_x_discrete}\NormalTok{(}\DataTypeTok{breaks =} \KeywordTok{c}\NormalTok{(}\StringTok{"1"}\NormalTok{, }\StringTok{"2"}\NormalTok{, }\StringTok{"3"}\NormalTok{, }\StringTok{"4"}\NormalTok{, }\StringTok{"5"}\NormalTok{, }\StringTok{"6"}\NormalTok{, }\StringTok{"7"}\NormalTok{), }\DataTypeTok{labels =} \KeywordTok{c}\NormalTok{(}\StringTok{"1"}\NormalTok{, }\StringTok{"2"}\NormalTok{, }\StringTok{"3"}\NormalTok{, }\StringTok{"4"}\NormalTok{, }\StringTok{"5"}\NormalTok{, }\StringTok{"6"}\NormalTok{, }\StringTok{"7"}\NormalTok{), }\DataTypeTok{limits =} \KeywordTok{c}\NormalTok{(}\StringTok{"1"}\NormalTok{, }\StringTok{"2"}\NormalTok{, }\StringTok{"3"}\NormalTok{, }\StringTok{"4"}\NormalTok{, }\StringTok{"5"}\NormalTok{, }\StringTok{"6"}\NormalTok{, }\StringTok{"7"}\NormalTok{))}\OperatorTok{+}
\StringTok{  }\KeywordTok{theme}\NormalTok{(}\DataTypeTok{plot.title =} \KeywordTok{element_text}\NormalTok{(}\DataTypeTok{size=}\DecValTok{10}\NormalTok{))}\OperatorTok{+}
\StringTok{  }\KeywordTok{theme_classic}\NormalTok{()}
\end{Highlighting}
\end{Shaded}

\begin{Shaded}
\begin{Highlighting}[]
\KeywordTok{plot_grid}\NormalTok{(p, q)}
\end{Highlighting}
\end{Shaded}

\begin{figure}
\centering
\includegraphics{main_files/figure-latex/unnamed-chunk-11-1.pdf}
\caption{Bar plot showing distributions of responses regarding the
control and truncated plots for question 2, for the R and Python
subgroups}
\end{figure}

For question 3, it is again seen that the responses in relation to the R
version of truncated plot (\(\bar{x} = 3.76\), \(\tilde{x} = 4\)) do not
differ significantly to those related to the Python version
(\(\bar{x} = 3.78\), \(\tilde{x} = 4\)), with a two sample MWW p-value
of 0.9708. Similarly the control plot, there is little difference
between the R (\(\bar{x} = 3.342\), \(\tilde{x} = 3\)) and the Python
(\(\bar{x} = 2.87\), \(\tilde{x} = 3\)) versions of the plot, again with
am MWW p-value 0f 0.1465.

\begin{Shaded}
\begin{Highlighting}[]
\NormalTok{control_}\DecValTok{3}\NormalTok{_r <-}\StringTok{ }\NormalTok{ctrl_y_scale_r}\OperatorTok{$}\NormalTok{con_}\DecValTok{3}
\NormalTok{truncated_}\DecValTok{3}\NormalTok{_r <-}\StringTok{ }\NormalTok{trnc_y_scale_r}\OperatorTok{$}\NormalTok{trn_}\DecValTok{3}
\NormalTok{resp <-}\StringTok{ }\KeywordTok{c}\NormalTok{(control_}\DecValTok{3}\NormalTok{_r, truncated_}\DecValTok{3}\NormalTok{_r)}
\NormalTok{type <-}\StringTok{ }\KeywordTok{c}\NormalTok{(}\KeywordTok{rep}\NormalTok{(}\StringTok{'Control'}\NormalTok{, }\DecValTok{38}\NormalTok{), }\KeywordTok{rep}\NormalTok{(}\StringTok{'Truncated'}\NormalTok{, }\DecValTok{38}\NormalTok{)) }
\NormalTok{stats_}\DecValTok{3}\NormalTok{_r <-}\StringTok{ }\KeywordTok{data.frame}\NormalTok{(resp, type)}

\NormalTok{control_}\DecValTok{3}\NormalTok{_py <-}\StringTok{ }\NormalTok{ctrl_y_scale_py}\OperatorTok{$}\NormalTok{con_}\DecValTok{3}
\NormalTok{truncated_}\DecValTok{3}\NormalTok{_py <-}\StringTok{ }\NormalTok{trnc_y_scale_py}\OperatorTok{$}\NormalTok{trn_}\DecValTok{3}
\NormalTok{resp <-}\StringTok{ }\KeywordTok{c}\NormalTok{(control_}\DecValTok{3}\NormalTok{_py, truncated_}\DecValTok{3}\NormalTok{_py)}
\NormalTok{type <-}\StringTok{ }\KeywordTok{c}\NormalTok{(}\KeywordTok{rep}\NormalTok{(}\StringTok{'Control'}\NormalTok{, }\DecValTok{32}\NormalTok{), }\KeywordTok{rep}\NormalTok{(}\StringTok{'Truncated'}\NormalTok{, }\DecValTok{32}\NormalTok{)) }
\NormalTok{stats_}\DecValTok{3}\NormalTok{_py <-}\StringTok{ }\KeywordTok{data.frame}\NormalTok{(resp, type)}

\NormalTok{p <-}\StringTok{ }\KeywordTok{ggplot}\NormalTok{(stats_}\DecValTok{3}\NormalTok{_r)}\OperatorTok{+}
\StringTok{  }\KeywordTok{geom_bar}\NormalTok{(}\KeywordTok{aes}\NormalTok{(}\DataTypeTok{x=}\NormalTok{resp, }\DataTypeTok{group =}\NormalTok{ type, }\DataTypeTok{fill =}\NormalTok{ type), }\DataTypeTok{position =} \StringTok{'dodge'}\NormalTok{) }\OperatorTok{+}
\StringTok{  }\KeywordTok{ylab}\NormalTok{(}\StringTok{"Number of Respondents"}\NormalTok{)}\OperatorTok{+}
\StringTok{  }\KeywordTok{xlab}\NormalTok{(}\StringTok{"Response"}\NormalTok{)}\OperatorTok{+}
\StringTok{  }\KeywordTok{labs}\NormalTok{(}\DataTypeTok{title =} \StringTok{"R Subgroup"}\NormalTok{)}\OperatorTok{+}
\StringTok{  }\KeywordTok{scale_fill_brewer}\NormalTok{(}\DataTypeTok{palette=}\StringTok{"Dark2"}\NormalTok{, }\DataTypeTok{labels =} \KeywordTok{c}\NormalTok{(}\StringTok{'Control'}\NormalTok{, }\StringTok{'Truncated'}\NormalTok{))}\OperatorTok{+}
\StringTok{  }\KeywordTok{scale_x_discrete}\NormalTok{(}\DataTypeTok{breaks =} \KeywordTok{c}\NormalTok{(}\StringTok{"1"}\NormalTok{, }\StringTok{"2"}\NormalTok{, }\StringTok{"3"}\NormalTok{, }\StringTok{"4"}\NormalTok{, }\StringTok{"5"}\NormalTok{, }\StringTok{"6"}\NormalTok{, }\StringTok{"7"}\NormalTok{), }\DataTypeTok{labels =} \KeywordTok{c}\NormalTok{(}\StringTok{"1"}\NormalTok{, }\StringTok{"2"}\NormalTok{, }\StringTok{"3"}\NormalTok{, }\StringTok{"4"}\NormalTok{, }\StringTok{"5"}\NormalTok{, }\StringTok{"6"}\NormalTok{, }\StringTok{"7"}\NormalTok{), }\DataTypeTok{limits =} \KeywordTok{c}\NormalTok{(}\StringTok{"1"}\NormalTok{, }\StringTok{"2"}\NormalTok{, }\StringTok{"3"}\NormalTok{, }\StringTok{"4"}\NormalTok{, }\StringTok{"5"}\NormalTok{, }\StringTok{"6"}\NormalTok{, }\StringTok{"7"}\NormalTok{))}\OperatorTok{+}
\StringTok{  }\KeywordTok{theme}\NormalTok{(}\DataTypeTok{plot.title =} \KeywordTok{element_text}\NormalTok{(}\DataTypeTok{size=}\DecValTok{10}\NormalTok{))}\OperatorTok{+}
\StringTok{  }\KeywordTok{theme_classic}\NormalTok{()}

\NormalTok{q <-}\StringTok{ }\KeywordTok{ggplot}\NormalTok{(stats_}\DecValTok{3}\NormalTok{_py)}\OperatorTok{+}
\StringTok{  }\KeywordTok{geom_bar}\NormalTok{(}\KeywordTok{aes}\NormalTok{(}\DataTypeTok{x=}\NormalTok{resp, }\DataTypeTok{group =}\NormalTok{ type, }\DataTypeTok{fill =}\NormalTok{ type), }\DataTypeTok{position =} \StringTok{'dodge'}\NormalTok{) }\OperatorTok{+}
\StringTok{  }\KeywordTok{ylab}\NormalTok{(}\StringTok{"Number of Respondents"}\NormalTok{)}\OperatorTok{+}
\StringTok{  }\KeywordTok{xlab}\NormalTok{(}\StringTok{"Response"}\NormalTok{)}\OperatorTok{+}
\StringTok{  }\KeywordTok{labs}\NormalTok{(}\DataTypeTok{title =} \StringTok{"Python Subgroup"}\NormalTok{)}\OperatorTok{+}
\StringTok{  }\KeywordTok{scale_fill_brewer}\NormalTok{(}\DataTypeTok{palette=}\StringTok{"Dark2"}\NormalTok{, }\DataTypeTok{labels =} \KeywordTok{c}\NormalTok{(}\StringTok{'Control'}\NormalTok{, }\StringTok{'Truncated'}\NormalTok{))}\OperatorTok{+}
\StringTok{  }\KeywordTok{scale_x_discrete}\NormalTok{(}\DataTypeTok{breaks =} \KeywordTok{c}\NormalTok{(}\StringTok{"1"}\NormalTok{, }\StringTok{"2"}\NormalTok{, }\StringTok{"3"}\NormalTok{, }\StringTok{"4"}\NormalTok{, }\StringTok{"5"}\NormalTok{, }\StringTok{"6"}\NormalTok{, }\StringTok{"7"}\NormalTok{), }\DataTypeTok{labels =} \KeywordTok{c}\NormalTok{(}\StringTok{"1"}\NormalTok{, }\StringTok{"2"}\NormalTok{, }\StringTok{"3"}\NormalTok{, }\StringTok{"4"}\NormalTok{, }\StringTok{"5"}\NormalTok{, }\StringTok{"6"}\NormalTok{, }\StringTok{"7"}\NormalTok{), }\DataTypeTok{limits =} \KeywordTok{c}\NormalTok{(}\StringTok{"1"}\NormalTok{, }\StringTok{"2"}\NormalTok{, }\StringTok{"3"}\NormalTok{, }\StringTok{"4"}\NormalTok{, }\StringTok{"5"}\NormalTok{, }\StringTok{"6"}\NormalTok{, }\StringTok{"7"}\NormalTok{))}\OperatorTok{+}
\StringTok{  }\KeywordTok{theme}\NormalTok{(}\DataTypeTok{plot.title =} \KeywordTok{element_text}\NormalTok{(}\DataTypeTok{size=}\DecValTok{10}\NormalTok{))}\OperatorTok{+}
\StringTok{  }\KeywordTok{theme_classic}\NormalTok{()}
\end{Highlighting}
\end{Shaded}

\begin{Shaded}
\begin{Highlighting}[]
\KeywordTok{plot_grid}\NormalTok{(p, q)}
\end{Highlighting}
\end{Shaded}

\begin{figure}
\centering
\includegraphics{main_files/figure-latex/unnamed-chunk-13-1.pdf}
\caption{Bar plot showing distributions of responses regarding the
control and truncated plots for question 3, for the R and Python
subgroups}
\end{figure}

Figure{[}?{]} shows both distributions, with the R appearing more
positively skewed and the python looking fairly symmetric for both plot
types, which was also found when performing symmetry tests. For the
Python it can also easily be seen that the bars for the truncated plot
responses seems `shifted' to the right slightly as compared to the
control.

Now considering subsetting for the respondents that saw the truncated
plot first out of the three. Note that 25 saw the control plot first and
23 saw the truncated plot first.

The distribution of responses for the truncated plot in question 1 shows
a slightly higher mean (\(41.696\)) and median (\(41.25\)) than for the
whole population, but a MWW test shows that the difference is not
significant (\(p=0.1379\)). Similarly for questions 2 and 3, performing
tests on the truncated plot for respondents who saw this first as
compared to the truncated plot responses for the whole population result
in p-values of \(0.2614\) and \(0.3145\), providing evidence that the
plot order doesn't have much of an impact on perception for the
truncated plot.

The conclusions appear to be consistent with results from the @YANG2021
paper, in which the researchers, similar to this survey, showed
participants a series of control bar plots alongside those with a
truncated axis, and concluded that the difference in values for the
truncated axis were perceived to be larger than those of the control
plots.

\subsection{Effect of Logarithmic Scaling}

\begin{Shaded}
\begin{Highlighting}[]
\NormalTok{control_}\DecValTok{1}\NormalTok{ <-}\StringTok{ }\NormalTok{ctrl_y_scale}\OperatorTok{$}\NormalTok{con_}\DecValTok{1}
\NormalTok{control_}\DecValTok{1}\NormalTok{[}\KeywordTok{which}\NormalTok{(control_}\DecValTok{1} \OperatorTok{==}\StringTok{ "41/42"}\NormalTok{)] <-}\StringTok{ }\CommentTok{# take midpoint of two values}
\NormalTok{control_}\DecValTok{1}\NormalTok{ <-}\StringTok{ }\KeywordTok{na.exclude}\NormalTok{(}\KeywordTok{as.numeric}\NormalTok{(control_}\DecValTok{1}\NormalTok{))}
\end{Highlighting}
\end{Shaded}

\begin{verbatim}
## Warning in na.exclude(as.numeric(control_1)): NAs introduced by coercion
\end{verbatim}

\begin{Shaded}
\begin{Highlighting}[]
\NormalTok{truncated_}\DecValTok{1}\NormalTok{ <-}\StringTok{ }\KeywordTok{as.numeric}\NormalTok{(trnc_y_scale}\OperatorTok{$}\NormalTok{trn_}\DecValTok{1}\NormalTok{)}

\NormalTok{logarithmic_}\DecValTok{1}\NormalTok{ <-}\StringTok{ }\NormalTok{log_y_scale}\OperatorTok{$}\NormalTok{log_}\DecValTok{1}
\NormalTok{logarithmic_}\DecValTok{1}\NormalTok{[}\KeywordTok{which}\NormalTok{(logarithmic_}\DecValTok{1} \OperatorTok{==}\StringTok{ "Don't know"}\NormalTok{)] <-}\StringTok{ }\OtherTok{NA}
\NormalTok{logarithmic_}\DecValTok{1}\NormalTok{[}\KeywordTok{which}\NormalTok{(logarithmic_}\DecValTok{1} \OperatorTok{==}\StringTok{ "Next to none."}\NormalTok{)] <-}\StringTok{ }\OtherTok{NA}
\NormalTok{logarithmic_}\DecValTok{1}\NormalTok{[}\KeywordTok{which}\NormalTok{(logarithmic_}\DecValTok{1} \OperatorTok{==}\StringTok{ "10^15"}\NormalTok{)] <-}\StringTok{ }\OtherTok{NA}
\NormalTok{logarithmic_}\DecValTok{1}\NormalTok{[}\KeywordTok{which}\NormalTok{(logarithmic_}\DecValTok{1} \OperatorTok{==}\StringTok{ "10^9"}\NormalTok{)] <-}\StringTok{ }\OtherTok{NA}
\NormalTok{logarithmic_}\DecValTok{1}\NormalTok{ <-}\StringTok{ }\KeywordTok{as.numeric}\NormalTok{(}\KeywordTok{na.exclude}\NormalTok{(logarithmic_}\DecValTok{1}\NormalTok{))}
\NormalTok{logarithmic_}\DecValTok{1}\NormalTok{ <-}\StringTok{ }\NormalTok{logarithmic_}\DecValTok{1}\NormalTok{[}\KeywordTok{which}\NormalTok{(logarithmic_}\DecValTok{1} \OperatorTok{>=}\StringTok{ }\FloatTok{14.25}\NormalTok{)]}
\NormalTok{logarithmic_}\DecValTok{1}\NormalTok{ <-}\StringTok{ }\NormalTok{logarithmic_}\DecValTok{1}\NormalTok{[}\KeywordTok{which}\NormalTok{(logarithmic_}\DecValTok{1} \OperatorTok{<=}\StringTok{ }\FloatTok{60.75}\NormalTok{)]}

\NormalTok{y_scale_}\DecValTok{1}\NormalTok{_all <-}\StringTok{ }\KeywordTok{cbind}\NormalTok{(control_}\DecValTok{1}\NormalTok{, truncated_}\DecValTok{1}\NormalTok{, logarithmic_}\DecValTok{1}\NormalTok{)}
\end{Highlighting}
\end{Shaded}

\begin{verbatim}
## Warning in cbind(control_1, truncated_1, logarithmic_1): number of rows of
## result is not a multiple of vector length (arg 1)
\end{verbatim}

Within the logarithmic responses, there were two invalid responses,
given as `Don't know' and `Next to none.'. These will be considered as
`NA' responses and discounted from the quantitative analysis, however
they do provide useful qualitative insights into how the respondents
reacted to the plots, particularly as both were entered for the
logarithmically scaled plot made in Python.

The mean of the responses for the logarithmically-scaled plot, on the
other hand, was magnitudes higher than the true value at 1.493e+13,
although with a median of 35; lower than the median response of the
control and truncated plots responses. The high magnitude is the result
of two answers of `10\^{}15' and `10\^{}9', both again for the python
version of the plot.

The default logarithmic scaling in Python uses standard form notation,
which perhaps the two participants who entered the high magnitude
answers were less exposed to and not as familiar with. Looking at the
degree subjects for these respondents, it is observed that they study
Social Sciences and Psychology, respectively. This could add to the idea
that they are less familiar with this notation as it is more commonly
used in mathematical and physical science disciplines. One of the
respondents also rated their numerical skills at 1/5, showing they feel
that numerical skill is not their specialty. The other rated their
numeric skills at 4/5, showing that even with a good self-perceived
level of numerical skill, standard form could be considered misleading.

This should perhaps be considered when designing visualisations; the
creator of the visualisations may find the logarithmic scale or standard
form more effective in showing the data, but they should consider the
target audience. Are the audience going to be familiar with this? If,
for example, visualisations are being published in a paper targeted at
academics in a subject likely to use such scalings often and understand
them, this may be a good way to depict the data. However, using this in
something such as an advertising campaign could mislead the public,
causing them to either over or under estimate values. As previously
discussed, however, this is often done deliberately in order to push the
message the creator wishes to sell.

The variance in the responses for the logarithmic plot is also high,
with value \(1.492 \times 10^{28}\), showing that a large amount of the
observations differ from the very high mean, and considering this
alongside the lower median may point towards many of the respondents
either giving an accurate response or even underestimating. Furthering
this point, the IQR for the logarithmic responses is the interval
\([30, 40.5]\), which sits below the true value, displaying that over
50\% of the observations in the total population actually underestimate
the value.

The distribution of responses in the R subgroup also shows on average a
slight underestimation (\(\bar{x}= 39.73\), \(\tilde{x} = 35\)) and, as
expected, vast overestimation for the Python version
(\(\bar{x}= 39.73\), \(\tilde{x} = 35\)). This shows that, with a
linearly notated logarithmic scale, the scale may cause underestimation,
but this is counteracted by using a standard form notation.

It can be considered to follow the convention of values that have value
outside the range \([Q1 - 1.5 \times IQR, Q3 + 1.5 \times IQR]\), where
\(Q1\) and \(Q3\) are the first and third quartiles, which here would be
the range \([14.25, 60.75]\) and results in a sample size of 59.
Consider now the response distribution for the logarithmically-scaled
plot, after removing these responses, for which figure{[}?{]} gives the
density plot. Both plots show the response distribution of the
outlier-removed set of responses, with the providing a comparison with
the distribution of responses relating to the control plot.

\begin{Shaded}
\begin{Highlighting}[]
\NormalTok{brks <-}\StringTok{ }\KeywordTok{c}\NormalTok{(}\StringTok{"Control"}\NormalTok{, }\StringTok{"Logarithmic"}\NormalTok{)}
\NormalTok{vals <-}\StringTok{ }\KeywordTok{c}\NormalTok{(}\StringTok{"#1c9e77"}\NormalTok{,  }\StringTok{"#7570b3"}\NormalTok{)}
\NormalTok{y_scale_}\DecValTok{1}\NormalTok{_all <-}\StringTok{ }\KeywordTok{cbind}\NormalTok{(control_}\DecValTok{1}\NormalTok{, logarithmic_}\DecValTok{1}\NormalTok{)}
\end{Highlighting}
\end{Shaded}

\begin{verbatim}
## Warning in cbind(control_1, logarithmic_1): number of rows of result is not a
## multiple of vector length (arg 2)
\end{verbatim}

\begin{Shaded}
\begin{Highlighting}[]
\NormalTok{y_scale_}\DecValTok{1}\NormalTok{_all <-}\StringTok{ }\KeywordTok{as.data.frame}\NormalTok{(y_scale_}\DecValTok{1}\NormalTok{_all)}

\NormalTok{ p <-}\StringTok{ }\KeywordTok{ggplot}\NormalTok{() }\OperatorTok{+}
\StringTok{  }\KeywordTok{geom_density}\NormalTok{(}\DataTypeTok{data =} \KeywordTok{as.data.frame}\NormalTok{(logarithmic_}\DecValTok{1}\NormalTok{), }\KeywordTok{aes}\NormalTok{(}\DataTypeTok{x=}\NormalTok{logarithmic_}\DecValTok{1}\NormalTok{, }\DataTypeTok{col =} \StringTok{"Logarithmic"}\NormalTok{))}\OperatorTok{+}
\StringTok{  }\KeywordTok{geom_density}\NormalTok{(}\DataTypeTok{data =} \KeywordTok{as.data.frame}\NormalTok{(control_}\DecValTok{1}\NormalTok{), }\KeywordTok{aes}\NormalTok{(}\DataTypeTok{x=}\NormalTok{control_}\DecValTok{1}\NormalTok{, }\DataTypeTok{col =} \StringTok{"Control"}\NormalTok{))}\OperatorTok{+}
\StringTok{  }\KeywordTok{labs}\NormalTok{(}\DataTypeTok{x=}\StringTok{"Response"}\NormalTok{, }\DataTypeTok{y=}\StringTok{"Density"}\NormalTok{)}\OperatorTok{+}
\StringTok{  }\KeywordTok{scale_colour_manual}\NormalTok{(}\DataTypeTok{name =} \StringTok{" "}\NormalTok{, }\DataTypeTok{breaks =}\NormalTok{ brks, }\DataTypeTok{values =}\NormalTok{ vals)}\OperatorTok{+}
\StringTok{  }\KeywordTok{theme_classic}\NormalTok{()}
 
\NormalTok{q <-}\StringTok{ }\KeywordTok{ggplot}\NormalTok{() }\OperatorTok{+}
\StringTok{  }\KeywordTok{geom_density}\NormalTok{(}\DataTypeTok{data =} \KeywordTok{as.data.frame}\NormalTok{(logarithmic_}\DecValTok{1}\NormalTok{), }\KeywordTok{aes}\NormalTok{(}\DataTypeTok{x=}\NormalTok{logarithmic_}\DecValTok{1}\NormalTok{, }\DataTypeTok{col =} \StringTok{"Logarithmic"}\NormalTok{))}\OperatorTok{+}
\StringTok{  }\KeywordTok{labs}\NormalTok{(}\DataTypeTok{x=}\StringTok{"Response"}\NormalTok{, }\DataTypeTok{y=}\StringTok{"Density"}\NormalTok{)}\OperatorTok{+}
\StringTok{  }\KeywordTok{scale_colour_manual}\NormalTok{(}\DataTypeTok{name =} \StringTok{" "}\NormalTok{, }\DataTypeTok{breaks =}\NormalTok{ brks, }\DataTypeTok{values =}\NormalTok{ vals)}\OperatorTok{+}
\StringTok{  }\KeywordTok{theme_classic}\NormalTok{()}
\end{Highlighting}
\end{Shaded}

\begin{Shaded}
\begin{Highlighting}[]
\KeywordTok{plot_grid}\NormalTok{(p, q)}
\end{Highlighting}
\end{Shaded}

\begin{figure}
\centering
\includegraphics{main_files/figure-latex/unnamed-chunk-16-1.pdf}
\caption{Density plot showing distributions of responses regarding the
control and logarithmic scaled plot, after removing values of greater or
equal to 1000}
\end{figure}

The Python default of standard form notation appears to have confused
certain respondents, who are perhaps not as used to seeing this
notation, and there was a large range in the responses along with one
person not even entering a number, but rather stating that they ``Don't
know'', and another stating they believed the value was ``Next to
none''. The ``Next to none'' entry is subjective, but could potentially
be be assumed as a value close to 0, once again maybe as a result of
standard form being less well known to this respondent.

The distribution of responses for question 2 is displayed in
figure{[}?{]}.

\begin{Shaded}
\begin{Highlighting}[]
\NormalTok{brks <-}\StringTok{ }\KeywordTok{c}\NormalTok{(}\StringTok{"Control"}\NormalTok{, }\StringTok{"Logarithmic"}\NormalTok{)}
\NormalTok{vals <-}\StringTok{ }\KeywordTok{c}\NormalTok{(}\StringTok{"#1c9e77"}\NormalTok{,  }\StringTok{"#7570b3"}\NormalTok{)}

\NormalTok{control_}\DecValTok{2}\NormalTok{ <-}\StringTok{ }\NormalTok{ctrl_y_scale}\OperatorTok{$}\NormalTok{con_}\DecValTok{2}
\NormalTok{logarithmic_}\DecValTok{2}\NormalTok{ <-}\StringTok{ }\NormalTok{log_y_scale}\OperatorTok{$}\NormalTok{log_}\DecValTok{2}

\NormalTok{y_scale_}\DecValTok{2}\NormalTok{_all <-}\StringTok{ }\KeywordTok{cbind}\NormalTok{(control_}\DecValTok{2}\NormalTok{, logarithmic_}\DecValTok{2}\NormalTok{)}
\NormalTok{y_scale_}\DecValTok{2}\NormalTok{_all <-}\StringTok{ }\KeywordTok{as.data.frame}\NormalTok{(y_scale_}\DecValTok{2}\NormalTok{_all)}
\NormalTok{resp <-}\StringTok{ }\KeywordTok{c}\NormalTok{(control_}\DecValTok{2}\NormalTok{, logarithmic_}\DecValTok{2}\NormalTok{)}
\NormalTok{type <-}\StringTok{ }\KeywordTok{c}\NormalTok{(}\KeywordTok{rep}\NormalTok{(}\StringTok{'Control'}\NormalTok{, }\DecValTok{70}\NormalTok{), }\KeywordTok{rep}\NormalTok{(}\StringTok{'Logarithmic'}\NormalTok{, }\DecValTok{70}\NormalTok{)) }
\NormalTok{stats_}\DecValTok{2}\NormalTok{ <-}\StringTok{ }\KeywordTok{data.frame}\NormalTok{(resp, type)}
\end{Highlighting}
\end{Shaded}

\begin{Shaded}
\begin{Highlighting}[]
\KeywordTok{ggplot}\NormalTok{(stats_}\DecValTok{2}\NormalTok{)}\OperatorTok{+}
\StringTok{  }\KeywordTok{geom_bar}\NormalTok{(}\KeywordTok{aes}\NormalTok{(}\DataTypeTok{x=}\NormalTok{resp, }\DataTypeTok{group =}\NormalTok{ type, }\DataTypeTok{fill =}\NormalTok{ type), }\DataTypeTok{position =} \StringTok{'dodge'}\NormalTok{) }\OperatorTok{+}
\StringTok{  }\KeywordTok{ylab}\NormalTok{(}\StringTok{"Number of Respondents"}\NormalTok{)}\OperatorTok{+}
\StringTok{  }\KeywordTok{xlab}\NormalTok{(}\StringTok{"Response"}\NormalTok{)}\OperatorTok{+}
\StringTok{  }\KeywordTok{labs}\NormalTok{(}\DataTypeTok{title =} \StringTok{"Responses selected Over the Whole Population"}\NormalTok{)}\OperatorTok{+}
\StringTok{  }\KeywordTok{scale_fill_manual}\NormalTok{(}\DataTypeTok{values=}\NormalTok{vals, }\DataTypeTok{breaks =} \KeywordTok{c}\NormalTok{(}\StringTok{'Control'}\NormalTok{, }\StringTok{'Logarithmic'}\NormalTok{))}\OperatorTok{+}
\StringTok{  }\KeywordTok{scale_x_discrete}\NormalTok{(}\DataTypeTok{breaks =} \KeywordTok{c}\NormalTok{(}\StringTok{"1"}\NormalTok{, }\StringTok{"2"}\NormalTok{, }\StringTok{"3"}\NormalTok{, }\StringTok{"4"}\NormalTok{, }\StringTok{"5"}\NormalTok{, }\StringTok{"6"}\NormalTok{, }\StringTok{"7"}\NormalTok{), }\DataTypeTok{labels =} \KeywordTok{c}\NormalTok{(}\StringTok{"1"}\NormalTok{, }\StringTok{"2"}\NormalTok{, }\StringTok{"3"}\NormalTok{, }\StringTok{"4"}\NormalTok{, }\StringTok{"5"}\NormalTok{, }\StringTok{"6"}\NormalTok{, }\StringTok{"7"}\NormalTok{), }\DataTypeTok{limits =} \KeywordTok{c}\NormalTok{(}\StringTok{"1"}\NormalTok{, }\StringTok{"2"}\NormalTok{, }\StringTok{"3"}\NormalTok{, }\StringTok{"4"}\NormalTok{, }\StringTok{"5"}\NormalTok{, }\StringTok{"6"}\NormalTok{, }\StringTok{"7"}\NormalTok{))}\OperatorTok{+}
\StringTok{  }\KeywordTok{theme_classic}\NormalTok{()}
\end{Highlighting}
\end{Shaded}

\begin{figure}
\centering
\includegraphics{main_files/figure-latex/unnamed-chunk-18-1.pdf}
\caption{Bar plot showing distributions of responses regarding the
control and logarithmic plots for question 2}
\end{figure}

The spread of logarithmic plot values is fairly wide, with at least one
response for each option, and the control is the same as stated before.
The plot depicts how there is a wide spread of values, with some
respondents having very different subjective views of the size of the
difference to others. On average, the subjective perceived difference in
bar heights was significantly lower for the logarithmic plot responses
(\(\bar{x}=3.67\), \(\tilde{x}=3.5\)) than for the control
(\(\bar{x}=3.35\), \(\tilde{x}=5\)). This is evidenced by a one-sided
sign test with the alternative hypothesis that the logarithmic plot
responses are on average lower than the control plot responses.

There is evidence to show that the difference between the R and Python
versions of the logarithmic plot is significant (\(p=0.00096\),
\(\bar{x}_R = 4.263\), \(\bar{x}_{Py} = 2.969\)). The distributions for
the two language subsets are shown in figure{[}?{]}.

\begin{Shaded}
\begin{Highlighting}[]
\NormalTok{brks <-}\StringTok{ }\KeywordTok{c}\NormalTok{(}\StringTok{"Control"}\NormalTok{, }\StringTok{"Logarithmic"}\NormalTok{)}
\NormalTok{vals <-}\StringTok{ }\KeywordTok{c}\NormalTok{(}\StringTok{"#1c9e77"}\NormalTok{,  }\StringTok{"#7570b3"}\NormalTok{)}

\NormalTok{control_}\DecValTok{2}\NormalTok{ <-}\StringTok{ }\NormalTok{ctrl_y_scale_r}\OperatorTok{$}\NormalTok{con_}\DecValTok{2}
\NormalTok{logarithmic_}\DecValTok{2}\NormalTok{ <-}\StringTok{ }\NormalTok{log_y_scale_r}\OperatorTok{$}\NormalTok{log_}\DecValTok{2}

\NormalTok{y_scale_}\DecValTok{2}\NormalTok{_all <-}\StringTok{ }\KeywordTok{cbind}\NormalTok{(control_}\DecValTok{2}\NormalTok{, logarithmic_}\DecValTok{2}\NormalTok{)}
\NormalTok{y_scale_}\DecValTok{2}\NormalTok{_all <-}\StringTok{ }\KeywordTok{as.data.frame}\NormalTok{(y_scale_}\DecValTok{2}\NormalTok{_all)}
\NormalTok{resp <-}\StringTok{ }\KeywordTok{c}\NormalTok{(control_}\DecValTok{2}\NormalTok{, logarithmic_}\DecValTok{2}\NormalTok{)}
\NormalTok{type <-}\StringTok{ }\KeywordTok{c}\NormalTok{(}\KeywordTok{rep}\NormalTok{(}\StringTok{'Control'}\NormalTok{, }\DecValTok{38}\NormalTok{), }\KeywordTok{rep}\NormalTok{(}\StringTok{'Logarithmic'}\NormalTok{, }\DecValTok{38}\NormalTok{)) }
\NormalTok{stats_}\DecValTok{2}\NormalTok{ <-}\StringTok{ }\KeywordTok{data.frame}\NormalTok{(resp, type)}

\NormalTok{p <-}\StringTok{ }\KeywordTok{ggplot}\NormalTok{(stats_}\DecValTok{2}\NormalTok{)}\OperatorTok{+}
\StringTok{  }\KeywordTok{geom_bar}\NormalTok{(}\KeywordTok{aes}\NormalTok{(}\DataTypeTok{x=}\NormalTok{resp, }\DataTypeTok{group =}\NormalTok{ type, }\DataTypeTok{fill =}\NormalTok{ type), }\DataTypeTok{position =} \StringTok{'dodge'}\NormalTok{) }\OperatorTok{+}
\StringTok{  }\KeywordTok{ylab}\NormalTok{(}\StringTok{"Number of Respondents"}\NormalTok{)}\OperatorTok{+}
\StringTok{  }\KeywordTok{xlab}\NormalTok{(}\StringTok{"Response"}\NormalTok{)}\OperatorTok{+}
\StringTok{  }\KeywordTok{labs}\NormalTok{(}\DataTypeTok{title =} \StringTok{"Responses selected Over the R Subgroup"}\NormalTok{)}\OperatorTok{+}
\StringTok{  }\KeywordTok{scale_fill_manual}\NormalTok{(}\DataTypeTok{values=}\NormalTok{vals, }\DataTypeTok{breaks =} \KeywordTok{c}\NormalTok{(}\StringTok{'Control'}\NormalTok{, }\StringTok{'Logarithmic'}\NormalTok{))}\OperatorTok{+}
\StringTok{  }\KeywordTok{scale_x_discrete}\NormalTok{(}\DataTypeTok{breaks =} \KeywordTok{c}\NormalTok{(}\StringTok{"1"}\NormalTok{, }\StringTok{"2"}\NormalTok{, }\StringTok{"3"}\NormalTok{, }\StringTok{"4"}\NormalTok{, }\StringTok{"5"}\NormalTok{, }\StringTok{"6"}\NormalTok{, }\StringTok{"7"}\NormalTok{), }\DataTypeTok{labels =} \KeywordTok{c}\NormalTok{(}\StringTok{"1"}\NormalTok{, }\StringTok{"2"}\NormalTok{, }\StringTok{"3"}\NormalTok{, }\StringTok{"4"}\NormalTok{, }\StringTok{"5"}\NormalTok{, }\StringTok{"6"}\NormalTok{, }\StringTok{"7"}\NormalTok{), }\DataTypeTok{limits =} \KeywordTok{c}\NormalTok{(}\StringTok{"1"}\NormalTok{, }\StringTok{"2"}\NormalTok{, }\StringTok{"3"}\NormalTok{, }\StringTok{"4"}\NormalTok{, }\StringTok{"5"}\NormalTok{, }\StringTok{"6"}\NormalTok{, }\StringTok{"7"}\NormalTok{))}\OperatorTok{+}
\StringTok{  }\KeywordTok{theme_classic}\NormalTok{()}

\NormalTok{control_}\DecValTok{2}\NormalTok{ <-}\StringTok{ }\NormalTok{ctrl_y_scale_py}\OperatorTok{$}\NormalTok{con_}\DecValTok{2}
\NormalTok{logarithmic_}\DecValTok{2}\NormalTok{ <-}\StringTok{ }\NormalTok{log_y_scale_py}\OperatorTok{$}\NormalTok{log_}\DecValTok{2}

\NormalTok{y_scale_}\DecValTok{2}\NormalTok{_all <-}\StringTok{ }\KeywordTok{cbind}\NormalTok{(control_}\DecValTok{2}\NormalTok{, logarithmic_}\DecValTok{2}\NormalTok{)}
\NormalTok{y_scale_}\DecValTok{2}\NormalTok{_all <-}\StringTok{ }\KeywordTok{as.data.frame}\NormalTok{(y_scale_}\DecValTok{2}\NormalTok{_all)}
\NormalTok{resp <-}\StringTok{ }\KeywordTok{c}\NormalTok{(control_}\DecValTok{2}\NormalTok{, logarithmic_}\DecValTok{2}\NormalTok{)}
\NormalTok{type <-}\StringTok{ }\KeywordTok{c}\NormalTok{(}\KeywordTok{rep}\NormalTok{(}\StringTok{'Control'}\NormalTok{, }\DecValTok{32}\NormalTok{), }\KeywordTok{rep}\NormalTok{(}\StringTok{'Logarithmic'}\NormalTok{, }\DecValTok{32}\NormalTok{)) }
\NormalTok{stats_}\DecValTok{2}\NormalTok{ <-}\StringTok{ }\KeywordTok{data.frame}\NormalTok{(resp, type)}

\NormalTok{q <-}\StringTok{ }\KeywordTok{ggplot}\NormalTok{(stats_}\DecValTok{2}\NormalTok{)}\OperatorTok{+}
\StringTok{  }\KeywordTok{geom_bar}\NormalTok{(}\KeywordTok{aes}\NormalTok{(}\DataTypeTok{x=}\NormalTok{resp, }\DataTypeTok{group =}\NormalTok{ type, }\DataTypeTok{fill =}\NormalTok{ type), }\DataTypeTok{position =} \StringTok{'dodge'}\NormalTok{) }\OperatorTok{+}
\StringTok{  }\KeywordTok{ylab}\NormalTok{(}\StringTok{"Number of Respondents"}\NormalTok{)}\OperatorTok{+}
\StringTok{  }\KeywordTok{xlab}\NormalTok{(}\StringTok{"Response"}\NormalTok{)}\OperatorTok{+}
\StringTok{  }\KeywordTok{labs}\NormalTok{(}\DataTypeTok{title =} \StringTok{"Responses selected Over the Python Subgroup"}\NormalTok{)}\OperatorTok{+}
\StringTok{  }\KeywordTok{scale_fill_manual}\NormalTok{(}\DataTypeTok{values=}\NormalTok{vals, }\DataTypeTok{breaks =} \KeywordTok{c}\NormalTok{(}\StringTok{'Control'}\NormalTok{, }\StringTok{'Logarithmic'}\NormalTok{))}\OperatorTok{+}
\StringTok{  }\KeywordTok{scale_x_discrete}\NormalTok{(}\DataTypeTok{breaks =} \KeywordTok{c}\NormalTok{(}\StringTok{"1"}\NormalTok{, }\StringTok{"2"}\NormalTok{, }\StringTok{"3"}\NormalTok{, }\StringTok{"4"}\NormalTok{, }\StringTok{"5"}\NormalTok{, }\StringTok{"6"}\NormalTok{, }\StringTok{"7"}\NormalTok{), }\DataTypeTok{labels =} \KeywordTok{c}\NormalTok{(}\StringTok{"1"}\NormalTok{, }\StringTok{"2"}\NormalTok{, }\StringTok{"3"}\NormalTok{, }\StringTok{"4"}\NormalTok{, }\StringTok{"5"}\NormalTok{, }\StringTok{"6"}\NormalTok{, }\StringTok{"7"}\NormalTok{), }\DataTypeTok{limits =} \KeywordTok{c}\NormalTok{(}\StringTok{"1"}\NormalTok{, }\StringTok{"2"}\NormalTok{, }\StringTok{"3"}\NormalTok{, }\StringTok{"4"}\NormalTok{, }\StringTok{"5"}\NormalTok{, }\StringTok{"6"}\NormalTok{, }\StringTok{"7"}\NormalTok{))}\OperatorTok{+}
\StringTok{  }\KeywordTok{theme_classic}\NormalTok{()}

\KeywordTok{plot_grid}\NormalTok{(p, q)}
\end{Highlighting}
\end{Shaded}

\begin{figure}
\centering
\includegraphics{main_files/figure-latex/unnamed-chunk-19-1.pdf}
\caption{Bar plots showing distributions of responses regarding the
control and logarithmic plots for the question 2, separated by language}
\end{figure}

In regard to question 3, see again figure{[}?{]} for the plotted
distributions.

\begin{Shaded}
\begin{Highlighting}[]
\NormalTok{brks <-}\StringTok{ }\KeywordTok{c}\NormalTok{(}\StringTok{"Control"}\NormalTok{, }\StringTok{"Logarithmic"}\NormalTok{)}
\NormalTok{vals <-}\StringTok{ }\KeywordTok{c}\NormalTok{(}\StringTok{"#1c9e77"}\NormalTok{,  }\StringTok{"#7570b3"}\NormalTok{)}

\NormalTok{control_}\DecValTok{3}\NormalTok{ <-}\StringTok{ }\NormalTok{ctrl_y_scale}\OperatorTok{$}\NormalTok{con_}\DecValTok{3}
\NormalTok{logarithmic_}\DecValTok{3}\NormalTok{ <-}\StringTok{ }\NormalTok{log_y_scale}\OperatorTok{$}\NormalTok{log_}\DecValTok{3}

\NormalTok{y_scale_}\DecValTok{3}\NormalTok{_all <-}\StringTok{ }\KeywordTok{cbind}\NormalTok{(control_}\DecValTok{3}\NormalTok{, logarithmic_}\DecValTok{3}\NormalTok{)}
\NormalTok{y_scale_}\DecValTok{3}\NormalTok{_all <-}\StringTok{ }\KeywordTok{as.data.frame}\NormalTok{(y_scale_}\DecValTok{3}\NormalTok{_all)}
\NormalTok{resp <-}\StringTok{ }\KeywordTok{c}\NormalTok{(control_}\DecValTok{3}\NormalTok{, logarithmic_}\DecValTok{3}\NormalTok{)}
\NormalTok{type <-}\StringTok{ }\KeywordTok{c}\NormalTok{(}\KeywordTok{rep}\NormalTok{(}\StringTok{'Control'}\NormalTok{, }\DecValTok{70}\NormalTok{), }\KeywordTok{rep}\NormalTok{(}\StringTok{'Logarithmic'}\NormalTok{, }\DecValTok{70}\NormalTok{)) }
\NormalTok{stats_}\DecValTok{2}\NormalTok{ <-}\StringTok{ }\KeywordTok{data.frame}\NormalTok{(resp, type)}

\KeywordTok{ggplot}\NormalTok{(stats_}\DecValTok{2}\NormalTok{)}\OperatorTok{+}
\StringTok{  }\KeywordTok{geom_bar}\NormalTok{(}\KeywordTok{aes}\NormalTok{(}\DataTypeTok{x=}\NormalTok{resp, }\DataTypeTok{group =}\NormalTok{ type, }\DataTypeTok{fill =}\NormalTok{ type), }\DataTypeTok{position =} \StringTok{'dodge'}\NormalTok{) }\OperatorTok{+}
\StringTok{  }\KeywordTok{ylab}\NormalTok{(}\StringTok{"Number of Respondents"}\NormalTok{)}\OperatorTok{+}
\StringTok{  }\KeywordTok{xlab}\NormalTok{(}\StringTok{"Response"}\NormalTok{)}\OperatorTok{+}
\StringTok{  }\KeywordTok{labs}\NormalTok{(}\DataTypeTok{title =} \StringTok{"Responses selected Over the Whole Population"}\NormalTok{)}\OperatorTok{+}
\StringTok{  }\KeywordTok{scale_fill_manual}\NormalTok{(}\DataTypeTok{values=}\NormalTok{vals, }\DataTypeTok{breaks =} \KeywordTok{c}\NormalTok{(}\StringTok{'Control'}\NormalTok{, }\StringTok{'Logarithmic'}\NormalTok{))}\OperatorTok{+}
\StringTok{  }\KeywordTok{scale_x_discrete}\NormalTok{(}\DataTypeTok{breaks =} \KeywordTok{c}\NormalTok{(}\StringTok{"1"}\NormalTok{, }\StringTok{"2"}\NormalTok{, }\StringTok{"3"}\NormalTok{, }\StringTok{"4"}\NormalTok{, }\StringTok{"5"}\NormalTok{, }\StringTok{"6"}\NormalTok{, }\StringTok{"7"}\NormalTok{), }\DataTypeTok{labels =} \KeywordTok{c}\NormalTok{(}\StringTok{"1"}\NormalTok{, }\StringTok{"2"}\NormalTok{, }\StringTok{"3"}\NormalTok{, }\StringTok{"4"}\NormalTok{, }\StringTok{"5"}\NormalTok{, }\StringTok{"6"}\NormalTok{, }\StringTok{"7"}\NormalTok{), }\DataTypeTok{limits =} \KeywordTok{c}\NormalTok{(}\StringTok{"1"}\NormalTok{, }\StringTok{"2"}\NormalTok{, }\StringTok{"3"}\NormalTok{, }\StringTok{"4"}\NormalTok{, }\StringTok{"5"}\NormalTok{, }\StringTok{"6"}\NormalTok{, }\StringTok{"7"}\NormalTok{))}\OperatorTok{+}
\StringTok{  }\KeywordTok{theme_classic}\NormalTok{()}
\end{Highlighting}
\end{Shaded}

\includegraphics{main_files/figure-latex/unnamed-chunk-20-1.pdf} The
responses for the logarithmically scaled plot are skewed towards the
lower end of the scale, similar to the control and truncated responses,
and there does not appear to be much difference between distributions of
the two populations. Looking at the numbers, however, the averages for
the logarithmic plot (\(\bar{x}=2.22\), \(\tilde{x}=2\)) seem lower than
that of the control plot (\(\bar{x}=3.77\), \(\tilde{x}=4\)). Indeed, a
one sided MWW test comparing the logarithmic and control plot responses
elicits a p-value of \(1.317e-06\), showing evidence that the
logarithmic scale resulted in lower rating in difference of bar height.

Figure {[}?{]} shows the distributions for R and Python subgroups.

\begin{Shaded}
\begin{Highlighting}[]
\NormalTok{brks <-}\StringTok{ }\KeywordTok{c}\NormalTok{(}\StringTok{"Control"}\NormalTok{, }\StringTok{"Logarithmic"}\NormalTok{)}
\NormalTok{vals <-}\StringTok{ }\KeywordTok{c}\NormalTok{(}\StringTok{"#1c9e77"}\NormalTok{,  }\StringTok{"#7570b3"}\NormalTok{)}

\NormalTok{control_}\DecValTok{3}\NormalTok{_r <-}\StringTok{ }\NormalTok{ctrl_y_scale_r}\OperatorTok{$}\NormalTok{con_}\DecValTok{3}
\NormalTok{logarithmic_}\DecValTok{3}\NormalTok{_r <-}\StringTok{ }\NormalTok{log_y_scale_r}\OperatorTok{$}\NormalTok{log_}\DecValTok{3}
\NormalTok{y_scale_}\DecValTok{3}\NormalTok{_r <-}\StringTok{ }\KeywordTok{cbind}\NormalTok{(control_}\DecValTok{3}\NormalTok{, logarithmic_}\DecValTok{3}\NormalTok{)}
\NormalTok{y_scale_}\DecValTok{3}\NormalTok{_r <-}\StringTok{ }\KeywordTok{as.data.frame}\NormalTok{(y_scale_}\DecValTok{3}\NormalTok{_all)}
\NormalTok{resp <-}\StringTok{ }\KeywordTok{c}\NormalTok{(control_}\DecValTok{3}\NormalTok{_r, logarithmic_}\DecValTok{3}\NormalTok{_r)}
\NormalTok{type <-}\StringTok{ }\KeywordTok{c}\NormalTok{(}\KeywordTok{rep}\NormalTok{(}\StringTok{'Control'}\NormalTok{, }\DecValTok{38}\NormalTok{), }\KeywordTok{rep}\NormalTok{(}\StringTok{'Logarithmic'}\NormalTok{, }\DecValTok{38}\NormalTok{)) }
\NormalTok{stats_}\DecValTok{3}\NormalTok{_r <-}\StringTok{ }\KeywordTok{data.frame}\NormalTok{(resp, type)}

\NormalTok{control_}\DecValTok{3}\NormalTok{_py <-}\StringTok{ }\NormalTok{ctrl_y_scale_py}\OperatorTok{$}\NormalTok{con_}\DecValTok{3}
\NormalTok{logarithmic_}\DecValTok{3}\NormalTok{_py <-}\StringTok{ }\NormalTok{log_y_scale_py}\OperatorTok{$}\NormalTok{log_}\DecValTok{3}
\NormalTok{y_scale_}\DecValTok{3}\NormalTok{_py <-}\StringTok{ }\KeywordTok{cbind}\NormalTok{(control_}\DecValTok{3}\NormalTok{, logarithmic_}\DecValTok{3}\NormalTok{)}
\NormalTok{y_scale_}\DecValTok{3}\NormalTok{_py <-}\StringTok{ }\KeywordTok{as.data.frame}\NormalTok{(y_scale_}\DecValTok{3}\NormalTok{_all)}
\NormalTok{resp <-}\StringTok{ }\KeywordTok{c}\NormalTok{(control_}\DecValTok{3}\NormalTok{_py, logarithmic_}\DecValTok{3}\NormalTok{_py)}
\NormalTok{type <-}\StringTok{ }\KeywordTok{c}\NormalTok{(}\KeywordTok{rep}\NormalTok{(}\StringTok{'Control'}\NormalTok{, }\DecValTok{32}\NormalTok{), }\KeywordTok{rep}\NormalTok{(}\StringTok{'Logarithmic'}\NormalTok{, }\DecValTok{32}\NormalTok{)) }
\NormalTok{stats_}\DecValTok{3}\NormalTok{_py <-}\StringTok{ }\KeywordTok{data.frame}\NormalTok{(resp, type)}

\NormalTok{p <-}\StringTok{ }\KeywordTok{ggplot}\NormalTok{(stats_}\DecValTok{3}\NormalTok{_r)}\OperatorTok{+}
\StringTok{  }\KeywordTok{geom_bar}\NormalTok{(}\KeywordTok{aes}\NormalTok{(}\DataTypeTok{x=}\NormalTok{resp, }\DataTypeTok{group =}\NormalTok{ type, }\DataTypeTok{fill =}\NormalTok{ type), }\DataTypeTok{position =} \StringTok{'dodge'}\NormalTok{) }\OperatorTok{+}
\StringTok{  }\KeywordTok{ylab}\NormalTok{(}\StringTok{"Number of Respondents"}\NormalTok{)}\OperatorTok{+}
\StringTok{  }\KeywordTok{xlab}\NormalTok{(}\StringTok{"Response"}\NormalTok{)}\OperatorTok{+}
\StringTok{  }\KeywordTok{labs}\NormalTok{(}\DataTypeTok{title =} \StringTok{"R Subgroup"}\NormalTok{)}\OperatorTok{+}
\StringTok{  }\KeywordTok{scale_fill_manual}\NormalTok{(}\DataTypeTok{values=}\NormalTok{vals, }\DataTypeTok{breaks =} \KeywordTok{c}\NormalTok{(}\StringTok{'Control'}\NormalTok{, }\StringTok{'Logarithmic'}\NormalTok{))}\OperatorTok{+}
\StringTok{  }\KeywordTok{scale_x_discrete}\NormalTok{(}\DataTypeTok{breaks =} \KeywordTok{c}\NormalTok{(}\StringTok{"1"}\NormalTok{, }\StringTok{"2"}\NormalTok{, }\StringTok{"3"}\NormalTok{, }\StringTok{"4"}\NormalTok{, }\StringTok{"5"}\NormalTok{, }\StringTok{"6"}\NormalTok{, }\StringTok{"7"}\NormalTok{), }\DataTypeTok{labels =} \KeywordTok{c}\NormalTok{(}\StringTok{"1"}\NormalTok{, }\StringTok{"2"}\NormalTok{, }\StringTok{"3"}\NormalTok{, }\StringTok{"4"}\NormalTok{, }\StringTok{"5"}\NormalTok{, }\StringTok{"6"}\NormalTok{, }\StringTok{"7"}\NormalTok{), }\DataTypeTok{limits =} \KeywordTok{c}\NormalTok{(}\StringTok{"1"}\NormalTok{, }\StringTok{"2"}\NormalTok{, }\StringTok{"3"}\NormalTok{, }\StringTok{"4"}\NormalTok{, }\StringTok{"5"}\NormalTok{, }\StringTok{"6"}\NormalTok{, }\StringTok{"7"}\NormalTok{))}\OperatorTok{+}
\StringTok{  }\KeywordTok{theme_classic}\NormalTok{()}

\NormalTok{q <-}\StringTok{ }\KeywordTok{ggplot}\NormalTok{(stats_}\DecValTok{3}\NormalTok{_py)}\OperatorTok{+}
\StringTok{  }\KeywordTok{geom_bar}\NormalTok{(}\KeywordTok{aes}\NormalTok{(}\DataTypeTok{x=}\NormalTok{resp, }\DataTypeTok{group =}\NormalTok{ type, }\DataTypeTok{fill =}\NormalTok{ type), }\DataTypeTok{position =} \StringTok{'dodge'}\NormalTok{) }\OperatorTok{+}
\StringTok{  }\KeywordTok{ylab}\NormalTok{(}\StringTok{"Number of Respondents"}\NormalTok{)}\OperatorTok{+}
\StringTok{  }\KeywordTok{xlab}\NormalTok{(}\StringTok{"Response"}\NormalTok{)}\OperatorTok{+}
\StringTok{  }\KeywordTok{labs}\NormalTok{(}\DataTypeTok{title =} \StringTok{"Py Subgroup"}\NormalTok{)}\OperatorTok{+}
\StringTok{  }\KeywordTok{scale_fill_manual}\NormalTok{(}\DataTypeTok{values=}\NormalTok{vals, }\DataTypeTok{breaks =} \KeywordTok{c}\NormalTok{(}\StringTok{'Control'}\NormalTok{, }\StringTok{'Logarithmic'}\NormalTok{))}\OperatorTok{+}
\StringTok{  }\KeywordTok{scale_x_discrete}\NormalTok{(}\DataTypeTok{breaks =} \KeywordTok{c}\NormalTok{(}\StringTok{"1"}\NormalTok{, }\StringTok{"2"}\NormalTok{, }\StringTok{"3"}\NormalTok{, }\StringTok{"4"}\NormalTok{, }\StringTok{"5"}\NormalTok{, }\StringTok{"6"}\NormalTok{, }\StringTok{"7"}\NormalTok{), }\DataTypeTok{labels =} \KeywordTok{c}\NormalTok{(}\StringTok{"1"}\NormalTok{, }\StringTok{"2"}\NormalTok{, }\StringTok{"3"}\NormalTok{, }\StringTok{"4"}\NormalTok{, }\StringTok{"5"}\NormalTok{, }\StringTok{"6"}\NormalTok{, }\StringTok{"7"}\NormalTok{), }\DataTypeTok{limits =} \KeywordTok{c}\NormalTok{(}\StringTok{"1"}\NormalTok{, }\StringTok{"2"}\NormalTok{, }\StringTok{"3"}\NormalTok{, }\StringTok{"4"}\NormalTok{, }\StringTok{"5"}\NormalTok{, }\StringTok{"6"}\NormalTok{, }\StringTok{"7"}\NormalTok{))}\OperatorTok{+}
\StringTok{  }\KeywordTok{theme_classic}\NormalTok{()}

\KeywordTok{plot_grid}\NormalTok{(p, q)}
\end{Highlighting}
\end{Shaded}

\begin{figure}
\centering
\includegraphics{main_files/figure-latex/unnamed-chunk-21-1.pdf}
\caption{Bar plots showing distributions of responses regarding the
control and logarithmic plots for the question 2, separated by language}
\end{figure}

The distributions of the logarithmic plot responses for the R
(\(\bar{x}=2.5\), \(\tilde{x}=2\)) and Python (\(\bar{x}=1.9\),
\(\tilde{x}=2\)) subgroups appear fairly similar, with the same median
albeit with the mean for the R subgroup being slightly higher. The plots
to appear to show the R subgroup responses being slightly positively
skewed and the Python responses more centered around 3. A two sample,
one sided MWW test provides sufficient evidence that the R responses
appear in average greater than the Python (\(p=0.03689\)).

Looking at the responses from the respondents who saw the logarithmic
plot first of the three, the average responses from this group for
question 1 (\(\bar{x}=40\), \(\tilde{x}=40\)) were closer to the true
value of 41 than for the whole population (\(\bar{x}=36.277\),
\(\tilde{x}=35\)), although the former still differs significantly from
the true value (\(p=6.104e-05\)), and there is not significant evidence
to state that the two distributions differ (\(p=0.1705\)). Comparing the
response statistics for the whole population and for those who saw the
logarithmic plot first, the log first group perhaps show the bar height
difference being perceived slightly higher than for the whole population
(\(\bar{x}_{overall}=3.67\), \(\bar{x}_{logfirst} = 4.13\)), however a
two-sample MWW test gives an insignificant p-value of 0.2614 when
comparing them. Similarly, the difference between the responses for the
whole population and for those who saw the logarithmic plot first for
question 3 is also statistically insignificant, with means of 3.08 and
2.68 for and a p-value of 0.1889.

\subsection{Differences Between Question 2 and 3 Responses}

Now take \(\bar{x}_{control} - \bar{x}_{truncated}\) and
\(\bar{x}_{control} - \bar{x}_{logarithmic}\) for each of questions 2
and 3, which is shown in figure{[}?{]}.

\begin{Shaded}
\begin{Highlighting}[]
\NormalTok{control <-}\StringTok{ }\NormalTok{ctrl_y_scale}\OperatorTok{$}\NormalTok{con_}\DecValTok{2}
\NormalTok{truncated <-}\StringTok{ }\NormalTok{trnc_y_scale}\OperatorTok{$}\NormalTok{trn_}\DecValTok{2}
\NormalTok{logarithmic <-}\StringTok{ }\NormalTok{log_y_scale}\OperatorTok{$}\NormalTok{log_}\DecValTok{2}

\NormalTok{y_scale_}\DecValTok{2}\NormalTok{_all <-}\StringTok{ }\KeywordTok{cbind}\NormalTok{(control, truncated, logarithmic)}

\NormalTok{con_mean_}\DecValTok{2}\NormalTok{ <-}\StringTok{  }\KeywordTok{mean}\NormalTok{(control)}
\NormalTok{trn_mean_}\DecValTok{2}\NormalTok{ <-}\StringTok{ }\KeywordTok{mean}\NormalTok{(truncated)}
\NormalTok{log_mean_}\DecValTok{2}\NormalTok{ <-}\StringTok{ }\KeywordTok{mean}\NormalTok{(logarithmic)}
\NormalTok{means_}\DecValTok{2}\NormalTok{ <-}\StringTok{ }\KeywordTok{c}\NormalTok{(trn_mean_}\DecValTok{2}\NormalTok{, log_mean_}\DecValTok{2}\NormalTok{)}

\NormalTok{control <-}\StringTok{ }\NormalTok{ctrl_y_scale}\OperatorTok{$}\NormalTok{con_}\DecValTok{3}
\NormalTok{truncated <-}\StringTok{ }\NormalTok{trnc_y_scale}\OperatorTok{$}\NormalTok{trn_}\DecValTok{3}
\NormalTok{logarithmic <-}\StringTok{ }\NormalTok{log_y_scale}\OperatorTok{$}\NormalTok{log_}\DecValTok{3}

\NormalTok{y_scale_}\DecValTok{3}\NormalTok{_all <-}\StringTok{ }\KeywordTok{cbind}\NormalTok{(control, truncated, logarithmic)}

\NormalTok{con_mean_}\DecValTok{3}\NormalTok{ <-}\StringTok{  }\KeywordTok{mean}\NormalTok{(control)}
\NormalTok{trn_mean_}\DecValTok{3}\NormalTok{ <-}\StringTok{ }\KeywordTok{mean}\NormalTok{(truncated)}
\NormalTok{log_mean_}\DecValTok{3}\NormalTok{ <-}\StringTok{ }\KeywordTok{mean}\NormalTok{(logarithmic)}
\NormalTok{means_}\DecValTok{3}\NormalTok{ <-}\StringTok{ }\KeywordTok{c}\NormalTok{(trn_mean_}\DecValTok{3}\NormalTok{, log_mean_}\DecValTok{3}\NormalTok{)}


\NormalTok{diff_mat <-}\StringTok{ }\KeywordTok{matrix}\NormalTok{(}\OtherTok{NA}\NormalTok{, }\DecValTok{2}\NormalTok{, }\DecValTok{2}\NormalTok{)}

\ControlFlowTok{for}\NormalTok{(i }\ControlFlowTok{in} \DecValTok{1}\OperatorTok{:}\DecValTok{2}\NormalTok{)\{}
\NormalTok{  diff_mat[}\DecValTok{1}\NormalTok{, i] <-}\StringTok{ }\NormalTok{con_mean_}\DecValTok{2}\OperatorTok{-}\NormalTok{means_}\DecValTok{2}\NormalTok{[i]}
\NormalTok{  diff_mat[}\DecValTok{2}\NormalTok{, i] <-}\StringTok{ }\NormalTok{con_mean_}\DecValTok{3}\OperatorTok{-}\NormalTok{means_}\DecValTok{3}\NormalTok{[i]}
\NormalTok{\}}
\end{Highlighting}
\end{Shaded}

\begin{Shaded}
\begin{Highlighting}[]
\KeywordTok{colnames}\NormalTok{(diff_mat) <-}\StringTok{ }\KeywordTok{c}\NormalTok{(}\StringTok{"Con - Trnc"}\NormalTok{, }\StringTok{"Con - Log"}\NormalTok{)}
\KeywordTok{rownames}\NormalTok{(diff_mat) <-}\StringTok{ }\KeywordTok{c}\NormalTok{(}\StringTok{"Q2"}\NormalTok{, }\StringTok{"Q3"}\NormalTok{)}
\KeywordTok{kable}\NormalTok{(diff_mat, }\DataTypeTok{caption =} \StringTok{"Table showing difference in the percieved difference for the logarithmic-scaled and truncated plots as compared to the control, for questions 2 and 3"}\NormalTok{) }\OperatorTok
\StringTok{  }\KeywordTok{kable_styling}\NormalTok{(}\DataTypeTok{latex_options =} \StringTok{"hold_position"}\NormalTok{)}
\end{Highlighting}
\end{Shaded}

\begin{table}[!h]

\caption{\label{tab:unnamed-chunk-23}Table showing difference in the percieved difference for the logarithmic-scaled and truncated plots as compared to the control, for questions 2 and 3}
\centering
\begin{tabular}[t]{l|r|r}
\hline
  & Con - Trnc & Con - Log\\
\hline
Q2 & -0.5142857 & 1.685714\\
\hline
Q3 & -0.6428571 & 0.900000\\
\hline
\end{tabular}
\end{table}

This again shows that the responses for the truncated plot were in
general rated higher than the control plot responses, and also that the
effect was more significant for the bars on opposite ends of the plot as
compared to the bars next to each other. The opposite is true for the
logarithmic plot responses; on average they were rated lower than the
control plot, but this was greatly more significant for the bars next to
each other, as opposed to the truncated plot. Figure{[}?{]} shows this
visually.

\begin{Shaded}
\begin{Highlighting}[]
\NormalTok{brks <-}\StringTok{ }\KeywordTok{c}\NormalTok{(}\StringTok{"Truncated"}\NormalTok{, }\StringTok{"Logarithmic"}\NormalTok{)}
\NormalTok{vals <-}\StringTok{ }\KeywordTok{c}\NormalTok{(}\StringTok{"#d95f02"}\NormalTok{,  }\StringTok{"#7570b3"}\NormalTok{)}

\NormalTok{df <-}\StringTok{ }\KeywordTok{data.frame}\NormalTok{(}\StringTok{"Trnc"}\NormalTok{=}\KeywordTok{c}\NormalTok{(diff_mat[,}\DecValTok{1}\NormalTok{]), }\StringTok{"Log"}\NormalTok{=}\KeywordTok{c}\NormalTok{(diff_mat[,}\DecValTok{2}\NormalTok{]), }\StringTok{"Question"}\NormalTok{ =}\StringTok{ }\KeywordTok{c}\NormalTok{(}\StringTok{"Log Grip vs Salmon Ladder"}\NormalTok{, }\StringTok{"Quintuple Steps vs Salmon Ladder"}\NormalTok{))}


\KeywordTok{ggplot}\NormalTok{()}\OperatorTok{+}
\StringTok{  }\KeywordTok{geom_bar}\NormalTok{(}\DataTypeTok{data=}\NormalTok{df, }\KeywordTok{aes}\NormalTok{(}\DataTypeTok{x =}\NormalTok{ Question, }\DataTypeTok{y=}\NormalTok{Trnc, }\DataTypeTok{fill =}\NormalTok{ brks[}\DecValTok{1}\NormalTok{]), }\DataTypeTok{stat=}\StringTok{"identity"}\NormalTok{)}\OperatorTok{+}
\StringTok{  }\KeywordTok{geom_bar}\NormalTok{(}\DataTypeTok{data=}\NormalTok{df, }\KeywordTok{aes}\NormalTok{(}\DataTypeTok{x =}\NormalTok{ Question, }\DataTypeTok{y=}\NormalTok{Log, }\DataTypeTok{fill =}\NormalTok{ brks[}\DecValTok{2}\NormalTok{]), }\DataTypeTok{stat=}\StringTok{"identity"}\NormalTok{)}\OperatorTok{+}
\StringTok{  }\KeywordTok{geom_hline}\NormalTok{(}\DataTypeTok{yintercept =} \DecValTok{0}\NormalTok{)}\OperatorTok{+}
\StringTok{  }\KeywordTok{scale_fill_manual}\NormalTok{(}\DataTypeTok{name=}\StringTok{"Scale"}\NormalTok{, }\DataTypeTok{breaks=}\NormalTok{brks, }\DataTypeTok{values=}\NormalTok{vals)}\OperatorTok{+}
\StringTok{  }\KeywordTok{scale_y_continuous}\NormalTok{(}\DataTypeTok{name =} \StringTok{"Average difference from control plot"}\NormalTok{, }\DataTypeTok{limits =} \KeywordTok{c}\NormalTok{(}\OperatorTok{-}\DecValTok{1}\NormalTok{, }\DecValTok{2}\NormalTok{), }\DataTypeTok{breaks =} \KeywordTok{seq}\NormalTok{(}\OperatorTok{-}\DecValTok{1}\NormalTok{, }\DecValTok{2}\NormalTok{, }\FloatTok{0.5}\NormalTok{))}\OperatorTok{+}
\StringTok{  }\KeywordTok{theme_classic}\NormalTok{()}
\end{Highlighting}
\end{Shaded}

\begin{figure}
\centering
\includegraphics{main_files/figure-latex/unnamed-chunk-24-1.pdf}
\caption{Bar plot giving a visual representation of the table}
\end{figure}

On average, truncating the scale had a similar effect for both
questions, albeit with slightly more effect for when comparing `Salmon
Ladder' with `Quintuple Steps' as opposed to `Log Grip'. For the
logarithmically scaled plots, however, the re-scaling appears to have
had a significantly greater effect when considering the bars directly
next to each other, with respondents on average judging the difference
in bar height to be greater by 1.68 on the 7-point scale, whereas this
is 0.9 for the bars further apart. It can be concluded from this that
truncating the scale had more of an impact when bars were on opposite
ends of the plot as opposed to next to each other, and the way round for
the bars close to each other; the logarithmic scaling had more of an
impact.

\section{American Ninja Warrior - Part 2}

This part of the survey assessed whether different aspect ratios would
have an impact on perception of bar height differences as well as
reading of true values. This part will be analysed question by question.

Question 1 asked
\textit{'How large would you say the difference between 'Jumping spider' and 'Salmon Ladder' is?'}.
This question once again uses the 7-point scale to gain a subjective
view on the degree to which respondents felt the heights between the two
bars corresponding to `Jumping Spider' and `Salmon Ladder' differed for
three bar plots of 7 obstacles, where `Salmon Ladder' is furthest to the
left, and `Jumping Spider' furthest to the right.

\begin{Shaded}
\begin{Highlighting}[]
\NormalTok{default <-}\StringTok{ }\NormalTok{def_ratio}\OperatorTok{$}\NormalTok{def_}\DecValTok{1}
\NormalTok{narrow <-}\StringTok{ }\NormalTok{nar_ratio}\OperatorTok{$}\NormalTok{nar_}\DecValTok{1}
\NormalTok{wide <-}\StringTok{ }\NormalTok{wid_ratio}\OperatorTok{$}\NormalTok{wid_}\DecValTok{1}
\NormalTok{ratio_}\DecValTok{1}\NormalTok{_all <-}\StringTok{ }\KeywordTok{cbind}\NormalTok{(default, wide, narrow)}
\end{Highlighting}
\end{Shaded}

Looking at the means and medians here, it doesn't seem like there is
that much of a difference in perception of the differences between the
three aspect ratios, as displayed in table{[}?{]}.

\begin{Shaded}
\begin{Highlighting}[]
\NormalTok{means <-}\StringTok{ }\KeywordTok{c}\NormalTok{(}\KeywordTok{mean}\NormalTok{(default), }\KeywordTok{mean}\NormalTok{(narrow), }\KeywordTok{mean}\NormalTok{(wide))}
\NormalTok{meds <-}\StringTok{ }\KeywordTok{c}\NormalTok{(}\KeywordTok{median}\NormalTok{(default), }\KeywordTok{median}\NormalTok{(narrow), }\KeywordTok{median}\NormalTok{(wide))}
\NormalTok{tab <-}\StringTok{ }\KeywordTok{rbind}\NormalTok{(means, meds)}
\KeywordTok{colnames}\NormalTok{(tab) <-}\StringTok{ }\KeywordTok{c}\NormalTok{(}\StringTok{"Default"}\NormalTok{, }\StringTok{"Narrow"}\NormalTok{, }\StringTok{"Wide"}\NormalTok{)}
\KeywordTok{rownames}\NormalTok{(tab) <-}\StringTok{ }\KeywordTok{c}\NormalTok{(}\StringTok{"Mean"}\NormalTok{, }\StringTok{"Median"}\NormalTok{)}
\KeywordTok{kable}\NormalTok{(tab, }\DataTypeTok{digits=}\DecValTok{3}\NormalTok{) }\OperatorTok
\StringTok{  }\KeywordTok{kable_styling}\NormalTok{(}\DataTypeTok{latex_options =} \StringTok{"hold_position"}\NormalTok{)}
\end{Highlighting}
\end{Shaded}

\begin{table}[!h]
\centering
\begin{tabular}{l|r|r|r}
\hline
  & Default & Narrow & Wide\\
\hline
Mean & 5.914 & 6.129 & 5.357\\
\hline
Median & 6.000 & 6.000 & 6.000\\
\hline
\end{tabular}
\end{table}

Note that `narrow' is defined as the plot with the aspect ratio of
smaller width to greater height, and vice versa for the `wide' plot. The
means show marginal differences, whereby the default plot mean is the
middle-valued mean of the three, with the mean perceived difference for
the wide plot being slightly smaller than this and the mean perceived
difference for the narrow plot is slightly larger. This result, although
at first glance marginal, follows the hypothesis that the wide plot
would cause differences to be perceived as smaller and narrow bars to
cause differences to be perceived to be greater.

Now looking at figure{[}?{]}, showing the three distributions. There
isn't an immediately obvious difference in distributions, but on closer
inspection it can be seen that the orange ``Wide'' bars dominate over
the three for the range {[}2, 5{]}, and the purple ``Narrow'' dominated
for the response of 7, following the above analysis of summary
statistics. There was a fairly strong consensus that in general that a
rating of 6 was applicable to all three plots.

\begin{Shaded}
\begin{Highlighting}[]
\NormalTok{brks <-}\StringTok{ }\KeywordTok{c}\NormalTok{(}\StringTok{"Default"}\NormalTok{, }\StringTok{"wide"}\NormalTok{, }\StringTok{"narrow"}\NormalTok{)}
\NormalTok{vals <-}\StringTok{ }\KeywordTok{c}\NormalTok{(}\StringTok{"#1c9e77"}\NormalTok{, }\StringTok{"#d95f02"}\NormalTok{, }\StringTok{"#7570b3"}\NormalTok{)}

\NormalTok{ratio_}\DecValTok{1}\NormalTok{_all <-}\StringTok{ }\KeywordTok{as.data.frame}\NormalTok{(ratio_}\DecValTok{1}\NormalTok{_all)}
\NormalTok{resp <-}\StringTok{ }\KeywordTok{c}\NormalTok{(default, wide, narrow)}
\NormalTok{type <-}\StringTok{ }\KeywordTok{c}\NormalTok{(}\KeywordTok{rep}\NormalTok{(}\StringTok{'Default'}\NormalTok{, }\DecValTok{70}\NormalTok{), }\KeywordTok{rep}\NormalTok{(}\StringTok{'wide'}\NormalTok{, }\DecValTok{70}\NormalTok{), }\KeywordTok{rep}\NormalTok{(}\StringTok{'narrow'}\NormalTok{, }\DecValTok{70}\NormalTok{)) }
\NormalTok{stats <-}\StringTok{ }\KeywordTok{data.frame}\NormalTok{(resp, type)}

\KeywordTok{ggplot}\NormalTok{(}\DataTypeTok{data =}\NormalTok{ stats)}\OperatorTok{+}
\StringTok{  }\KeywordTok{geom_bar}\NormalTok{(}\KeywordTok{aes}\NormalTok{(}\DataTypeTok{x=}\NormalTok{resp, }\DataTypeTok{group=}\NormalTok{type, }\DataTypeTok{fill=}\NormalTok{type), }\DataTypeTok{position =} \StringTok{'dodge'}\NormalTok{)}\OperatorTok{+}
\StringTok{  }\KeywordTok{theme_classic}\NormalTok{()}\OperatorTok{+}
\StringTok{  }\KeywordTok{scale_fill_manual}\NormalTok{(}\DataTypeTok{breaks =}\NormalTok{ brks, }\DataTypeTok{values =}\NormalTok{ vals)}\OperatorTok{+}
\StringTok{  }\KeywordTok{scale_x_discrete}\NormalTok{(}\DataTypeTok{breaks =} \KeywordTok{c}\NormalTok{(}\StringTok{"1"}\NormalTok{, }\StringTok{"2"}\NormalTok{, }\StringTok{"3"}\NormalTok{, }\StringTok{"4"}\NormalTok{, }\StringTok{"5"}\NormalTok{, }\StringTok{"6"}\NormalTok{, }\StringTok{"7"}\NormalTok{), }\DataTypeTok{limits =} \KeywordTok{c}\NormalTok{(}\StringTok{"1"}\NormalTok{, }\StringTok{"2"}\NormalTok{, }\StringTok{"3"}\NormalTok{, }\StringTok{"4"}\NormalTok{, }\StringTok{"5"}\NormalTok{, }\StringTok{"6"}\NormalTok{, }\StringTok{"7"}\NormalTok{), }\DataTypeTok{labels =} \KeywordTok{c}\NormalTok{(}\StringTok{"1"}\NormalTok{, }\StringTok{"2"}\NormalTok{, }\StringTok{"3"}\NormalTok{, }\StringTok{"4"}\NormalTok{, }\StringTok{"5"}\NormalTok{, }\StringTok{"6"}\NormalTok{, }\StringTok{"7"}\NormalTok{))}\OperatorTok{+}
\StringTok{  }\KeywordTok{xlab}\NormalTok{(}\StringTok{"Response"}\NormalTok{)}\OperatorTok{+}
\StringTok{  }\KeywordTok{ylab}\NormalTok{(}\StringTok{"Number of Respondents"}\NormalTok{)}
\end{Highlighting}
\end{Shaded}

\begin{figure}
\centering
\includegraphics{main_files/figure-latex/unnamed-chunk-27-1.pdf}
\caption{Bar plots showing distributions of responses regarding the
three plots}
\end{figure}

\begin{Shaded}
\begin{Highlighting}[]
\KeywordTok{SIGN.test}\NormalTok{(default, wide, }\DataTypeTok{alt=}\StringTok{"g"}\NormalTok{)}
\end{Highlighting}
\end{Shaded}

\begin{verbatim}
## 
##  Dependent-samples Sign-Test
## 
## data:  default and wide
## S = 31, p-value = 6.457e-06
## alternative hypothesis: true median difference is greater than 0
## 95 percent confidence interval:
##    0 Inf
## sample estimates:
## median of x-y 
##             0 
## 
## Achieved and Interpolated Confidence Intervals: 
## 
##                   Conf.Level L.E.pt U.E.pt
## Lower Achieved CI     0.9402      0    Inf
## Interpolated CI       0.9500      0    Inf
## Upper Achieved CI     0.9639      0    Inf
\end{verbatim}

Running a one-sided MWW test to compare the responses for default plot
to the narrow plot, it is confirmed that there is evidence to suggest
that using a `narrow' aspect ratio causes the perceived difference to be
greater (\(p=0.0468\)). Then applying a one-sided sign test to compare
the default to the wide plot, the perceived difference is shown to be
smaller (\(p=6.457e-06\)).

\begin{Shaded}
\begin{Highlighting}[]
\KeywordTok{ggplot}\NormalTok{() }\OperatorTok{+}
\StringTok{  }\KeywordTok{geom_density}\NormalTok{(}\DataTypeTok{data =} \KeywordTok{as.data.frame}\NormalTok{(default ), }\KeywordTok{aes}\NormalTok{(}\DataTypeTok{x=}\NormalTok{default , }\DataTypeTok{col =} \StringTok{"Default"}\NormalTok{))}\OperatorTok{+}
\StringTok{  }\KeywordTok{geom_density}\NormalTok{(}\DataTypeTok{data =} \KeywordTok{as.data.frame}\NormalTok{(wide   ), }\KeywordTok{aes}\NormalTok{(}\DataTypeTok{x=}\NormalTok{wide   , }\DataTypeTok{col =} \StringTok{"wide"}\NormalTok{))}\OperatorTok{+}
\StringTok{  }\KeywordTok{geom_density}\NormalTok{(}\DataTypeTok{data =} \KeywordTok{as.data.frame}\NormalTok{(narrow), }\KeywordTok{aes}\NormalTok{(}\DataTypeTok{x=}\NormalTok{narrow, }\DataTypeTok{col =} \StringTok{"narrow"}\NormalTok{))}\OperatorTok{+}
\StringTok{  }\KeywordTok{labs}\NormalTok{(}\DataTypeTok{x=}\StringTok{"Response"}\NormalTok{, }\DataTypeTok{y=}\StringTok{"Density"}\NormalTok{)}\OperatorTok{+}
\StringTok{  }\KeywordTok{scale_colour_manual}\NormalTok{(}\DataTypeTok{name =} \StringTok{" "}\NormalTok{, }\DataTypeTok{breaks =}\NormalTok{ brks, }\DataTypeTok{values =}\NormalTok{ vals)}\OperatorTok{+}
\StringTok{  }\KeywordTok{theme_classic}\NormalTok{()}
\end{Highlighting}
\end{Shaded}

\begin{figure}
\centering
\includegraphics{main_files/figure-latex/unnamed-chunk-28-1.pdf}
\caption{Density plot showing distributions of responses regarding the
three plots}
\end{figure}

Question 2 then went on to ask
\textit{'How large would you say the difference between 'Log Grip' and 'Floating Steps' is?'}.
Similar to part 1, there are two questions for gauging differences
between bars, for which one asks about bars far away from each other,
and one about bars next to each other. In the case of this section, the
first question contained bars on opposite ends of the x-axis, and this
question asks about two bars that sit adjacent to one another.

\begin{Shaded}
\begin{Highlighting}[]
\NormalTok{default <-}\StringTok{ }\NormalTok{def_ratio}\OperatorTok{$}\NormalTok{def_}\DecValTok{2}
\NormalTok{narrow <-}\StringTok{ }\NormalTok{nar_ratio}\OperatorTok{$}\NormalTok{nar_}\DecValTok{2}
\NormalTok{wide <-}\StringTok{ }\NormalTok{wid_ratio}\OperatorTok{$}\NormalTok{wid_}\DecValTok{2}

\NormalTok{ratio_}\DecValTok{2}\NormalTok{_all <-}\StringTok{ }\KeywordTok{cbind}\NormalTok{(default, wide, narrow)}
\end{Highlighting}
\end{Shaded}

The analysis results here show that altering the axis ratio appears to
have even less of an effect than in the first question, with the means
of the responses for the default and wide plots being identical at
3.057, with the mean of the narrow plot responses only 0.157 greater
at3.214. The median for all three is 3, and the IQRs are all \([2, 7]\).
The variances, however, do differ from one another, with values 1.301,
0.866 and 1.214 for the default, wide and narrow bars, respectively. The
distribution of values are shown in figure{[}?{]}. The results of
two-sided MWW tests show that neither aspect ratio appears to have a
significant effect on the rating of the perceived difference
(\(p=0.2446\) and \(p=0.5688\)).

\begin{Shaded}
\begin{Highlighting}[]
\NormalTok{brks <-}\StringTok{ }\KeywordTok{c}\NormalTok{(}\StringTok{"Default"}\NormalTok{, }\StringTok{"wide"}\NormalTok{, }\StringTok{"narrow"}\NormalTok{)}
\NormalTok{vals <-}\StringTok{ }\KeywordTok{c}\NormalTok{(}\StringTok{"#1c9e77"}\NormalTok{, }\StringTok{"#d95f02"}\NormalTok{, }\StringTok{"#7570b3"}\NormalTok{)}

\NormalTok{ratio_}\DecValTok{1}\NormalTok{_all <-}\StringTok{ }\KeywordTok{as.data.frame}\NormalTok{(ratio_}\DecValTok{2}\NormalTok{_all)}
\NormalTok{resp <-}\StringTok{ }\KeywordTok{c}\NormalTok{(default, wide, narrow)}
\NormalTok{type <-}\StringTok{ }\KeywordTok{c}\NormalTok{(}\KeywordTok{rep}\NormalTok{(}\StringTok{'Default'}\NormalTok{, }\DecValTok{70}\NormalTok{), }\KeywordTok{rep}\NormalTok{(}\StringTok{'wide'}\NormalTok{, }\DecValTok{70}\NormalTok{), }\KeywordTok{rep}\NormalTok{(}\StringTok{'narrow'}\NormalTok{, }\DecValTok{70}\NormalTok{)) }
\NormalTok{stats <-}\StringTok{ }\KeywordTok{data.frame}\NormalTok{(resp, type)}

\KeywordTok{ggplot}\NormalTok{(}\DataTypeTok{data =}\NormalTok{ stats, }\KeywordTok{aes}\NormalTok{(}\DataTypeTok{x=}\NormalTok{type, }\DataTypeTok{y=}\NormalTok{resp, }\DataTypeTok{fill=}\NormalTok{type))}\OperatorTok{+}
\StringTok{  }\KeywordTok{geom_boxplot}\NormalTok{(}\DataTypeTok{outlier.colour=}\StringTok{"black"}\NormalTok{, }\DataTypeTok{outlier.shape=}\DecValTok{1}\NormalTok{,}
             \DataTypeTok{outlier.size=}\DecValTok{2}\NormalTok{, }\DataTypeTok{notch=}\NormalTok{F)}\OperatorTok{+}
\StringTok{  }\KeywordTok{theme_classic}\NormalTok{()}\OperatorTok{+}
\StringTok{  }\KeywordTok{stat_summary}\NormalTok{(}\DataTypeTok{fun=}\NormalTok{mean, }\DataTypeTok{geom=}\StringTok{"point"}\NormalTok{, }\DataTypeTok{shape=}\DecValTok{15}\NormalTok{, }\DataTypeTok{size=}\DecValTok{4}\NormalTok{)}\OperatorTok{+}
\StringTok{  }\KeywordTok{scale_fill_manual}\NormalTok{(}\DataTypeTok{breaks =}\NormalTok{ brks, }\DataTypeTok{values =}\NormalTok{ vals)}\OperatorTok{+}
\StringTok{  }\KeywordTok{scale_y_continuous}\NormalTok{(}\DataTypeTok{labels =} \KeywordTok{seq}\NormalTok{(}\DecValTok{1}\NormalTok{, }\DecValTok{7}\NormalTok{, }\DecValTok{1}\NormalTok{), }\DataTypeTok{breaks =} \KeywordTok{seq}\NormalTok{(}\DecValTok{1}\NormalTok{, }\DecValTok{7}\NormalTok{, }\DecValTok{1}\NormalTok{))}\OperatorTok{+}
\StringTok{  }\KeywordTok{ylab}\NormalTok{(}\StringTok{"Value"}\NormalTok{)}\OperatorTok{+}
\StringTok{  }\KeywordTok{xlab}\NormalTok{(}\StringTok{"Plot Type"}\NormalTok{)}
\end{Highlighting}
\end{Shaded}

\includegraphics{main_files/figure-latex/unnamed-chunk-30-1.pdf} At
least \(50/%
\) of respondents placed the difference in the range \([2, 4]\) for all
three plots, showing that they believed the difference was small to
moderate, and this didn't change depending on the plot type, and thus
for the bars further apart from each other, changing the aspect ratio
does not appear to make much of a difference. The overall distributions
are shown in the figure{[}?{]}

\begin{Shaded}
\begin{Highlighting}[]
\KeywordTok{ggplot}\NormalTok{() }\OperatorTok{+}
\StringTok{  }\KeywordTok{geom_density}\NormalTok{(}\DataTypeTok{data =} \KeywordTok{as.data.frame}\NormalTok{(default ), }\KeywordTok{aes}\NormalTok{(}\DataTypeTok{x=}\NormalTok{default , }\DataTypeTok{col =} \StringTok{"Default"}\NormalTok{))}\OperatorTok{+}
\StringTok{  }\KeywordTok{geom_density}\NormalTok{(}\DataTypeTok{data =} \KeywordTok{as.data.frame}\NormalTok{(wide   ), }\KeywordTok{aes}\NormalTok{(}\DataTypeTok{x=}\NormalTok{wide   , }\DataTypeTok{col =} \StringTok{"wide"}\NormalTok{))}\OperatorTok{+}
\StringTok{  }\KeywordTok{geom_density}\NormalTok{(}\DataTypeTok{data =} \KeywordTok{as.data.frame}\NormalTok{(narrow), }\KeywordTok{aes}\NormalTok{(}\DataTypeTok{x=}\NormalTok{narrow, }\DataTypeTok{col =} \StringTok{"narrow"}\NormalTok{))}\OperatorTok{+}
\StringTok{  }\KeywordTok{labs}\NormalTok{(}\DataTypeTok{x=}\StringTok{"Response"}\NormalTok{, }\DataTypeTok{y=}\StringTok{"Density"}\NormalTok{)}\OperatorTok{+}
\StringTok{  }\KeywordTok{scale_x_continuous}\NormalTok{(}\DataTypeTok{limits=}\KeywordTok{c}\NormalTok{(}\DecValTok{0}\NormalTok{, }\DecValTok{7}\NormalTok{))}\OperatorTok{+}
\StringTok{  }\KeywordTok{scale_colour_manual}\NormalTok{(}\DataTypeTok{name =} \StringTok{" "}\NormalTok{, }\DataTypeTok{breaks =}\NormalTok{ brks, }\DataTypeTok{values =}\NormalTok{ vals)}\OperatorTok{+}
\StringTok{  }\KeywordTok{theme_classic}\NormalTok{()}
\end{Highlighting}
\end{Shaded}

\begin{figure}
\centering
\includegraphics{main_files/figure-latex/unnamed-chunk-31-1.pdf}
\caption{Density plots showing distributions of responses regarding the
three plots}
\end{figure}

All three distributions are very similar, and almost appear to form bell
curve shaped distributions, albeit with some irregularities and very
slight negative skew.

As in part 1, the two height difference perception questions will be
compared, calculating \(\bar{x}_{default} - \bar{x}_{narrow}\) and
\(\bar{x}_{default} - \bar{x}_{wide}\).

\begin{Shaded}
\begin{Highlighting}[]
\NormalTok{default <-}\StringTok{ }\NormalTok{def_ratio}\OperatorTok{$}\NormalTok{def_}\DecValTok{1}
\NormalTok{narrow <-}\StringTok{ }\NormalTok{nar_ratio}\OperatorTok{$}\NormalTok{nar_}\DecValTok{1}
\NormalTok{wide <-}\StringTok{ }\NormalTok{wid_ratio}\OperatorTok{$}\NormalTok{wid_}\DecValTok{1}
\NormalTok{ratio_}\DecValTok{1}\NormalTok{_all <-}\StringTok{ }\KeywordTok{cbind}\NormalTok{(default, wide, narrow)}

\NormalTok{def_mean_}\DecValTok{1}\NormalTok{ <-}\StringTok{  }\KeywordTok{mean}\NormalTok{(default)}
\NormalTok{nar_mean_}\DecValTok{1}\NormalTok{ <-}\StringTok{ }\KeywordTok{mean}\NormalTok{(narrow)}
\NormalTok{wid_mean_}\DecValTok{1}\NormalTok{ <-}\StringTok{ }\KeywordTok{mean}\NormalTok{(wide)}
\NormalTok{means_}\DecValTok{1}\NormalTok{ <-}\StringTok{ }\KeywordTok{c}\NormalTok{(nar_mean_}\DecValTok{1}\NormalTok{, wid_mean_}\DecValTok{1}\NormalTok{)}

\NormalTok{default <-}\StringTok{ }\NormalTok{def_ratio}\OperatorTok{$}\NormalTok{def_}\DecValTok{2}
\NormalTok{narrow <-}\StringTok{ }\NormalTok{nar_ratio}\OperatorTok{$}\NormalTok{nar_}\DecValTok{2}
\NormalTok{wide <-}\StringTok{ }\NormalTok{wid_ratio}\OperatorTok{$}\NormalTok{wid_}\DecValTok{2}
\NormalTok{ratio_}\DecValTok{2}\NormalTok{_all <-}\StringTok{ }\KeywordTok{cbind}\NormalTok{(default, wide, narrow)}

\NormalTok{def_mean_}\DecValTok{2}\NormalTok{ <-}\StringTok{  }\KeywordTok{mean}\NormalTok{(default)}
\NormalTok{nar_mean_}\DecValTok{2}\NormalTok{ <-}\StringTok{ }\KeywordTok{mean}\NormalTok{(narrow)}
\NormalTok{wid_mean_}\DecValTok{2}\NormalTok{ <-}\StringTok{ }\KeywordTok{mean}\NormalTok{(wide)}
\NormalTok{means_}\DecValTok{2}\NormalTok{ <-}\StringTok{ }\KeywordTok{c}\NormalTok{(nar_mean_}\DecValTok{2}\NormalTok{, wid_mean_}\DecValTok{2}\NormalTok{)}

\NormalTok{diff_mat <-}\StringTok{ }\KeywordTok{matrix}\NormalTok{(}\OtherTok{NA}\NormalTok{, }\DecValTok{2}\NormalTok{, }\DecValTok{2}\NormalTok{)}

\ControlFlowTok{for}\NormalTok{(i }\ControlFlowTok{in} \DecValTok{1}\OperatorTok{:}\DecValTok{2}\NormalTok{)\{}
\NormalTok{  diff_mat[}\DecValTok{1}\NormalTok{, i] <-}\StringTok{ }\NormalTok{def_mean_}\DecValTok{1}\OperatorTok{-}\NormalTok{means_}\DecValTok{1}\NormalTok{[i]}
\NormalTok{  diff_mat[}\DecValTok{2}\NormalTok{, i] <-}\StringTok{ }\NormalTok{def_mean_}\DecValTok{2}\OperatorTok{-}\NormalTok{means_}\DecValTok{2}\NormalTok{[i]}
\NormalTok{\}}
\end{Highlighting}
\end{Shaded}

\begin{Shaded}
\begin{Highlighting}[]
\KeywordTok{colnames}\NormalTok{(diff_mat) <-}\StringTok{ }\KeywordTok{c}\NormalTok{(}\StringTok{"Def - Narrow"}\NormalTok{, }\StringTok{"Def - Wide"}\NormalTok{)}
\KeywordTok{rownames}\NormalTok{(diff_mat) <-}\StringTok{ }\KeywordTok{c}\NormalTok{(}\StringTok{"Q1"}\NormalTok{, }\StringTok{"Q2"}\NormalTok{)}
\KeywordTok{kable}\NormalTok{(diff_mat, }\DataTypeTok{caption =} \StringTok{"Table showing difference in the percieved difference for plots with narrow and wide bars as compared to the default, for questions 1 and 2"}\NormalTok{) }\OperatorTok
\StringTok{  }\KeywordTok{kable_styling}\NormalTok{(}\DataTypeTok{latex_options =} \StringTok{"hold_position"}\NormalTok{)}
\end{Highlighting}
\end{Shaded}

\begin{table}[!h]

\caption{\label{tab:unnamed-chunk-33}Table showing difference in the percieved difference for plots with narrow and wide bars as compared to the default, for questions 1 and 2}
\centering
\begin{tabular}[t]{l|r|r}
\hline
  & Def - Narrow & Def - Wide\\
\hline
Q1 & -0.2142857 & 0.5571429\\
\hline
Q2 & -0.1571429 & 0.0000000\\
\hline
\end{tabular}
\end{table}

As before, the table below gives a visual representation.

\begin{Shaded}
\begin{Highlighting}[]
\NormalTok{df <-}\StringTok{ }\KeywordTok{data.frame}\NormalTok{(}\StringTok{"narrow"}\NormalTok{=}\KeywordTok{c}\NormalTok{(diff_mat[,}\DecValTok{1}\NormalTok{]), }\StringTok{"wide"}\NormalTok{=}\KeywordTok{c}\NormalTok{(diff_mat[,}\DecValTok{2}\NormalTok{]), }\StringTok{"Question"}\NormalTok{ =}\StringTok{ }\KeywordTok{c}\NormalTok{(}\StringTok{"Jumping Spider vs Salmon Ladder"}\NormalTok{, }\StringTok{"Log Grip vs Floating Steps"}\NormalTok{))}


\KeywordTok{ggplot}\NormalTok{()}\OperatorTok{+}
\StringTok{  }\KeywordTok{geom_bar}\NormalTok{(}\DataTypeTok{data=}\NormalTok{df, }\KeywordTok{aes}\NormalTok{(}\DataTypeTok{x =}\NormalTok{ Question, }\DataTypeTok{y=}\NormalTok{narrow, }\DataTypeTok{fill =}\NormalTok{ brks[}\DecValTok{2}\NormalTok{]), }\DataTypeTok{stat=}\StringTok{"identity"}\NormalTok{)}\OperatorTok{+}
\StringTok{  }\KeywordTok{geom_bar}\NormalTok{(}\DataTypeTok{data=}\NormalTok{df, }\KeywordTok{aes}\NormalTok{(}\DataTypeTok{x =}\NormalTok{ Question, }\DataTypeTok{y=}\NormalTok{wide, }\DataTypeTok{fill =}\NormalTok{ brks[}\DecValTok{3}\NormalTok{]), }\DataTypeTok{stat=}\StringTok{"identity"}\NormalTok{)}\OperatorTok{+}
\StringTok{  }\KeywordTok{geom_hline}\NormalTok{(}\DataTypeTok{yintercept =} \DecValTok{0}\NormalTok{)}\OperatorTok{+}
\StringTok{  }\KeywordTok{scale_fill_manual}\NormalTok{(}\DataTypeTok{name=}\StringTok{"Scale"}\NormalTok{ ,}\DataTypeTok{breaks=}\NormalTok{brks[}\DecValTok{2}\OperatorTok{:}\DecValTok{3}\NormalTok{], }\DataTypeTok{values=}\NormalTok{vals[}\DecValTok{2}\OperatorTok{:}\DecValTok{3}\NormalTok{])}\OperatorTok{+}
\StringTok{  }\KeywordTok{scale_y_continuous}\NormalTok{(}\DataTypeTok{name =} \StringTok{"Average difference from default plot"}\NormalTok{, }\DataTypeTok{limits =} \KeywordTok{c}\NormalTok{(}\OperatorTok{-}\FloatTok{0.25}\NormalTok{, }\FloatTok{0.6}\NormalTok{), }\DataTypeTok{breaks =} \KeywordTok{seq}\NormalTok{(}\OperatorTok{-}\FloatTok{0.25}\NormalTok{, }\FloatTok{0.6}\NormalTok{, }\FloatTok{0.1}\NormalTok{))}\OperatorTok{+}
\StringTok{  }\KeywordTok{theme_classic}\NormalTok{()}
\end{Highlighting}
\end{Shaded}

\includegraphics{main_files/figure-latex/unnamed-chunk-34-1.pdf} Both by
eye comparisons of values and statistical testing show that the language
used has negligible effect on the perceived difference, as does the
order in which the plots were shown. See tables 51 - 61 in the appendix
for more details.

\subsection{How many times would you say 'Floating Steps' were used?}

This is again similar to question 1 of part 1, where participants were
asked to state what they believed to be the height of the bar for
`Salmon Ladder', however this time the third bar from the axis is
chosen. This is to ascertain whether the distance of the bar from the
axis may have an effect alongside any potential perceived distortion of
values. Note that the true value was 28.

\begin{Shaded}
\begin{Highlighting}[]
\NormalTok{default <-}\StringTok{ }\NormalTok{def_ratio}\OperatorTok{$}\NormalTok{def_}\DecValTok{3}
\NormalTok{narrow <-}\StringTok{ }\NormalTok{nar_ratio}\OperatorTok{$}\NormalTok{nar_}\DecValTok{3}
\NormalTok{wide <-}\StringTok{ }\NormalTok{wid_ratio}\OperatorTok{$}\NormalTok{wid_}\DecValTok{3}

\NormalTok{ratio_}\DecValTok{1}\NormalTok{_all <-}\StringTok{ }\KeywordTok{cbind}\NormalTok{(default, wide, narrow)}
\end{Highlighting}
\end{Shaded}

The means of each of the three sets of responses were very close to the
true value, at 27.97, 28.04 and 27.39, respectively for the default,
wide and narrow, and the medians are exactly equal to the true value.
Based on the means and medians it appears that, once again, altering the
aspect ratio had minimal, if any, effect on interpretation of the data
value. The value for the default plot also appears to be closer to the
true value than the control plot in part 1, question 1.

\begin{Shaded}
\begin{Highlighting}[]
\NormalTok{brks <-}\StringTok{ }\KeywordTok{c}\NormalTok{(}\StringTok{"Default"}\NormalTok{, }\StringTok{"wide"}\NormalTok{, }\StringTok{"narrow"}\NormalTok{)}
\NormalTok{vals <-}\StringTok{ }\KeywordTok{c}\NormalTok{(}\StringTok{"#1c9e77"}\NormalTok{, }\StringTok{"#d95f02"}\NormalTok{, }\StringTok{"#7570b3"}\NormalTok{)}

\NormalTok{resp <-}\StringTok{ }\KeywordTok{c}\NormalTok{(default, wide, narrow)}
\NormalTok{type <-}\StringTok{ }\KeywordTok{c}\NormalTok{(}\KeywordTok{rep}\NormalTok{(}\StringTok{'Default'}\NormalTok{, }\DecValTok{70}\NormalTok{), }\KeywordTok{rep}\NormalTok{(}\StringTok{'wide'}\NormalTok{, }\DecValTok{70}\NormalTok{), }\KeywordTok{rep}\NormalTok{(}\StringTok{'narrow'}\NormalTok{, }\DecValTok{70}\NormalTok{)) }
\NormalTok{stats <-}\StringTok{ }\KeywordTok{data.frame}\NormalTok{(resp, type)}

\KeywordTok{ggplot}\NormalTok{(}\DataTypeTok{data =}\NormalTok{ stats, }\KeywordTok{aes}\NormalTok{(}\DataTypeTok{x=}\NormalTok{type, }\DataTypeTok{y=}\NormalTok{resp, }\DataTypeTok{fill=}\NormalTok{type))}\OperatorTok{+}
\StringTok{  }\KeywordTok{geom_boxplot}\NormalTok{(}\DataTypeTok{outlier.colour=}\StringTok{"black"}\NormalTok{, }\DataTypeTok{outlier.shape=}\DecValTok{1}\NormalTok{,}
             \DataTypeTok{outlier.size=}\DecValTok{2}\NormalTok{, }\DataTypeTok{notch=}\NormalTok{F)}\OperatorTok{+}
\StringTok{  }\KeywordTok{theme_classic}\NormalTok{()}\OperatorTok{+}
\StringTok{  }\KeywordTok{stat_summary}\NormalTok{(}\DataTypeTok{fun=}\NormalTok{mean, }\DataTypeTok{geom=}\StringTok{"point"}\NormalTok{, }\DataTypeTok{shape=}\DecValTok{15}\NormalTok{, }\DataTypeTok{size=}\DecValTok{4}\NormalTok{)}\OperatorTok{+}
\StringTok{  }\KeywordTok{scale_fill_manual}\NormalTok{(}\DataTypeTok{breaks =}\NormalTok{ brks, }\DataTypeTok{values =}\NormalTok{ vals)}\OperatorTok{+}
\StringTok{  }\KeywordTok{ylab}\NormalTok{(}\StringTok{"Value"}\NormalTok{)}\OperatorTok{+}
\StringTok{  }\KeywordTok{xlab}\NormalTok{(}\StringTok{"Plot Type"}\NormalTok{)}
\end{Highlighting}
\end{Shaded}

\begin{figure}
\centering
\includegraphics{main_files/figure-latex/unnamed-chunk-36-1.pdf}
\caption{Box plots showing distributions of responses regarding the
three plots}
\end{figure}

Looking at the box plots, there are very small ranges in the values,
signifying that there was a large consensus between respondents in terms
of what they perceived the height to be. It can also be seen that there
are three outliers below the box plot for the narrow plot responses, and
two above for the default plot responses. There is very little overlap
between the boxes, and it appears again that there altering the aspect
ratio of the bar plot has little to no impact on reading the height of
the bar. Additionally, there was less agreement between respondents for
the wide plot than for the other two, although this doesn't seem to be
too significant.

\begin{Shaded}
\begin{Highlighting}[]
\KeywordTok{ggplot}\NormalTok{() }\OperatorTok{+}
\StringTok{  }\KeywordTok{geom_density}\NormalTok{(}\DataTypeTok{data =} \KeywordTok{as.data.frame}\NormalTok{(default ), }\KeywordTok{aes}\NormalTok{(}\DataTypeTok{x=}\NormalTok{default , }\DataTypeTok{col =} \StringTok{"Default"}\NormalTok{))}\OperatorTok{+}
\StringTok{  }\KeywordTok{geom_density}\NormalTok{(}\DataTypeTok{data =} \KeywordTok{as.data.frame}\NormalTok{(wide   ), }\KeywordTok{aes}\NormalTok{(}\DataTypeTok{x=}\NormalTok{wide   , }\DataTypeTok{col =} \StringTok{"wide"}\NormalTok{))}\OperatorTok{+}
\StringTok{  }\KeywordTok{geom_density}\NormalTok{(}\DataTypeTok{data =} \KeywordTok{as.data.frame}\NormalTok{(narrow), }\KeywordTok{aes}\NormalTok{(}\DataTypeTok{x=}\NormalTok{narrow, }\DataTypeTok{col =} \StringTok{"narrow"}\NormalTok{))}\OperatorTok{+}
\StringTok{  }\KeywordTok{labs}\NormalTok{(}\DataTypeTok{x=}\StringTok{"Response"}\NormalTok{, }\DataTypeTok{y=}\StringTok{"Density"}\NormalTok{)}\OperatorTok{+}
\StringTok{  }\KeywordTok{scale_colour_manual}\NormalTok{(}\DataTypeTok{name =} \StringTok{" "}\NormalTok{, }\DataTypeTok{breaks =}\NormalTok{ brks, }\DataTypeTok{values =}\NormalTok{ vals)}\OperatorTok{+}
\StringTok{  }\KeywordTok{theme_classic}\NormalTok{()}
\end{Highlighting}
\end{Shaded}

\includegraphics{main_files/figure-latex/unnamed-chunk-37-1.pdf} The
distributions for the default and narrow plot responses are similar,
both seeming to be fairly centred on the mean with a steep decrease in
density on either side of the mean to shallow tails within the range
\([25, 30]\). The responses for the wide plot appear to be more spread
with lower density function values, with a slight negative skew.

\begin{Shaded}
\begin{Highlighting}[]
\NormalTok{default <-}\StringTok{ }\NormalTok{default[}\OperatorTok{-}\KeywordTok{which}\NormalTok{(default }\OperatorTok{>=}\StringTok{ }\DecValTok{30}\NormalTok{)]}
\NormalTok{narrow <-}\StringTok{ }\NormalTok{narrow[}\OperatorTok{-}\KeywordTok{which}\NormalTok{(narrow }\OperatorTok{<=}\StringTok{ }\DecValTok{25}\NormalTok{)]}

\NormalTok{ratio_}\DecValTok{1}\NormalTok{_all <-}\StringTok{ }\KeywordTok{cbind}\NormalTok{(default, wide, narrow)}
\end{Highlighting}
\end{Shaded}

\begin{verbatim}
## Warning in cbind(default, wide, narrow): number of rows of result is not a
## multiple of vector length (arg 1)
\end{verbatim}

After removing the outliers the medians have stayed the same, and the
mean has obviously decreased for the default and increased for the
narrow, however, these means are all still fairly similar to each other
and at a first glance prior to testing it again seems that changing the
aspect ratio, at least to the degree tested here, is inconsequential to
interpretation of the actual value. As expected as well, the variances
for the outlier-removed sets have decreased.

However, statistical tests do actually show that while the default
responses did not differ significantly from the true value of 28
(\(p=0.5667\)), the responses for the narrow plot did
(\(p=2.0955e-09\)), but the wide didn't (\(p=0.5067\)).

Changing the language and plot order was once again inconsequential
here.

\subsection{Differences Between Question 1 and 2 Responses}

The last set of questions in part 2 show respondents all three of the
bar plots presented in this section and ask them to select which they
find most aesthetically pleasing, and which they find easiest and
hardest to interpret. Table{[}?{]} gives the number of respondents that
selected each plot for each of the three questions.

\begin{table}[!h]
\centering
\begin{tabular}{l|r|r|r}
\hline
  & Default & Narrow & Wide\\
\hline
Most aesthetically pleasing? & 37 & 14 & 18\\
\hline
Easiest to read and interpret? & 36 & 15 & 19\\
\hline
Hardest to read and interpret? & 20 & 20 & 30\\
\hline
\end{tabular}
\end{table}

For the first question, relating to how aesthetically pleasing
respondents found each plot, just over half of the respondents chose the
default aspect ratio as the most aesthetically pleasing, with 37 out of
the 69 who responded selecting this.

Similarly, 37 out of the 70 that responded to the second question found
the plot with the default aspect ratio easiest to read and interpret.
Perhaps the people that preferred this aspect ratio aesthetically did so
because they found it easiest to interpret. Investigating this, 27
respondents who chose the default for question 1 also chose this for
question 2.

The plot judged hardest to read and interpret by the most respondents
was the one with the wide bars, with 30 selecting this and 20 selecting
each of the other two. While a significant number chose the default and
narrow bars, the slightly higher amount selecting the plot with wide
bars matches the previously stated hypothesis formulated from following
the Stephen Few paper, which discusses that an ratio of greater width to
length could suffer from perceptual imbalance. While this imbalance
isn't seen in the numbers from the previous questions, the result here
does give some indication that the aspect ratio producing wide bars may
impact on ease of interpretation.

\section{American Ninja Warrior - Part 3}

The third and final part of the questions about the American Ninja
Warrior data discusses stacked bars and colour schemes. The questions
asked in this part are used to decipher how data with multiple
categories may be best represented in a bar plot. The plots presented
use the same bars as in part 1, but this time the number of times each
obstacle was used in each stage of the competition for each bar is
highlighted. Each participant was shown both a stacked and a grouped bar
plot in one of three colour schemes; the default for the language,
viridis, and greyscale. For three versions of the survey, the stacked
bars were shown first, and for the other three versions the first shown
was the grouped bars. The final question of this part also asked
respondents to compare two colour schemes, and through the 6 surveys
there are comparisons of every colour scheme against every other colour
scheme.

The question
\textit{"How many times would you say 'Floating Steps' were used in the Finals (Regional/City) round?"}
is the first here, and is regarding the reading of a numerical value off
the axis. In this question respondents were asked about `Floating
Steps', which is the bar third along from the y-axis. The question asks
respondents to view the bar plot, where the bars will either be grouped
of stacked, and decipher how many times this obstacle was used in the
specified round of the competition. The true value for this was 11. The
hypothesis for this question is that the respondents will more
accurately gauge the value for the grouped bar than the stacked, which
as seen below appears to be the case.

\begin{Shaded}
\begin{Highlighting}[]
\NormalTok{vir_stacked_}\DecValTok{1}\NormalTok{ <-}\StringTok{ }\NormalTok{vir_stacked}\OperatorTok{$}\NormalTok{vir_sta_}\DecValTok{1}
\NormalTok{def_stacked_}\DecValTok{1}\NormalTok{ <-}\StringTok{ }\NormalTok{def_stacked}\OperatorTok{$}\NormalTok{def_sta_}\DecValTok{1}
\NormalTok{gr_stacked_}\DecValTok{1}\NormalTok{ <-}\StringTok{ }\NormalTok{gr_stacked}\OperatorTok{$}\NormalTok{gr_sta_}\DecValTok{1}

\NormalTok{vir_grouped_}\DecValTok{1}\NormalTok{ <-}\StringTok{ }\NormalTok{vir_grouped}\OperatorTok{$}\NormalTok{vir_grp_}\DecValTok{1}
\NormalTok{def_grouped_}\DecValTok{1}\NormalTok{ <-}\StringTok{ }\NormalTok{def_grouped}\OperatorTok{$}\NormalTok{def_grp_}\DecValTok{1}
\NormalTok{gr_grouped_}\DecValTok{1}\NormalTok{ <-}\StringTok{ }\NormalTok{gr_grouped}\OperatorTok{$}\NormalTok{gr_grp_}\DecValTok{1}

\NormalTok{stacked_}\DecValTok{1}\NormalTok{ <-}\StringTok{ }\KeywordTok{c}\NormalTok{(vir_stacked_}\DecValTok{1}\NormalTok{, def_stacked_}\DecValTok{1}\NormalTok{, gr_stacked_}\DecValTok{1}\NormalTok{)}
\NormalTok{grouped_}\DecValTok{1}\NormalTok{ <-}\StringTok{ }\KeywordTok{c}\NormalTok{(vir_grouped_}\DecValTok{1}\NormalTok{, def_grouped_}\DecValTok{1}\NormalTok{, gr_grouped_}\DecValTok{1}\NormalTok{)}

\NormalTok{bars <-}\StringTok{ }\KeywordTok{data.frame}\NormalTok{(stacked_}\DecValTok{1}\NormalTok{, grouped_}\DecValTok{1}\NormalTok{)}
\end{Highlighting}
\end{Shaded}

The mean for the values estimated by respondents using the stacked bars
is 14.32, a fair bit larger than the true value of 11, and the mean
estimated value for the grouped bars was closer to the true value, at
11.8. The IQR for the grouped bars is also smaller than for the stacked,
and comprises of the range \([11, 12]\), insinuating that the estimated
values tended to be fairly accurate but with some respondents perhaps
slightly overestimating. The IQR for the stacked bars on the other hand
covers the interval \([10, 14]\), which does contain the true value, but
shows a tendency for both over and underestimation of respondents.
Additionally to this, there is a large variance in the responses to this
question, at 54.8 compared to the variance of 13.1 for the responses
regarding the grouped bar plots. This adds to the picture that there was
much less agreement between respondents, with many straying away from
the mean of 14.3. It is seen however that the median for both the
stacked and grouped bars is 11, showing that the higher mean of the
stacked bars may be a result of an influential value at the upper end of
the distribution, and that many observations do actually sit around 11.
The fact that many values actually sit around 11 could be contributing
to the higher variance, as variance is simply the sum of the squared
distances from the mean, and so will be elevated if there are many
values that sit some distance away from the mean. The higher mean could
be reflected in the maximum of the stacked responses being 35, although
the maximum of the grouped responses is 40, so there may be more than
one influential point in the stacked responses. Outliers can be checked
for by looking at the box plots for this data.

\begin{Shaded}
\begin{Highlighting}[]
\NormalTok{brks <-}\StringTok{ }\KeywordTok{c}\NormalTok{(}\StringTok{"Stacked"}\NormalTok{, }\StringTok{"Grouped"}\NormalTok{)}
\NormalTok{vals <-}\StringTok{ }\KeywordTok{c}\NormalTok{(}\StringTok{"#1c9e77"}\NormalTok{, }\StringTok{"#d95f02"}\NormalTok{)}

\NormalTok{resp <-}\StringTok{ }\KeywordTok{c}\NormalTok{(stacked_}\DecValTok{1}\NormalTok{, grouped_}\DecValTok{1}\NormalTok{)}
\NormalTok{type <-}\StringTok{ }\KeywordTok{c}\NormalTok{(}\KeywordTok{rep}\NormalTok{(}\StringTok{'Stacked'}\NormalTok{, }\DecValTok{70}\NormalTok{), }\KeywordTok{rep}\NormalTok{(}\StringTok{'Grouped'}\NormalTok{, }\DecValTok{70}\NormalTok{)) }
\NormalTok{stats_}\DecValTok{1}\NormalTok{ <-}\StringTok{ }\KeywordTok{data.frame}\NormalTok{(resp, type)}

\KeywordTok{ggplot}\NormalTok{(}\DataTypeTok{data =}\NormalTok{ stats_}\DecValTok{1}\NormalTok{, }\KeywordTok{aes}\NormalTok{(}\DataTypeTok{x=}\NormalTok{type, }\DataTypeTok{y=}\NormalTok{resp, }\DataTypeTok{fill=}\NormalTok{type))}\OperatorTok{+}
\StringTok{  }\KeywordTok{geom_boxplot}\NormalTok{(}\DataTypeTok{outlier.colour=}\StringTok{"black"}\NormalTok{, }\DataTypeTok{outlier.shape=}\DecValTok{1}\NormalTok{,}
             \DataTypeTok{outlier.size=}\DecValTok{2}\NormalTok{, }\DataTypeTok{notch=}\NormalTok{F)}\OperatorTok{+}
\StringTok{  }\KeywordTok{theme_classic}\NormalTok{()}\OperatorTok{+}
\StringTok{  }\KeywordTok{stat_summary}\NormalTok{(}\DataTypeTok{fun=}\NormalTok{mean, }\DataTypeTok{geom=}\StringTok{"point"}\NormalTok{, }\DataTypeTok{shape=}\DecValTok{15}\NormalTok{, }\DataTypeTok{size=}\DecValTok{4}\NormalTok{)}\OperatorTok{+}
\StringTok{  }\KeywordTok{scale_fill_manual}\NormalTok{(}\DataTypeTok{breaks =}\NormalTok{ brks, }\DataTypeTok{values =}\NormalTok{ vals)}\OperatorTok{+}
\StringTok{  }\KeywordTok{scale_y_continuous}\NormalTok{(}\DataTypeTok{labels =} \KeywordTok{seq}\NormalTok{(}\DecValTok{1}\NormalTok{, }\DecValTok{7}\NormalTok{, }\DecValTok{1}\NormalTok{), }\DataTypeTok{breaks =} \KeywordTok{seq}\NormalTok{(}\DecValTok{1}\NormalTok{, }\DecValTok{7}\NormalTok{, }\DecValTok{1}\NormalTok{))}\OperatorTok{+}
\StringTok{  }\KeywordTok{ylab}\NormalTok{(}\StringTok{"Value"}\NormalTok{)}\OperatorTok{+}
\StringTok{  }\KeywordTok{xlab}\NormalTok{(}\StringTok{"Plot Type"}\NormalTok{)}
\end{Highlighting}
\end{Shaded}

\includegraphics{main_files/figure-latex/unnamed-chunk-41-1.pdf} It can
in fact be seen that the box for the grouped responses is short and
centered around 11. The box for the stacked responses shows many high
valued outliers that could be causing the mean to be higher, although
the IQR is still a fair bit larger than that of the responses for the
grouped bars. The mean for this also sits above the IQR, and thus the
outliers may be having a significant influence. Now the outliers will be
removed, assuming, from the box plot, that outliers are any values above
or equal to 25 for the stacked responses and above or equal to 20 for
the grouped.

\begin{Shaded}
\begin{Highlighting}[]
\NormalTok{stacked_}\DecValTok{1}\NormalTok{ <-}\StringTok{ }\KeywordTok{c}\NormalTok{(vir_stacked_}\DecValTok{1}\NormalTok{, def_stacked_}\DecValTok{1}\NormalTok{, gr_stacked_}\DecValTok{1}\NormalTok{)}
\NormalTok{grouped_}\DecValTok{1}\NormalTok{ <-}\StringTok{ }\KeywordTok{c}\NormalTok{(vir_grouped_}\DecValTok{1}\NormalTok{, def_grouped_}\DecValTok{1}\NormalTok{, gr_grouped_}\DecValTok{1}\NormalTok{)}
\end{Highlighting}
\end{Shaded}

Removing the outliers as specified by the box plot, the mean of the
stacked responses is now just above 11, and actually closer to the true
value than the mean of the other set of responses, and the median has
decreased to 10. From this one could infer that there is no difference
between each type of bar plot in terms of gauging the size of the bars.
However, there are 12 outliers in the stacked responses, which leads to
the idea that these are not in fact all outliers and may be valid
responses that just sit on the upper end of the distribution. However,
it seems the cause of the high values could be respondents taking the
whole height of the bar, which has an actual height of 28, rather than
the section of interest. Many of the potentially influential values fall
around the range \([25, 30]\), with all but 2 of the 12 potential
outliers sitting in this interval, with the remaining two both being 35.
Looking below at the summary statistics for only the values picked up as
outliers, there is a mean of 29.83, which is higher than the true value
of 28, and interestingly goes against the analysis from part 1, question
2 whereby respondents were asked to judge the height of this bar and on
average underestimated. The fact that so many participants
misinterpreted this plot and signify that stacked bar plots may not be
the best way to present data to general public, as there may be the
potential to misread the height of the whole bar as the size of the top
category.

\begin{Shaded}
\begin{Highlighting}[]
\NormalTok{vir_stacked_}\DecValTok{1}\NormalTok{ <-}\StringTok{ }\NormalTok{vir_stacked}\OperatorTok{$}\NormalTok{vir_sta_}\DecValTok{1}
\NormalTok{def_stacked_}\DecValTok{1}\NormalTok{ <-}\StringTok{ }\NormalTok{def_stacked}\OperatorTok{$}\NormalTok{def_sta_}\DecValTok{1}
\NormalTok{gr_stacked_}\DecValTok{1}\NormalTok{ <-}\StringTok{ }\NormalTok{gr_stacked}\OperatorTok{$}\NormalTok{gr_sta_}\DecValTok{1}

\NormalTok{vir_grouped_}\DecValTok{1}\NormalTok{ <-}\StringTok{ }\NormalTok{vir_stacked}\OperatorTok{$}\NormalTok{vir_grp_}\DecValTok{1}
\NormalTok{def_grouped_}\DecValTok{1}\NormalTok{ <-}\StringTok{ }\NormalTok{def_stacked}\OperatorTok{$}\NormalTok{def_grp_}\DecValTok{1}
\NormalTok{gr_grouped_}\DecValTok{1}\NormalTok{ <-}\StringTok{ }\NormalTok{gr_stacked}\OperatorTok{$}\NormalTok{gr_grp_}\DecValTok{1}

\NormalTok{stacked_}\DecValTok{1}\NormalTok{ <-}\StringTok{ }\KeywordTok{c}\NormalTok{(vir_stacked_}\DecValTok{1}\NormalTok{, def_stacked_}\DecValTok{1}\NormalTok{, gr_stacked_}\DecValTok{1}\NormalTok{)}
\NormalTok{grouped_}\DecValTok{1}\NormalTok{ <-}\StringTok{ }\KeywordTok{c}\NormalTok{(vir_grouped_}\DecValTok{1}\NormalTok{, def_grouped_}\DecValTok{1}\NormalTok{, gr_grouped_}\DecValTok{1}\NormalTok{)}

\NormalTok{stacked_}\DecValTok{1}\NormalTok{ <-}\StringTok{ }\NormalTok{stacked_}\DecValTok{1}\NormalTok{[}\KeywordTok{which}\NormalTok{(stacked_}\DecValTok{1} \OperatorTok{>=}\StringTok{ }\DecValTok{25}\NormalTok{)] }
\end{Highlighting}
\end{Shaded}

As a result of this, this set of 12 values will be discounted from the
analysis, and thus come to the conclusion that, for the respondents that
appear to have judged the height of the correct section, there was
little to no impact when using stacked vs grouped bar charts, and most
of the difference comes from misinterpretation of the plot itself, as
opposed to a poorer judgment of size.

To see if either of these values are significantly far from the true
value, tests are once again run. A sign test on the stacked bar plot
responses gives a high p-value of 0.5258, showing that for the stacked
bar plot responses (after removing the values as priorly specified), the
participant estimated values do not differ significantly from the true
value. For the grouped bar plot the obtained p-value is 0.009
\textless{} 0.05, and thus these responses are statistically
significantly different from the true value.

The next question,
\textit{'How many times would you say 'Log Grip' was used in the Finals (Regional/City) round?'},
is similar the above, but for the next bar to the right. The purpose of
this question was to test the same hypothesis as the previous question,
and also to lead into the following question, where respondents were
asked to compare the `Floating Steps' and `Log Grip'. Additionally, the
bar in the previous question had only two categories, of which the
respondents were asked to judge the size of the category on the top of
the bar in the stacked plot, whereas the bar for `Log Grip' has 5
categories, of which the category of interest sits above 4. The true
value of this was 9.

\begin{Shaded}
\begin{Highlighting}[]
\NormalTok{vir_stacked_}\DecValTok{2}\NormalTok{ <-}\StringTok{ }\NormalTok{vir_stacked}\OperatorTok{$}\NormalTok{vir_sta_}\DecValTok{2}
\NormalTok{def_stacked_}\DecValTok{2}\NormalTok{ <-}\StringTok{ }\NormalTok{def_stacked}\OperatorTok{$}\NormalTok{def_sta_}\DecValTok{2}
\NormalTok{gr_stacked_}\DecValTok{2}\NormalTok{ <-}\StringTok{ }\NormalTok{gr_stacked}\OperatorTok{$}\NormalTok{gr_sta_}\DecValTok{2}

\NormalTok{vir_grouped_}\DecValTok{2}\NormalTok{ <-}\StringTok{ }\NormalTok{vir_grouped}\OperatorTok{$}\NormalTok{vir_grp_}\DecValTok{2}
\NormalTok{def_grouped_}\DecValTok{2}\NormalTok{ <-}\StringTok{ }\NormalTok{def_grouped}\OperatorTok{$}\NormalTok{def_grp_}\DecValTok{2}
\NormalTok{gr_grouped_}\DecValTok{2}\NormalTok{ <-}\StringTok{ }\NormalTok{gr_grouped}\OperatorTok{$}\NormalTok{gr_grp_}\DecValTok{2}

\NormalTok{stacked_}\DecValTok{2}\NormalTok{ <-}\StringTok{ }\KeywordTok{c}\NormalTok{(vir_stacked_}\DecValTok{2}\NormalTok{, def_stacked_}\DecValTok{2}\NormalTok{, gr_stacked_}\DecValTok{2}\NormalTok{)}
\NormalTok{grouped_}\DecValTok{2}\NormalTok{ <-}\StringTok{ }\KeywordTok{c}\NormalTok{(vir_grouped_}\DecValTok{2}\NormalTok{, def_grouped_}\DecValTok{2}\NormalTok{, gr_grouped_}\DecValTok{2}\NormalTok{)}

\NormalTok{bars <-}\StringTok{ }\KeywordTok{data.frame}\NormalTok{(stacked_}\DecValTok{2}\NormalTok{, grouped_}\DecValTok{2}\NormalTok{)}
\end{Highlighting}
\end{Shaded}

Similarly to the previous question, the mean response for the stacked
bar plots are higher than that of the grouped, and the mean of the
stacked also slightly overestimates the value. Once again however, a
selection of respondents appeared to judge the full height of the bar
rather than the category as asked.

\begin{Shaded}
\begin{Highlighting}[]
\NormalTok{brks <-}\StringTok{ }\KeywordTok{c}\NormalTok{(}\StringTok{"Stacked"}\NormalTok{, }\StringTok{"Grouped"}\NormalTok{)}
\NormalTok{vals <-}\StringTok{ }\KeywordTok{c}\NormalTok{(}\StringTok{"#1c9e77"}\NormalTok{, }\StringTok{"#d95f02"}\NormalTok{)}

\NormalTok{resp <-}\StringTok{ }\KeywordTok{c}\NormalTok{(stacked_}\DecValTok{2}\NormalTok{, grouped_}\DecValTok{2}\NormalTok{)}
\NormalTok{type <-}\StringTok{ }\KeywordTok{c}\NormalTok{(}\KeywordTok{rep}\NormalTok{(}\StringTok{'Stacked'}\NormalTok{, }\DecValTok{70}\NormalTok{), }\KeywordTok{rep}\NormalTok{(}\StringTok{'Grouped'}\NormalTok{, }\DecValTok{70}\NormalTok{)) }
\NormalTok{stats_}\DecValTok{2}\NormalTok{ <-}\StringTok{ }\KeywordTok{data.frame}\NormalTok{(resp, type)}

\KeywordTok{ggplot}\NormalTok{(}\DataTypeTok{data =}\NormalTok{ stats_}\DecValTok{2}\NormalTok{)}\OperatorTok{+}
\StringTok{  }\KeywordTok{geom_bar}\NormalTok{(}\KeywordTok{aes}\NormalTok{(}\DataTypeTok{x=}\NormalTok{resp, }\DataTypeTok{group =}\NormalTok{ type, }\DataTypeTok{fill =}\NormalTok{ type), }\DataTypeTok{position =} \StringTok{'dodge'}\NormalTok{)}\OperatorTok{+}
\StringTok{  }\KeywordTok{theme_classic}\NormalTok{()}\OperatorTok{+}
\StringTok{  }\KeywordTok{scale_fill_manual}\NormalTok{(}\DataTypeTok{breaks =}\NormalTok{ brks, }\DataTypeTok{values =}\NormalTok{ vals)}\OperatorTok{+}
\StringTok{  }\KeywordTok{ylab}\NormalTok{(}\StringTok{"Value"}\NormalTok{)}\OperatorTok{+}
\StringTok{  }\KeywordTok{xlab}\NormalTok{(}\StringTok{"Plot Type"}\NormalTok{)}
\end{Highlighting}
\end{Shaded}

\includegraphics{main_files/figure-latex/unnamed-chunk-45-1.pdf} Indeed,
the distributions of values for each of the two response sets appear to
be almost identical.

\begin{Shaded}
\begin{Highlighting}[]
\NormalTok{stacked_}\DecValTok{2}\NormalTok{ <-}\StringTok{ }\KeywordTok{c}\NormalTok{(vir_stacked_}\DecValTok{2}\NormalTok{, def_stacked_}\DecValTok{2}\NormalTok{, gr_stacked_}\DecValTok{2}\NormalTok{)}
\NormalTok{grouped_}\DecValTok{2}\NormalTok{ <-}\StringTok{ }\KeywordTok{c}\NormalTok{(vir_grouped_}\DecValTok{2}\NormalTok{, def_grouped_}\DecValTok{2}\NormalTok{, gr_grouped_}\DecValTok{2}\NormalTok{)}
\end{Highlighting}
\end{Shaded}

After removing the outlying values, there tended to be a slight
underestimation in the value for the stacked bar plot, however this is
approximately 0.46 away from the true value, and unlikely to be
significant.

Once again the response sets are non-normally distributed and
asymmetric, and so sign tests are applicable. The response set for the
stacked bar plots produces a p-value of around 0.04, which shows a
statistically significant difference in the responses from the true
value of 9 at the 0.05 level of significance. However, this would easily
become insignificant by slightly lowering the significance level to,
say, 0.035. The p-value for the grouped bar responses, however, is
\textgreater\textgreater{} 0.05, as expected given that the median of
the data sits at the true value.

The respondents were then asked to
\textit{'How many times would you say 'Log Grip' was used in the Finals (Regional/City) round?'}.
This question asked respondents to judge whether log grip was used more,
less, or an equal amount in the Finals (Regional/City) and
Qualifying(Regional/City) rounds. This was to see how well differences
between sizes of categories are judged when relating to the same
variable, and are in the same bar. The results for this are given in the
table below.

\begin{Shaded}
\begin{Highlighting}[]
\NormalTok{vir_stacked_}\DecValTok{3}\NormalTok{ <-}\StringTok{ }\NormalTok{vir_stacked}\OperatorTok{$}\NormalTok{vir_sta_}\DecValTok{3}
\NormalTok{def_stacked_}\DecValTok{3}\NormalTok{ <-}\StringTok{ }\NormalTok{def_stacked}\OperatorTok{$}\NormalTok{def_sta_}\DecValTok{3}
\NormalTok{gr_stacked_}\DecValTok{3}\NormalTok{ <-}\StringTok{ }\NormalTok{gr_stacked}\OperatorTok{$}\NormalTok{gr_sta_}\DecValTok{3}

\NormalTok{vir_grouped_}\DecValTok{3}\NormalTok{ <-}\StringTok{ }\NormalTok{vir_stacked}\OperatorTok{$}\NormalTok{vir_grp_}\DecValTok{3}
\NormalTok{def_grouped_}\DecValTok{3}\NormalTok{ <-}\StringTok{ }\NormalTok{def_stacked}\OperatorTok{$}\NormalTok{def_grp_}\DecValTok{3}
\NormalTok{gr_grouped_}\DecValTok{3}\NormalTok{ <-}\StringTok{ }\NormalTok{gr_stacked}\OperatorTok{$}\NormalTok{gr_grp_}\DecValTok{3}

\NormalTok{stacked_}\DecValTok{3}\NormalTok{ <-}\StringTok{ }\KeywordTok{c}\NormalTok{(vir_stacked_}\DecValTok{3}\NormalTok{, def_stacked_}\DecValTok{3}\NormalTok{, gr_stacked_}\DecValTok{3}\NormalTok{)}
\NormalTok{grouped_}\DecValTok{3}\NormalTok{ <-}\StringTok{ }\KeywordTok{c}\NormalTok{(vir_grouped_}\DecValTok{3}\NormalTok{, def_grouped_}\DecValTok{3}\NormalTok{, gr_grouped_}\DecValTok{3}\NormalTok{)}
\end{Highlighting}
\end{Shaded}

The table shows overwhelmingly that significantly more people accurately
judged that the two values were the same for the grouped bars than for
the stacked bars. This was the hypothesised result, and has presented to
an even greater extent than previously anticipated. All but 7 of the
respondents who responded to this question correctly judged from the
grouped bars that the obstacle was used an equal number of times in each
of the two rounds, whereas the responses for the grouped bar seemed
fairly well split between the three options. It may be interesting in
the multivariate analysis section to compare responses depending on
whether respondents were shown the stacked or grouped bars first.
Perhaps a reason for the incorrect judging with the stacked

Respondents were then asked
\textit{'Which obstacle do you think was used MORE in Finals (Regional/City) rounds, 'Log Grip' or 'Floating Steps'?'}Similar
to the previous question, this asks for a comparison between the size of
two categories, but this time about how many times two different
obstacles were used in the round Finals (Regional/City), where these two
obstacles are those discussed at the start of this part of the survey.

\begin{Shaded}
\begin{Highlighting}[]
\NormalTok{vir_stacked_}\DecValTok{4}\NormalTok{ <-}\StringTok{ }\NormalTok{vir_stacked}\OperatorTok{$}\NormalTok{vir_sta_}\DecValTok{4}
\NormalTok{def_stacked_}\DecValTok{4}\NormalTok{ <-}\StringTok{ }\NormalTok{def_stacked}\OperatorTok{$}\NormalTok{def_sta_}\DecValTok{4}
\NormalTok{gr_stacked_}\DecValTok{4}\NormalTok{ <-}\StringTok{ }\NormalTok{gr_stacked}\OperatorTok{$}\NormalTok{gr_sta_}\DecValTok{4}

\NormalTok{vir_grouped_}\DecValTok{4}\NormalTok{ <-}\StringTok{ }\NormalTok{vir_stacked}\OperatorTok{$}\NormalTok{vir_grp_}\DecValTok{4}
\NormalTok{def_grouped_}\DecValTok{4}\NormalTok{ <-}\StringTok{ }\NormalTok{def_stacked}\OperatorTok{$}\NormalTok{def_grp_}\DecValTok{4}
\NormalTok{gr_grouped_}\DecValTok{4}\NormalTok{ <-}\StringTok{ }\NormalTok{gr_stacked}\OperatorTok{$}\NormalTok{gr_grp_}\DecValTok{4}

\NormalTok{stacked_}\DecValTok{4}\NormalTok{ <-}\StringTok{ }\KeywordTok{c}\NormalTok{(vir_stacked_}\DecValTok{4}\NormalTok{, def_stacked_}\DecValTok{4}\NormalTok{, gr_stacked_}\DecValTok{4}\NormalTok{)}
\NormalTok{grouped_}\DecValTok{4}\NormalTok{ <-}\StringTok{ }\KeywordTok{c}\NormalTok{(vir_grouped_}\DecValTok{4}\NormalTok{, def_grouped_}\DecValTok{4}\NormalTok{, gr_grouped_}\DecValTok{4}\NormalTok{)}
\end{Highlighting}
\end{Shaded}

This was a potentially poorly formulated question, as the respondents
had already been asked to specify how many times each of these obstacle
was used in this round and respondents mostly judged this accurately
with regard to both plots, but this could have been impacted by the
previous questions. However, this does follow from the results from the
past questions showing that respondents mostly accurately judged the
values correctly, aside from those who instead judged the height of the
whole bar.

The aim of the question
\textit{'Which bar chart do you feel is easiest to read and interpret?'}
was to assess the perceived ease of interpretation of both bar plots.
This is to gain an understanding in how data may best be presented in an
easily understandable, easily readable manner. This is an important
factor in visualisation, as a main aim in creating visuals is to provide
an aid for the viewer to simply and quickly see the message. The
opposite may be beneficial in certain applications however; based on the
misreadings in the question regarding judging the number of times `Log
Grip' was used in the specific round, viewers of the visualisations
could be easily mislead by incorrectly interpreting the plot. The people
being shown the plot in, for example, an advert, may only take a
fleeting look and not go beyond to analyse the plot to see accurate
differences between values, and thus it is important to produce a plot
that gives the easiest interpretation.

\begin{Shaded}
\begin{Highlighting}[]
\NormalTok{a_}\DecValTok{1}\NormalTok{ <-}\StringTok{ }\NormalTok{set_a}\OperatorTok{$}\NormalTok{sta_grp}
\NormalTok{b_}\DecValTok{1}\NormalTok{ <-}\StringTok{ }\NormalTok{set_b}\OperatorTok{$}\NormalTok{sta_grp}
\NormalTok{c_}\DecValTok{1}\NormalTok{ <-}\StringTok{ }\NormalTok{set_c}\OperatorTok{$}\NormalTok{sta_grp}
\NormalTok{d_}\DecValTok{1}\NormalTok{ <-}\StringTok{ }\NormalTok{set_d}\OperatorTok{$}\NormalTok{sta_grp}
\NormalTok{e_}\DecValTok{1}\NormalTok{ <-}\StringTok{ }\NormalTok{set_e}\OperatorTok{$}\NormalTok{sta_grp}
\NormalTok{f_}\DecValTok{1}\NormalTok{ <-}\StringTok{ }\NormalTok{set_f}\OperatorTok{$}\NormalTok{sta_grp}

\ControlFlowTok{for}\NormalTok{(i }\ControlFlowTok{in} \DecValTok{1}\OperatorTok{:}\KeywordTok{length}\NormalTok{(a_}\DecValTok{1}\NormalTok{))\{}
  \ControlFlowTok{if}\NormalTok{(a_}\DecValTok{1}\NormalTok{[i] }\OperatorTok{==}\StringTok{ "A"}\NormalTok{)\{}
\NormalTok{    a_}\DecValTok{1}\NormalTok{[i] <-}\StringTok{ "Stacked"}
\NormalTok{  \} }\ControlFlowTok{else}\NormalTok{ a_}\DecValTok{1}\NormalTok{[i] <-}\StringTok{ "Grouped"}
\NormalTok{\}}

\ControlFlowTok{for}\NormalTok{(i }\ControlFlowTok{in} \DecValTok{1}\OperatorTok{:}\KeywordTok{length}\NormalTok{(b_}\DecValTok{1}\NormalTok{))\{}
  \ControlFlowTok{if}\NormalTok{(b_}\DecValTok{1}\NormalTok{[i] }\OperatorTok{==}\StringTok{ "A"}\NormalTok{)\{}
\NormalTok{    b_}\DecValTok{1}\NormalTok{[i] <-}\StringTok{ "Stacked"}
\NormalTok{  \} }\ControlFlowTok{else}\NormalTok{ b_}\DecValTok{1}\NormalTok{[i] <-}\StringTok{ "Grouped"}
\NormalTok{\}}

\ControlFlowTok{for}\NormalTok{(i }\ControlFlowTok{in} \DecValTok{1}\OperatorTok{:}\KeywordTok{length}\NormalTok{(d_}\DecValTok{1}\NormalTok{))\{ }
  \ControlFlowTok{if}\NormalTok{(d_}\DecValTok{1}\NormalTok{[i] }\OperatorTok{==}\StringTok{ "A"}\NormalTok{)\{}
\NormalTok{    d_}\DecValTok{1}\NormalTok{[i] <-}\StringTok{ "Stacked"}
\NormalTok{  \} }\ControlFlowTok{else}\NormalTok{ d_}\DecValTok{1}\NormalTok{[i] <-}\StringTok{ "Grouped"}
\NormalTok{\}}

\ControlFlowTok{for}\NormalTok{(i }\ControlFlowTok{in} \DecValTok{1}\OperatorTok{:}\KeywordTok{length}\NormalTok{(c_}\DecValTok{1}\NormalTok{))\{}
  \ControlFlowTok{if}\NormalTok{(c_}\DecValTok{1}\NormalTok{[i] }\OperatorTok{==}\StringTok{ "B"}\NormalTok{)\{}
\NormalTok{    c_}\DecValTok{1}\NormalTok{[i] <-}\StringTok{ "Stacked"}
\NormalTok{  \} }\ControlFlowTok{else}\NormalTok{ c_}\DecValTok{1}\NormalTok{[i] <-}\StringTok{ "Grouped"}
\NormalTok{\}}

\ControlFlowTok{for}\NormalTok{(i }\ControlFlowTok{in} \DecValTok{1}\OperatorTok{:}\KeywordTok{length}\NormalTok{(e_}\DecValTok{1}\NormalTok{))\{}
  \ControlFlowTok{if}\NormalTok{(e_}\DecValTok{1}\NormalTok{[i] }\OperatorTok{==}\StringTok{ "B"}\NormalTok{)\{}
\NormalTok{    e_}\DecValTok{1}\NormalTok{[i] <-}\StringTok{ "Stacked"}
\NormalTok{  \} }\ControlFlowTok{else}\NormalTok{ e_}\DecValTok{1}\NormalTok{[i] <-}\StringTok{ "Grouped"}
\NormalTok{\}}

\ControlFlowTok{for}\NormalTok{(i }\ControlFlowTok{in} \DecValTok{1}\OperatorTok{:}\KeywordTok{length}\NormalTok{(f_}\DecValTok{1}\NormalTok{))\{}
  \ControlFlowTok{if}\NormalTok{(f_}\DecValTok{1}\NormalTok{[i] }\OperatorTok{==}\StringTok{ "B"}\NormalTok{)\{}
\NormalTok{    f_}\DecValTok{1}\NormalTok{[i] <-}\StringTok{ "Stacked"}
\NormalTok{  \} }\ControlFlowTok{else}\NormalTok{ f_}\DecValTok{1}\NormalTok{[i] <-}\StringTok{ "Grouped"}
\NormalTok{\}}

\NormalTok{tab <-}\StringTok{ }\KeywordTok{table}\NormalTok{(}\KeywordTok{c}\NormalTok{(a_}\DecValTok{1}\NormalTok{, b_}\DecValTok{1}\NormalTok{, c_}\DecValTok{1}\NormalTok{, d_}\DecValTok{1}\NormalTok{, e_}\DecValTok{1}\NormalTok{, f_}\DecValTok{1}\NormalTok{))}
\KeywordTok{kable}\NormalTok{(tab)}\OperatorTok
\StringTok{  }\KeywordTok{kable_styling}\NormalTok{(}\DataTypeTok{latex_options =} \StringTok{"hold_position"}\NormalTok{)}
\end{Highlighting}
\end{Shaded}

\begin{table}[!h]
\centering
\begin{tabular}{l|r}
\hline
Var1 & Freq\\
\hline
Grouped & 59\\
\hline
Stacked & 11\\
\hline
\end{tabular}
\end{table}

The large majority of participants found the grouped bar chart easier to
read and interpret, as predicted.

The questions
\textit{'Which bar chart do you feel is easiest to read and interpret?'}
and the one following
\textit{'Do you feel that one of the colour schemes makes it easier to read and interpret? If so, please select which one.'}
are asked with the purpose of assessing the colour scheme that gives the
greatest aesthetic pleasure, or effectively which colour palette the
respondents feel is subjectively the `prettiest' or `nicest'. It is
important to note here that aesthetics and readability do not always go
hand-in-hand; a plot that is made to look very aesthetically pleasing
may sacrifice readability, and vice versa. For each of the two
languages, six pairings of three different colour palettes were created,
whereby the first colour was the one displayed for the main questions,
and the second used only for the comparison questions. As previously
discussed, the three colour schemes considered are viridis, greyscale,
and each language's default plotting colour palette. The colour palette
pairings are outlined below, where each set of two colours is assigned a
`Pairing ID' from A to F.

\begin{Shaded}
\begin{Highlighting}[]
\NormalTok{set <-}\StringTok{ }\KeywordTok{c}\NormalTok{(}\StringTok{"A"}\NormalTok{, }\StringTok{"B"}\NormalTok{, }\StringTok{"C"}\NormalTok{, }\StringTok{"D"}\NormalTok{, }\StringTok{"E"}\NormalTok{, }\StringTok{"F"}\NormalTok{)}
\NormalTok{first_col <-}\StringTok{ }\KeywordTok{c}\NormalTok{(}\StringTok{"Viridis"}\NormalTok{, }\StringTok{"Default"}\NormalTok{, }\StringTok{"Default"}\NormalTok{, }\StringTok{"Greyscale"}\NormalTok{, }\StringTok{"Viridis"}\NormalTok{, }\StringTok{"Greyscale"}\NormalTok{)}
\NormalTok{second_col <-}\StringTok{ }\KeywordTok{c}\NormalTok{(}\StringTok{"Default"}\NormalTok{, }\StringTok{"Viridis"}\NormalTok{, }\StringTok{"Greyscale"}\NormalTok{, }\StringTok{"Default"}\NormalTok{, }\StringTok{"Greyscale"}\NormalTok{, }\StringTok{"Viridis"}\NormalTok{)}

\NormalTok{df <-}\StringTok{ }\KeywordTok{data.frame}\NormalTok{(set, first_col, second_col)}
\KeywordTok{names}\NormalTok{(df) <-}\StringTok{ }\KeywordTok{c}\NormalTok{(}\StringTok{"Pairing ID"}\NormalTok{, }\StringTok{"Main Palette"}\NormalTok{, }\StringTok{"Secondary Pallette"}\NormalTok{)}
\NormalTok{knitr}\OperatorTok{::}\KeywordTok{kable}\NormalTok{(df, }\DataTypeTok{caption =} \StringTok{"Colour pairings"}\NormalTok{)}\OperatorTok
\StringTok{  }\KeywordTok{kable_styling}\NormalTok{(}\DataTypeTok{latex_options =} \StringTok{"hold_position"}\NormalTok{)}
\end{Highlighting}
\end{Shaded}

\begin{table}[!h]

\caption{\label{tab:unnamed-chunk-50}Colour pairings}
\centering
\begin{tabular}[t]{l|l|l}
\hline
Pairing ID & Main Palette & Secondary Pallette\\
\hline
A & Viridis & Default\\
\hline
B & Default & Viridis\\
\hline
C & Default & Greyscale\\
\hline
D & Greyscale & Default\\
\hline
E & Viridis & Greyscale\\
\hline
F & Greyscale & Viridis\\
\hline
\end{tabular}
\end{table}

\begin{table}[!h]

\caption{\label{tab:unnamed-chunk-51}Easiest to read and interpret colour scheme}
\centering
\begin{tabular}[t]{l|r|r}
\hline
  & A & B\\
\hline
Set A & 7 & 6\\
\hline
Set B & 6 & 6\\
\hline
Set C & 9 & 1\\
\hline
Set D & 3 & 9\\
\hline
Set E & 11 & 0\\
\hline
Set F & 1 & 11\\
\hline
\end{tabular}
\end{table}

\begin{table}[!h]

\caption{\label{tab:unnamed-chunk-51}Easiest to read and interpret colour scheme, for R}
\centering
\begin{tabular}[t]{l|r|r}
\hline
  & A & B\\
\hline
Set A & 4 & 4\\
\hline
Set B & 2 & 4\\
\hline
Set C & 4 & 1\\
\hline
Set D & 2 & 5\\
\hline
Set E & 5 & 0\\
\hline
Set F & 1 & 6\\
\hline
\end{tabular}
\end{table}

\begin{table}[!h]

\caption{\label{tab:unnamed-chunk-51}Easiest to read and interpret colour scheme, for Python}
\centering
\begin{tabular}[t]{l|r|r}
\hline
  & A & B\\
\hline
Set A & 3 & 2\\
\hline
Set B & 4 & 2\\
\hline
Set C & 5 & 5\\
\hline
Set D & 1 & 4\\
\hline
Set E & 6 & 0\\
\hline
Set F & 5 & 5\\
\hline
\end{tabular}
\end{table}

This table shows that when it came to the default/viridis pairings,
displayed in the first two rows, the respondents tended to have no
preference overall. Comparing this to the bottom two rows, in which
viridis is put against greyscale, only 1 respondent out of the 23, a
proportion of 0.04, found the grey more aesthetically pleasing, as
hypothesised. When considering greyscale/default, there was still a
majority preferring the non-greyscale palette, but a higher proportion
preferred this as compared to the viridis/greyscale, with 4 out of the
22, or a proportion of 0.18, preferring the grey. Overall, 35 preferred
viridis, 30 the default, and 5 the greyscale.

As anticipated, the two more-colourful palettes are preferred
aesthetically over the grey, and the viridis was preferred over the
default.

Complementing the aesthetic preferences, the second question assesses
the colour preference with regard to readability and ease of
interpretation. As mentioned before, this will be used to test both the
colour palette preference itself alongside whether this preference
matches up with aesthetic preference.

\begin{Shaded}
\begin{Highlighting}[]
\NormalTok{a_}\DecValTok{2}\NormalTok{ <-}\StringTok{ }\NormalTok{set_a}\OperatorTok{$}\NormalTok{a_cols_}\DecValTok{2}
\NormalTok{b_}\DecValTok{2}\NormalTok{ <-}\StringTok{ }\NormalTok{set_b}\OperatorTok{$}\NormalTok{b_cols_}\DecValTok{2}
\NormalTok{c_}\DecValTok{2}\NormalTok{ <-}\StringTok{ }\NormalTok{set_c}\OperatorTok{$}\NormalTok{c_cols_}\DecValTok{2}
\NormalTok{d_}\DecValTok{2}\NormalTok{ <-}\StringTok{ }\NormalTok{set_d}\OperatorTok{$}\NormalTok{d_cols_}\DecValTok{2}
\NormalTok{e_}\DecValTok{2}\NormalTok{ <-}\StringTok{ }\NormalTok{set_e}\OperatorTok{$}\NormalTok{e_cols_}\DecValTok{2}
\NormalTok{f_}\DecValTok{2}\NormalTok{ <-}\StringTok{ }\NormalTok{set_f}\OperatorTok{$}\NormalTok{f_cols_}\DecValTok{2}

\NormalTok{a_}\DecValTok{2}\NormalTok{_r <-}\StringTok{ }\NormalTok{set_a_r}\OperatorTok{$}\NormalTok{a_cols_}\DecValTok{2}
\NormalTok{b_}\DecValTok{2}\NormalTok{_r <-}\StringTok{ }\NormalTok{set_b_r}\OperatorTok{$}\NormalTok{b_cols_}\DecValTok{2}
\NormalTok{c_}\DecValTok{2}\NormalTok{_r <-}\StringTok{ }\NormalTok{set_c_r}\OperatorTok{$}\NormalTok{c_cols_}\DecValTok{2}
\NormalTok{d_}\DecValTok{2}\NormalTok{_r <-}\StringTok{ }\NormalTok{set_d_r}\OperatorTok{$}\NormalTok{d_cols_}\DecValTok{2}
\NormalTok{e_}\DecValTok{2}\NormalTok{_r <-}\StringTok{ }\NormalTok{set_e_r}\OperatorTok{$}\NormalTok{e_cols_}\DecValTok{2}
\NormalTok{f_}\DecValTok{2}\NormalTok{_r <-}\StringTok{ }\NormalTok{set_f_r}\OperatorTok{$}\NormalTok{f_cols_}\DecValTok{2}

\NormalTok{a_}\DecValTok{2}\NormalTok{_py <-}\StringTok{ }\NormalTok{set_a_py}\OperatorTok{$}\NormalTok{a_cols_}\DecValTok{2}
\NormalTok{b_}\DecValTok{2}\NormalTok{_py <-}\StringTok{ }\NormalTok{set_b_py}\OperatorTok{$}\NormalTok{b_cols_}\DecValTok{2}
\NormalTok{c_}\DecValTok{2}\NormalTok{_py <-}\StringTok{ }\NormalTok{set_c_py}\OperatorTok{$}\NormalTok{c_cols_}\DecValTok{2}
\NormalTok{d_}\DecValTok{2}\NormalTok{_py <-}\StringTok{ }\NormalTok{set_d_py}\OperatorTok{$}\NormalTok{d_cols_}\DecValTok{2}
\NormalTok{e_}\DecValTok{2}\NormalTok{_py <-}\StringTok{ }\NormalTok{set_e_py}\OperatorTok{$}\NormalTok{e_cols_}\DecValTok{2}
\NormalTok{f_}\DecValTok{2}\NormalTok{_py <-}\StringTok{ }\NormalTok{set_f_py}\OperatorTok{$}\NormalTok{f_cols_}\DecValTok{2}

\ControlFlowTok{for}\NormalTok{(i }\ControlFlowTok{in} \DecValTok{1}\OperatorTok{:}\KeywordTok{length}\NormalTok{(a_}\DecValTok{2}\NormalTok{))\{}
  \ControlFlowTok{if}\NormalTok{(a_}\DecValTok{2}\NormalTok{[i] }\OperatorTok{==}\StringTok{ "Yes, A is easier"}\NormalTok{)\{}
\NormalTok{    a_}\DecValTok{2}\NormalTok{[i] <-}\StringTok{ "A"}
\NormalTok{  \} }\ControlFlowTok{else} \ControlFlowTok{if}\NormalTok{(a_}\DecValTok{2}\NormalTok{[i] }\OperatorTok{==}\StringTok{ "Yes, B is easier"}\NormalTok{)\{}
\NormalTok{  a_}\DecValTok{2}\NormalTok{[i] <-}\StringTok{ "B"}
\NormalTok{  \} }\ControlFlowTok{else}\NormalTok{ a_}\DecValTok{2}\NormalTok{[i] <-}\StringTok{ "None"}
\NormalTok{\}}

\ControlFlowTok{for}\NormalTok{(i }\ControlFlowTok{in} \DecValTok{1}\OperatorTok{:}\KeywordTok{length}\NormalTok{(b_}\DecValTok{2}\NormalTok{))\{}
  \ControlFlowTok{if}\NormalTok{(b_}\DecValTok{2}\NormalTok{[i] }\OperatorTok{==}\StringTok{ "Yes, B is easier"}\NormalTok{)\{}
\NormalTok{    b_}\DecValTok{2}\NormalTok{[i] <-}\StringTok{ "B"}
\NormalTok{  \} }\ControlFlowTok{else} \ControlFlowTok{if}\NormalTok{(b_}\DecValTok{2}\NormalTok{[i] }\OperatorTok{==}\StringTok{ "Yes, A is easier"}\NormalTok{)\{ }
\NormalTok{    b_}\DecValTok{2}\NormalTok{[i] <-}\StringTok{ "A"}
\NormalTok{    \}}\ControlFlowTok{else}\NormalTok{ b_}\DecValTok{2}\NormalTok{[i] <-}\StringTok{ "None"}
\NormalTok{\}}

\ControlFlowTok{for}\NormalTok{(i }\ControlFlowTok{in} \DecValTok{1}\OperatorTok{:}\KeywordTok{length}\NormalTok{(e_}\DecValTok{2}\NormalTok{))\{}
  \ControlFlowTok{if}\NormalTok{(e_}\DecValTok{2}\NormalTok{[i] }\OperatorTok{==}\StringTok{ "Yes, A is easier"}\NormalTok{)\{}
\NormalTok{    e_}\DecValTok{2}\NormalTok{[i] <-}\StringTok{ "A"}
\NormalTok{  \} }\ControlFlowTok{else} \ControlFlowTok{if}\NormalTok{(e_}\DecValTok{2}\NormalTok{[i] }\OperatorTok{==}\StringTok{ "Yes, B is easier"}\NormalTok{)\{}
\NormalTok{    e_}\DecValTok{2}\NormalTok{[i] <-}\StringTok{ "B"}
\NormalTok{  \} }\ControlFlowTok{else}\NormalTok{ e_}\DecValTok{2}\NormalTok{[i] <-}\StringTok{ "None"}
\NormalTok{\}}

\ControlFlowTok{for}\NormalTok{(i }\ControlFlowTok{in} \DecValTok{1}\OperatorTok{:}\KeywordTok{length}\NormalTok{(f_}\DecValTok{2}\NormalTok{))\{}
  \ControlFlowTok{if}\NormalTok{(f_}\DecValTok{2}\NormalTok{[i] }\OperatorTok{==}\StringTok{ "Yes, B is easier"}\NormalTok{)\{}
\NormalTok{    f_}\DecValTok{2}\NormalTok{[i] <-}\StringTok{ "B"}
\NormalTok{  \} }\ControlFlowTok{else} \ControlFlowTok{if}\NormalTok{(f_}\DecValTok{2}\NormalTok{[i] }\OperatorTok{==}\StringTok{ "Yes, A is easier"}\NormalTok{)\{}
\NormalTok{    f_}\DecValTok{2}\NormalTok{[i] <-}\StringTok{ "A"}
\NormalTok{  \} }\ControlFlowTok{else}\NormalTok{ f_}\DecValTok{2}\NormalTok{[i] <-}\StringTok{ "None"}
\NormalTok{\}}

\ControlFlowTok{for}\NormalTok{(i }\ControlFlowTok{in} \DecValTok{1}\OperatorTok{:}\KeywordTok{length}\NormalTok{(c_}\DecValTok{2}\NormalTok{))\{}
  \ControlFlowTok{if}\NormalTok{(c_}\DecValTok{2}\NormalTok{[i] }\OperatorTok{==}\StringTok{ "Yes, A is easier"}\NormalTok{)\{}
\NormalTok{    c_}\DecValTok{2}\NormalTok{[i] <-}\StringTok{ "A"}
\NormalTok{  \} }\ControlFlowTok{else} \ControlFlowTok{if}\NormalTok{(c_}\DecValTok{2}\NormalTok{[i] }\OperatorTok{==}\StringTok{ "Yes, B is easier"}\NormalTok{)\{ }
\NormalTok{    c_}\DecValTok{2}\NormalTok{[i] <-}\StringTok{ "B"}
\NormalTok{  \} }\ControlFlowTok{else}\NormalTok{ c_}\DecValTok{2}\NormalTok{[i] <-}\StringTok{ "None"}
\NormalTok{\}}

\ControlFlowTok{for}\NormalTok{(i }\ControlFlowTok{in} \DecValTok{1}\OperatorTok{:}\KeywordTok{length}\NormalTok{(d_}\DecValTok{2}\NormalTok{))\{}
  \ControlFlowTok{if}\NormalTok{(d_}\DecValTok{2}\NormalTok{[i] }\OperatorTok{==}\StringTok{ "Yes, A is easier"}\NormalTok{)\{}
\NormalTok{    d_}\DecValTok{2}\NormalTok{[i] <-}\StringTok{ "A"}
\NormalTok{  \} }\ControlFlowTok{else} \ControlFlowTok{if}\NormalTok{(d_}\DecValTok{2}\NormalTok{[i] }\OperatorTok{==}\StringTok{ "Yes, B is easier"}\NormalTok{)\{ }
\NormalTok{    d_}\DecValTok{2}\NormalTok{[i] <-}\StringTok{ "B"}
\NormalTok{  \} }\ControlFlowTok{else}\NormalTok{ d_}\DecValTok{2}\NormalTok{[i] <-}\StringTok{ "None"}
\NormalTok{\}}





\ControlFlowTok{for}\NormalTok{(i }\ControlFlowTok{in} \DecValTok{1}\OperatorTok{:}\KeywordTok{length}\NormalTok{(a_}\DecValTok{2}\NormalTok{_r))\{}
  \ControlFlowTok{if}\NormalTok{(a_}\DecValTok{2}\NormalTok{_r[i] }\OperatorTok{==}\StringTok{ "Yes, A is easier"}\NormalTok{)\{}
\NormalTok{    a_}\DecValTok{2}\NormalTok{_r[i] <-}\StringTok{ "A"}
\NormalTok{  \} }\ControlFlowTok{else} \ControlFlowTok{if}\NormalTok{(a_}\DecValTok{2}\NormalTok{_r[i] }\OperatorTok{==}\StringTok{ "Yes, B is easier"}\NormalTok{)\{}
\NormalTok{  a_}\DecValTok{2}\NormalTok{_r[i] <-}\StringTok{ "B"}
\NormalTok{  \} }\ControlFlowTok{else}\NormalTok{ a_}\DecValTok{2}\NormalTok{_r[i] <-}\StringTok{ "None"}
\NormalTok{\}}

\ControlFlowTok{for}\NormalTok{(i }\ControlFlowTok{in} \DecValTok{1}\OperatorTok{:}\KeywordTok{length}\NormalTok{(b_}\DecValTok{2}\NormalTok{_r))\{}
  \ControlFlowTok{if}\NormalTok{(b_}\DecValTok{2}\NormalTok{_r[i] }\OperatorTok{==}\StringTok{ "Yes, B is easier"}\NormalTok{)\{}
\NormalTok{    b_}\DecValTok{2}\NormalTok{_r[i] <-}\StringTok{ "B"}
\NormalTok{  \} }\ControlFlowTok{else} \ControlFlowTok{if}\NormalTok{(b_}\DecValTok{2}\NormalTok{_r[i] }\OperatorTok{==}\StringTok{ "Yes, A is easier"}\NormalTok{)\{ }
\NormalTok{    b_}\DecValTok{2}\NormalTok{_r[i] <-}\StringTok{ "A"}
\NormalTok{    \}}\ControlFlowTok{else}\NormalTok{ b_}\DecValTok{2}\NormalTok{_r[i] <-}\StringTok{ "None"}
\NormalTok{\}}

\ControlFlowTok{for}\NormalTok{(i }\ControlFlowTok{in} \DecValTok{1}\OperatorTok{:}\KeywordTok{length}\NormalTok{(e_}\DecValTok{2}\NormalTok{_r))\{}
  \ControlFlowTok{if}\NormalTok{(e_}\DecValTok{2}\NormalTok{_r[i] }\OperatorTok{==}\StringTok{ "Yes, A is easier"}\NormalTok{)\{}
\NormalTok{    e_}\DecValTok{2}\NormalTok{_r[i] <-}\StringTok{ "A"}
\NormalTok{  \} }\ControlFlowTok{else} \ControlFlowTok{if}\NormalTok{(e_}\DecValTok{2}\NormalTok{_r[i] }\OperatorTok{==}\StringTok{ "Yes, B is easier"}\NormalTok{)\{}
\NormalTok{    e_}\DecValTok{2}\NormalTok{_r[i] <-}\StringTok{ "B"}
\NormalTok{  \} }\ControlFlowTok{else}\NormalTok{ e_}\DecValTok{2}\NormalTok{_r[i] <-}\StringTok{ "None"}
\NormalTok{\}}

\ControlFlowTok{for}\NormalTok{(i }\ControlFlowTok{in} \DecValTok{1}\OperatorTok{:}\KeywordTok{length}\NormalTok{(f_}\DecValTok{2}\NormalTok{_r))\{}
  \ControlFlowTok{if}\NormalTok{(f_}\DecValTok{2}\NormalTok{_r[i] }\OperatorTok{==}\StringTok{ "Yes, B is easier"}\NormalTok{)\{}
\NormalTok{    f_}\DecValTok{2}\NormalTok{_r[i] <-}\StringTok{ "B"}
\NormalTok{  \} }\ControlFlowTok{else} \ControlFlowTok{if}\NormalTok{(f_}\DecValTok{2}\NormalTok{_r[i] }\OperatorTok{==}\StringTok{ "Yes, A is easier"}\NormalTok{)\{}
\NormalTok{    f_}\DecValTok{2}\NormalTok{_r[i] <-}\StringTok{ "A"}
\NormalTok{  \} }\ControlFlowTok{else}\NormalTok{ f_}\DecValTok{2}\NormalTok{_r[i] <-}\StringTok{ "None"}
\NormalTok{\}}

\ControlFlowTok{for}\NormalTok{(i }\ControlFlowTok{in} \DecValTok{1}\OperatorTok{:}\KeywordTok{length}\NormalTok{(c_}\DecValTok{2}\NormalTok{_r))\{}
  \ControlFlowTok{if}\NormalTok{(c_}\DecValTok{2}\NormalTok{_r[i] }\OperatorTok{==}\StringTok{ "Yes, A is easier"}\NormalTok{)\{}
\NormalTok{    c_}\DecValTok{2}\NormalTok{_r[i] <-}\StringTok{ "A"}
\NormalTok{  \} }\ControlFlowTok{else} \ControlFlowTok{if}\NormalTok{(c_}\DecValTok{2}\NormalTok{_r[i] }\OperatorTok{==}\StringTok{ "Yes, B is easier"}\NormalTok{)\{ }
\NormalTok{    c_}\DecValTok{2}\NormalTok{_r[i] <-}\StringTok{ "B"}
\NormalTok{  \} }\ControlFlowTok{else}\NormalTok{ c_}\DecValTok{2}\NormalTok{_r[i] <-}\StringTok{ "None"}
\NormalTok{\}}

\ControlFlowTok{for}\NormalTok{(i }\ControlFlowTok{in} \DecValTok{1}\OperatorTok{:}\KeywordTok{length}\NormalTok{(d_}\DecValTok{2}\NormalTok{_r))\{}
  \ControlFlowTok{if}\NormalTok{(d_}\DecValTok{2}\NormalTok{_r[i] }\OperatorTok{==}\StringTok{ "Yes, A is easier"}\NormalTok{)\{}
\NormalTok{    d_}\DecValTok{2}\NormalTok{_r[i] <-}\StringTok{ "A"}
\NormalTok{  \} }\ControlFlowTok{else} \ControlFlowTok{if}\NormalTok{(d_}\DecValTok{2}\NormalTok{_r[i] }\OperatorTok{==}\StringTok{ "Yes, B is easier"}\NormalTok{)\{ }
\NormalTok{    d_}\DecValTok{2}\NormalTok{_r[i] <-}\StringTok{ "B"}
\NormalTok{  \} }\ControlFlowTok{else}\NormalTok{ d_}\DecValTok{2}\NormalTok{_r[i] <-}\StringTok{ "None"}
\NormalTok{\}}


\ControlFlowTok{for}\NormalTok{(i }\ControlFlowTok{in} \DecValTok{1}\OperatorTok{:}\KeywordTok{length}\NormalTok{(a_}\DecValTok{2}\NormalTok{_py))\{}
  \ControlFlowTok{if}\NormalTok{(a_}\DecValTok{2}\NormalTok{_py[i] }\OperatorTok{==}\StringTok{ "Yes, A is easier"}\NormalTok{)\{}
\NormalTok{    a_}\DecValTok{2}\NormalTok{_py[i] <-}\StringTok{ "A"}
\NormalTok{  \} }\ControlFlowTok{else} \ControlFlowTok{if}\NormalTok{(a_}\DecValTok{2}\NormalTok{_py[i] }\OperatorTok{==}\StringTok{ "Yes, B is easier"}\NormalTok{)\{}
\NormalTok{  a_}\DecValTok{2}\NormalTok{_py[i] <-}\StringTok{ "B"}
\NormalTok{  \} }\ControlFlowTok{else}\NormalTok{ a_}\DecValTok{2}\NormalTok{_py[i] <-}\StringTok{ "None"}
\NormalTok{\}}

\ControlFlowTok{for}\NormalTok{(i }\ControlFlowTok{in} \DecValTok{1}\OperatorTok{:}\KeywordTok{length}\NormalTok{(b_}\DecValTok{2}\NormalTok{_py))\{}
  \ControlFlowTok{if}\NormalTok{(b_}\DecValTok{2}\NormalTok{_py[i] }\OperatorTok{==}\StringTok{ "Yes, B is easier"}\NormalTok{)\{}
\NormalTok{    b_}\DecValTok{2}\NormalTok{_py[i] <-}\StringTok{ "B"}
\NormalTok{  \} }\ControlFlowTok{else} \ControlFlowTok{if}\NormalTok{(b_}\DecValTok{2}\NormalTok{_py[i] }\OperatorTok{==}\StringTok{ "Yes, A is easier"}\NormalTok{)\{ }
\NormalTok{    b_}\DecValTok{2}\NormalTok{_py[i] <-}\StringTok{ "A"}
\NormalTok{    \}}\ControlFlowTok{else}\NormalTok{ b_}\DecValTok{2}\NormalTok{_py[i] <-}\StringTok{ "None"}
\NormalTok{\}}

\ControlFlowTok{for}\NormalTok{(i }\ControlFlowTok{in} \DecValTok{1}\OperatorTok{:}\KeywordTok{length}\NormalTok{(e_}\DecValTok{2}\NormalTok{_py))\{}
  \ControlFlowTok{if}\NormalTok{(e_}\DecValTok{2}\NormalTok{_py[i] }\OperatorTok{==}\StringTok{ "Yes, A is easier"}\NormalTok{)\{}
\NormalTok{    e_}\DecValTok{2}\NormalTok{_py[i] <-}\StringTok{ "A"}
\NormalTok{  \} }\ControlFlowTok{else} \ControlFlowTok{if}\NormalTok{(e_}\DecValTok{2}\NormalTok{_py[i] }\OperatorTok{==}\StringTok{ "Yes, B is easier"}\NormalTok{)\{}
\NormalTok{    e_}\DecValTok{2}\NormalTok{_py[i] <-}\StringTok{ "B"}
\NormalTok{  \} }\ControlFlowTok{else}\NormalTok{ e_}\DecValTok{2}\NormalTok{_py[i] <-}\StringTok{ "None"}
\NormalTok{\}}

\ControlFlowTok{for}\NormalTok{(i }\ControlFlowTok{in} \DecValTok{1}\OperatorTok{:}\KeywordTok{length}\NormalTok{(f_}\DecValTok{2}\NormalTok{_py))\{}
  \ControlFlowTok{if}\NormalTok{(f_}\DecValTok{2}\NormalTok{_py[i] }\OperatorTok{==}\StringTok{ "Yes, B is easier"}\NormalTok{)\{}
\NormalTok{    f_}\DecValTok{2}\NormalTok{_py[i] <-}\StringTok{ "B"}
\NormalTok{  \} }\ControlFlowTok{else} \ControlFlowTok{if}\NormalTok{(f_}\DecValTok{2}\NormalTok{_py[i] }\OperatorTok{==}\StringTok{ "Yes, A is easier"}\NormalTok{)\{}
\NormalTok{    f_}\DecValTok{2}\NormalTok{_py[i] <-}\StringTok{ "A"}
\NormalTok{  \} }\ControlFlowTok{else}\NormalTok{ f_}\DecValTok{2}\NormalTok{_py[i] <-}\StringTok{ "None"}
\NormalTok{\}}

\ControlFlowTok{for}\NormalTok{(i }\ControlFlowTok{in} \DecValTok{1}\OperatorTok{:}\KeywordTok{length}\NormalTok{(c_}\DecValTok{2}\NormalTok{_py))\{}
  \ControlFlowTok{if}\NormalTok{(c_}\DecValTok{2}\NormalTok{_py[i] }\OperatorTok{==}\StringTok{ "Yes, A is easier"}\NormalTok{)\{}
\NormalTok{    c_}\DecValTok{2}\NormalTok{_py[i] <-}\StringTok{ "A"}
\NormalTok{  \} }\ControlFlowTok{else} \ControlFlowTok{if}\NormalTok{(c_}\DecValTok{2}\NormalTok{_py[i] }\OperatorTok{==}\StringTok{ "Yes, B is easier"}\NormalTok{)\{ }
\NormalTok{    c_}\DecValTok{2}\NormalTok{_py[i] <-}\StringTok{ "B"}
\NormalTok{  \} }\ControlFlowTok{else}\NormalTok{ c_}\DecValTok{2}\NormalTok{_py[i] <-}\StringTok{ "None"}
\NormalTok{\}}

\ControlFlowTok{for}\NormalTok{(i }\ControlFlowTok{in} \DecValTok{1}\OperatorTok{:}\KeywordTok{length}\NormalTok{(d_}\DecValTok{2}\NormalTok{_py))\{}
  \ControlFlowTok{if}\NormalTok{(d_}\DecValTok{2}\NormalTok{_py[i] }\OperatorTok{==}\StringTok{ "Yes, A is easier"}\NormalTok{)\{}
\NormalTok{    d_}\DecValTok{2}\NormalTok{_py[i] <-}\StringTok{ "A"}
\NormalTok{  \} }\ControlFlowTok{else} \ControlFlowTok{if}\NormalTok{(d_}\DecValTok{2}\NormalTok{_py[i] }\OperatorTok{==}\StringTok{ "Yes, B is easier"}\NormalTok{)\{ }
\NormalTok{    d_}\DecValTok{2}\NormalTok{_py[i] <-}\StringTok{ "B"}
\NormalTok{  \} }\ControlFlowTok{else}\NormalTok{ d_}\DecValTok{2}\NormalTok{_py[i] <-}\StringTok{ "None"}
\NormalTok{\}}


\NormalTok{tab <-}\StringTok{ }\KeywordTok{rbind}\NormalTok{(}\KeywordTok{table}\NormalTok{(a_}\DecValTok{2}\NormalTok{), }\KeywordTok{c}\NormalTok{(}\KeywordTok{table}\NormalTok{(b_}\DecValTok{2}\NormalTok{)[}\DecValTok{1}\NormalTok{], }\DecValTok{0}\NormalTok{,}\KeywordTok{table}\NormalTok{(b_}\DecValTok{2}\NormalTok{)[}\DecValTok{1}\NormalTok{]), }\KeywordTok{c}\NormalTok{(}\KeywordTok{table}\NormalTok{(c_}\DecValTok{2}\NormalTok{), }\DecValTok{0}\NormalTok{), }\KeywordTok{c}\NormalTok{(}\KeywordTok{table}\NormalTok{(d_}\DecValTok{2}\NormalTok{), }\DecValTok{0}\NormalTok{), }\KeywordTok{c}\NormalTok{(}\KeywordTok{table}\NormalTok{(e_}\DecValTok{2}\NormalTok{), }\DecValTok{0}\NormalTok{), }\KeywordTok{table}\NormalTok{(f_}\DecValTok{2}\NormalTok{))}
\end{Highlighting}
\end{Shaded}

\begin{verbatim}
## Warning in rbind(table(a_2), c(table(b_2)[1], 0, table(b_2)[1]), c(table(c_2), :
## number of columns of result is not a multiple of vector length (arg 5)
\end{verbatim}

\begin{Shaded}
\begin{Highlighting}[]
\KeywordTok{rownames}\NormalTok{(tab) <-}\StringTok{ }\KeywordTok{c}\NormalTok{(}\StringTok{"Set A"}\NormalTok{, }\StringTok{"Set B"}\NormalTok{, }\StringTok{"Set C"}\NormalTok{, }\StringTok{"Set D"}\NormalTok{, }\StringTok{"Set E"}\NormalTok{, }\StringTok{"Set F"}\NormalTok{)}
\KeywordTok{colnames}\NormalTok{(tab) <-}\StringTok{ }\KeywordTok{c}\NormalTok{(}\StringTok{"A"}\NormalTok{, }\StringTok{"B"}\NormalTok{, }\StringTok{"None"}\NormalTok{)}
\KeywordTok{kable}\NormalTok{(tab, }\DataTypeTok{caption =} \StringTok{"Aesthetic preference of colour schemes"}\NormalTok{)}\OperatorTok
\StringTok{  }\KeywordTok{kable_styling}\NormalTok{(}\DataTypeTok{latex_options =} \StringTok{"hold_position"}\NormalTok{)}
\end{Highlighting}
\end{Shaded}

\begin{table}[!h]

\caption{\label{tab:unnamed-chunk-52}Aesthetic preference of colour schemes}
\centering
\begin{tabular}[t]{l|r|r|r}
\hline
  & A & B & None\\
\hline
Set A & 7 & 3 & 3\\
\hline
Set B & 11 & 0 & 11\\
\hline
Set C & 9 & 1 & 0\\
\hline
Set D & 2 & 10 & 0\\
\hline
Set E & 11 & 0 & 11\\
\hline
Set F & 2 & 9 & 1\\
\hline
\end{tabular}
\end{table}

\begin{Shaded}
\begin{Highlighting}[]
\NormalTok{tab <-}\StringTok{ }\KeywordTok{rbind}\NormalTok{(}\KeywordTok{table}\NormalTok{(a_}\DecValTok{2}\NormalTok{_r), }\KeywordTok{c}\NormalTok{(}\KeywordTok{table}\NormalTok{(b_}\DecValTok{2}\NormalTok{_r)[}\DecValTok{1}\NormalTok{], }\DecValTok{0}\NormalTok{, }\KeywordTok{table}\NormalTok{(b_}\DecValTok{2}\NormalTok{_r)[}\DecValTok{2}\NormalTok{]), }\KeywordTok{c}\NormalTok{(}\KeywordTok{table}\NormalTok{(c_}\DecValTok{2}\NormalTok{_r), }\DecValTok{0}\NormalTok{), }\KeywordTok{c}\NormalTok{(}\KeywordTok{table}\NormalTok{(d_}\DecValTok{2}\NormalTok{_r), }\DecValTok{0}\NormalTok{), }\KeywordTok{c}\NormalTok{(}\KeywordTok{table}\NormalTok{(e_}\DecValTok{2}\NormalTok{_r), }\DecValTok{0}\NormalTok{, }\DecValTok{0}\NormalTok{), }\KeywordTok{table}\NormalTok{(f_}\DecValTok{2}\NormalTok{_r))}
\end{Highlighting}
\end{Shaded}

\begin{verbatim}
## Warning in rbind(table(a_2_r), c(table(b_2_r)[1], 0, table(b_2_r)[2]),
## c(table(c_2_r), : number of columns of result is not a multiple of vector length
## (arg 1)
\end{verbatim}

\begin{Shaded}
\begin{Highlighting}[]
\KeywordTok{rownames}\NormalTok{(tab) <-}\StringTok{ }\KeywordTok{c}\NormalTok{(}\StringTok{"Set A"}\NormalTok{, }\StringTok{"Set B"}\NormalTok{, }\StringTok{"Set C"}\NormalTok{, }\StringTok{"Set D"}\NormalTok{, }\StringTok{"Set E"}\NormalTok{, }\StringTok{"Set F"}\NormalTok{)}
\KeywordTok{colnames}\NormalTok{(tab) <-}\StringTok{ }\KeywordTok{c}\NormalTok{(}\StringTok{"A"}\NormalTok{, }\StringTok{"B"}\NormalTok{, }\StringTok{"None"}\NormalTok{)}
\KeywordTok{kable}\NormalTok{(tab, }\DataTypeTok{caption =} \StringTok{"Aesthetic preference of colour schemes, for R"}\NormalTok{)}\OperatorTok
\StringTok{  }\KeywordTok{kable_styling}\NormalTok{(}\DataTypeTok{latex_options =} \StringTok{"hold_position"}\NormalTok{)}
\end{Highlighting}
\end{Shaded}

\begin{table}[!h]

\caption{\label{tab:unnamed-chunk-52}Aesthetic preference of colour schemes, for R}
\centering
\begin{tabular}[t]{l|r|r|r}
\hline
  & A & B & None\\
\hline
Set A & 5 & 3 & 5\\
\hline
Set B & 5 & 0 & 1\\
\hline
Set C & 4 & 1 & 0\\
\hline
Set D & 1 & 6 & 0\\
\hline
Set E & 5 & 0 & 0\\
\hline
Set F & 2 & 4 & 1\\
\hline
\end{tabular}
\end{table}

\begin{Shaded}
\begin{Highlighting}[]
\NormalTok{tab <-}\StringTok{ }\KeywordTok{rbind}\NormalTok{(}\KeywordTok{table}\NormalTok{(a_}\DecValTok{2}\NormalTok{_py), }\KeywordTok{c}\NormalTok{(}\KeywordTok{table}\NormalTok{(b_}\DecValTok{2}\NormalTok{_py)[}\DecValTok{1}\NormalTok{], }\DecValTok{0}\NormalTok{, }\DecValTok{0}\NormalTok{), }\KeywordTok{c}\NormalTok{(}\KeywordTok{table}\NormalTok{(c_}\DecValTok{2}\NormalTok{_py), }\DecValTok{0}\NormalTok{, }\DecValTok{0}\NormalTok{), }\KeywordTok{c}\NormalTok{(}\KeywordTok{table}\NormalTok{(d_}\DecValTok{2}\NormalTok{_py), }\DecValTok{0}\NormalTok{), }\KeywordTok{c}\NormalTok{(}\KeywordTok{table}\NormalTok{(e_}\DecValTok{2}\NormalTok{_py), }\DecValTok{0}\NormalTok{, }\DecValTok{0}\NormalTok{), }\KeywordTok{c}\NormalTok{(}\DecValTok{0}\NormalTok{, }\KeywordTok{table}\NormalTok{(f_}\DecValTok{2}\NormalTok{_py), }\DecValTok{0}\NormalTok{))}
\end{Highlighting}
\end{Shaded}

\begin{verbatim}
## Warning in rbind(table(a_2_py), c(table(b_2_py)[1], 0, 0), c(table(c_2_py), :
## number of columns of result is not a multiple of vector length (arg 1)
\end{verbatim}

\begin{Shaded}
\begin{Highlighting}[]
\KeywordTok{rownames}\NormalTok{(tab) <-}\StringTok{ }\KeywordTok{c}\NormalTok{(}\StringTok{"Set A"}\NormalTok{, }\StringTok{"Set B"}\NormalTok{, }\StringTok{"Set C"}\NormalTok{, }\StringTok{"Set D"}\NormalTok{, }\StringTok{"Set E"}\NormalTok{, }\StringTok{"Set F"}\NormalTok{)}
\KeywordTok{colnames}\NormalTok{(tab) <-}\StringTok{ }\KeywordTok{c}\NormalTok{(}\StringTok{"A"}\NormalTok{, }\StringTok{"B"}\NormalTok{, }\StringTok{"None"}\NormalTok{)}
\KeywordTok{kable}\NormalTok{(tab, }\DataTypeTok{caption =} \StringTok{"Aesthetic preference of colour schemes, for Python"}\NormalTok{)}\OperatorTok
\StringTok{  }\KeywordTok{kable_styling}\NormalTok{(}\DataTypeTok{latex_options =} \StringTok{"hold_position"}\NormalTok{)}
\end{Highlighting}
\end{Shaded}

\begin{table}[!h]

\caption{\label{tab:unnamed-chunk-52}Aesthetic preference of colour schemes, for Python}
\centering
\begin{tabular}[t]{l|r|r|r}
\hline
  & A & B & None\\
\hline
Set A & 2 & 3 & 2\\
\hline
Set B & 6 & 0 & 0\\
\hline
Set C & 5 & 0 & 0\\
\hline
Set D & 1 & 4 & 0\\
\hline
Set E & 6 & 0 & 0\\
\hline
Set F & 0 & 5 & 0\\
\hline
\end{tabular}
\end{table}

Interestingly here, the top two rows appear to give slightly opposing
results; the respondents who were presented with viridis for the main
questions and the default as a secondary palette stated that they found
either viridis easier to interpret or had no preference, whereas those
presented with the default first and viridis second tended to find the
default easier. This could perhaps be a result of the respondents
becoming used to their primary colour scheme.

Once again looking at the comparisons with the greyscale, there were
some respondents that found this easier to read, but the majority chose
the alternative, whether this is viridis or the default.

The results seem fairly similar for the R and Python responses, showing
that the default colourings for each language elicit a similar level of
ease of interpretation.

The sample of respondents with colour blindness was too small to test
this analysis.

\section{Sales - Part 1}

Now consider the sales part of the survey. In this section data was
taken from a the \texttt{BJsales} data set in R, which is a time series
data set containing 150 observations. This data set constitutes a single
vector of values with no specified timings, and the visualisation data
was formed by taking subsets of size 12 this and setting a month between
each point to give a year of fictional sales data.

\subsection{How much would you say sales of each company increased between January and December? [Company A]}

This question was included for the purpose of testing whether, again,
axis scaling impacts the perceived differences between values, but this
time with time series line plots as opposed to bar plots. Respondents
were asked to assess how much the sales of company A increased over the
course of the year, or in other words to look at and compare each end of
the line.

The plot for which the respondents, on average, found the difference to
be smallest was the zeroed, followed by the truncated, and then the
separated, with means of 1.371, 2.414 and 3.043 respectively. These
differences are found to be statistically significant, as outlined in
table{[}?{]}.

\begin{Shaded}
\begin{Highlighting}[]
\NormalTok{hyp <-}\StringTok{ }\KeywordTok{c}\NormalTok{(}\StringTok{"Truncated > Zeroed"}\NormalTok{, }\StringTok{"Truncated < Separated"}\NormalTok{, }\StringTok{"Separated > Zeroed"}\NormalTok{)}
\NormalTok{pval <-}\StringTok{ }\KeywordTok{c}\NormalTok{(}\StringTok{"8.870681966755e-14"}\NormalTok{, }\FloatTok{0.00654175643803223}\NormalTok{, }\FloatTok{3.48079934270661e-13}\NormalTok{)}
\NormalTok{tab <-}\StringTok{ }\KeywordTok{data.frame}\NormalTok{(hyp, pval)}
\KeywordTok{colnames}\NormalTok{(tab) <-}\StringTok{ }\KeywordTok{c}\NormalTok{(}\StringTok{"Alternatve Hypothesis"}\NormalTok{, }\StringTok{"P-value"}\NormalTok{)}
\KeywordTok{kable}\NormalTok{(tab, }\DataTypeTok{caption=}\StringTok{"Table of p-values for this question"}\NormalTok{) }\OperatorTok\StringTok{ }
\StringTok{  }\KeywordTok{kable_styling}\NormalTok{(}\DataTypeTok{latex_options =} \StringTok{'hold_position'}\NormalTok{)}
\end{Highlighting}
\end{Shaded}

\begin{table}[!h]

\caption{\label{tab:unnamed-chunk-53}Table of p-values for this question}
\centering
\begin{tabular}[t]{l|l}
\hline
Alternatve Hypothesis & P-value\\
\hline
Truncated > Zeroed & 8.870681966755e-14\\
\hline
Truncated < Separated & 0.00654175643803223\\
\hline
Separated > Zeroed & 3.48079934270661e-13\\
\hline
\end{tabular}
\end{table}

The differences between languages and plot ordering were shown to be
inconsequential (see table {[}?{]})

\subsection{How much would you say sales of each company increased between January and December? [Company B]}

The zeroed was once again perceived to have the smallest difference
(\(\bar{x} = 1.371\)), but this time with the separated in the middle
(\(\bar{x} = 4.1304\)) and truncated with the largest difference
(\(\bar{x} = 4.1304\)). See appendix 2, table {[}?{]} for p-values. The
p-values show sufficient evidence that the truncated responses were on
average greater than the zeroed, as were the responses for the separated
plots. However, the difference between the ratings for the truncated and
separated plot responses was inconsequential, along with the language
comparisons and plot order.

\begin{Shaded}
\begin{Highlighting}[]
\NormalTok{hyp <-}\StringTok{ }\KeywordTok{c}\NormalTok{(}\StringTok{"Truncated > Zeroed"}\NormalTok{, }\StringTok{"Truncated not equal to Separated"}\NormalTok{, }\StringTok{"Separated not equal to Zeroed"}\NormalTok{)}
\NormalTok{pval <-}\StringTok{ }\KeywordTok{c}\NormalTok{(}\StringTok{"8.95254768631571e-23"}\NormalTok{, }\StringTok{"0.2162"}\NormalTok{, }\StringTok{"12.46327564235365e-23"}\NormalTok{)}
\NormalTok{tab <-}\StringTok{ }\KeywordTok{data.frame}\NormalTok{(hyp, pval)}
\KeywordTok{colnames}\NormalTok{(tab) <-}\StringTok{ }\KeywordTok{c}\NormalTok{(}\StringTok{"Hypothesis"}\NormalTok{, }\StringTok{"P-value"}\NormalTok{)}
\KeywordTok{kable}\NormalTok{(tab, }\DataTypeTok{caption=}\StringTok{"Table of p-values for this question"}\NormalTok{) }\OperatorTok\StringTok{ }
\StringTok{  }\KeywordTok{kable_styling}\NormalTok{(}\DataTypeTok{latex_options =} \StringTok{"hold_position"}\NormalTok{)}
\end{Highlighting}
\end{Shaded}

\begin{table}[!h]

\caption{\label{tab:unnamed-chunk-54}Table of p-values for this question}
\centering
\begin{tabular}[t]{l|l}
\hline
Hypothesis & P-value\\
\hline
Truncated > Zeroed & 8.95254768631571e-23\\
\hline
Truncated not equal to Separated & 0.2162\\
\hline
Separated not equal to Zeroed & 12.46327564235365e-23\\
\hline
\end{tabular}
\end{table}

\subsection{How large would you say the drop in sales between April and July of Company A  is?}

The means for this question appear very significantly different by eye,
once again with the zeroed plot eliciting the lowest average rating
(\(\bar{x} = 1.371429\)), followed by the truncated
(\(\bar{x} = 2.814286\)) and then the separated
(\(\bar{x} = 4.028571\)). The p-values confirm the significance of the
differences between all three variables.

\begin{Shaded}
\begin{Highlighting}[]
\NormalTok{hyp <-}\StringTok{ }\KeywordTok{c}\NormalTok{(}\StringTok{"Truncated not equal to Zeroed"}\NormalTok{, }\StringTok{"Truncated not equal to Separated"}\NormalTok{, }\StringTok{"Separated not equal to Zeroed"}\NormalTok{)}
\NormalTok{pval <-}\StringTok{ }\KeywordTok{c}\NormalTok{(}\StringTok{"1.03832498155043e-11"}\NormalTok{, }\StringTok{"0.00012743463393642"}\NormalTok{, }\StringTok{"1.1261341031207e-16"}\NormalTok{)}
\NormalTok{tab <-}\StringTok{ }\KeywordTok{data.frame}\NormalTok{(hyp, pval)}
\KeywordTok{colnames}\NormalTok{(tab) <-}\StringTok{ }\KeywordTok{c}\NormalTok{(}\StringTok{"Hypothesis"}\NormalTok{, }\StringTok{"P-value"}\NormalTok{)}
\KeywordTok{kable}\NormalTok{(tab, }\DataTypeTok{caption=}\StringTok{"Table of p-values for this question"}\NormalTok{) }\OperatorTok\StringTok{ }
\StringTok{  }\KeywordTok{kable_styling}\NormalTok{(}\DataTypeTok{latex_options =} \StringTok{"hold_position"}\NormalTok{)}
\end{Highlighting}
\end{Shaded}

\begin{table}[!h]

\caption{\label{tab:unnamed-chunk-55}Table of p-values for this question}
\centering
\begin{tabular}[t]{l|l}
\hline
Hypothesis & P-value\\
\hline
Truncated not equal to Zeroed & 1.03832498155043e-11\\
\hline
Truncated not equal to Separated & 0.00012743463393642\\
\hline
Separated not equal to Zeroed & 1.1261341031207e-16\\
\hline
\end{tabular}
\end{table}

\begin{Shaded}
\begin{Highlighting}[]
\NormalTok{sep_}\DecValTok{2}\NormalTok{ <-}\StringTok{ }\KeywordTok{na.exclude}\NormalTok{(ab_sep}\OperatorTok{$}\NormalTok{sep_}\DecValTok{2}\NormalTok{)}
\NormalTok{trn_}\DecValTok{2}\NormalTok{ <-}\StringTok{ }\KeywordTok{na.exclude}\NormalTok{(ab_trn}\OperatorTok{$}\NormalTok{ab_trn_}\DecValTok{2}\NormalTok{)}
\NormalTok{zro_}\DecValTok{2}\NormalTok{ <-}\StringTok{ }\KeywordTok{na.exclude}\NormalTok{(ab_zero}\OperatorTok{$}\NormalTok{ab_zro_}\DecValTok{2}\NormalTok{)}
\end{Highlighting}
\end{Shaded}

\section{Sales - Part 2}

\subsection{Based on the above graph, how large would you say the difference is between the number of sales Company C makes and the number of sales Company D makes?}

The final question of the survey compares just two plots, for which the
difference in the ratings is shown to be significant, with the mean for
the truncated plot ratings at \(\bar{x} = 4.271\) and for the zeroed
\(\bar{x} = 2.7\) and a one-sided p-value of \(p=4.44089209850063e-15\)
showing the difference in the truncated was on average rated as larger
than for the zeroed.

\begin{Shaded}
\begin{Highlighting}[]
\NormalTok{trn_cd <-}\StringTok{ }\KeywordTok{na.exclude}\NormalTok{(cd_trn}\OperatorTok{$}\NormalTok{cd_trn)}
\NormalTok{zro_cd <-}\StringTok{ }\KeywordTok{na.exclude}\NormalTok{(cd_zro}\OperatorTok{$}\NormalTok{cd_zro)}
\end{Highlighting}
\end{Shaded}

\section{Conclusion}

From this analysis, it can be concluded that altering axis scales in the
way of truncating the axis or converting to a logarithmic scaling may
have an effect on interpretation of differences in values, for both bar
and line plots. The axis truncation has the effect of increasing the
perceived difference in value and the logarithmic does the opposite. In
general, from both literature, it is advised against to truncate the
axis of a bar plot and this study confirms that it does in fact have an
effect on interpretation. A logarithmic scaling may be ill-advised where
it will distort the perceived size of the difference in point value,
such as for the bar chart used here, however as discussed before from
literature could be useful for other purposes, such as data that differs
greatly in orders of magnitude. The labeling of this may also need to be
considered, since the standard form labeling here confused some
respondents.

Altering the aspect ratios had less of an effect, however there was a
marginal effect of the wide plot making the difference in bar height
appear smaller, and vice versa for the narrow.

\end{document}
