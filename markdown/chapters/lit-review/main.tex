% Options for packages loaded elsewhere
\PassOptionsToPackage{unicode}{hyperref}
\PassOptionsToPackage{hyphens}{url}
%
\documentclass[
  12pt,
]{book}
\usepackage{lmodern}
\usepackage{amssymb,amsmath}
\usepackage{ifxetex,ifluatex}
\ifnum 0\ifxetex 1\fi\ifluatex 1\fi=0 % if pdftex
  \usepackage[T1]{fontenc}
  \usepackage[utf8]{inputenc}
  \usepackage{textcomp} % provide euro and other symbols
\else % if luatex or xetex
  \usepackage{unicode-math}
  \defaultfontfeatures{Scale=MatchLowercase}
  \defaultfontfeatures[\rmfamily]{Ligatures=TeX,Scale=1}
\fi
% Use upquote if available, for straight quotes in verbatim environments
\IfFileExists{upquote.sty}{\usepackage{upquote}}{}
\IfFileExists{microtype.sty}{% use microtype if available
  \usepackage[]{microtype}
  \UseMicrotypeSet[protrusion]{basicmath} % disable protrusion for tt fonts
}{}
\makeatletter
\@ifundefined{KOMAClassName}{% if non-KOMA class
  \IfFileExists{parskip.sty}{%
    \usepackage{parskip}
  }{% else
    \setlength{\parindent}{0pt}
    \setlength{\parskip}{6pt plus 2pt minus 1pt}}
}{% if KOMA class
  \KOMAoptions{parskip=half}}
\makeatother
\usepackage{xcolor}
\IfFileExists{xurl.sty}{\usepackage{xurl}}{} % add URL line breaks if available
\IfFileExists{bookmark.sty}{\usepackage{bookmark}}{\usepackage{hyperref}}
\hypersetup{
  hidelinks,
  pdfcreator={LaTeX via pandoc}}
\urlstyle{same} % disable monospaced font for URLs
\usepackage[margin=1in]{geometry}
\usepackage{graphicx,grffile}
\makeatletter
\def\maxwidth{\ifdim\Gin@nat@width>\linewidth\linewidth\else\Gin@nat@width\fi}
\def\maxheight{\ifdim\Gin@nat@height>\textheight\textheight\else\Gin@nat@height\fi}
\makeatother
% Scale images if necessary, so that they will not overflow the page
% margins by default, and it is still possible to overwrite the defaults
% using explicit options in \includegraphics[width, height, ...]{}
\setkeys{Gin}{width=\maxwidth,height=\maxheight,keepaspectratio}
% Set default figure placement to htbp
\makeatletter
\def\fps@figure{htbp}
\makeatother
\setlength{\emergencystretch}{3em} % prevent overfull lines
\providecommand{\tightlist}{%
  \setlength{\itemsep}{0pt}\setlength{\parskip}{0pt}}
\setcounter{secnumdepth}{-\maxdimen} % remove section numbering
\setlength{\parskip}{1em}
\usepackage{float}
\usepackage{booktabs}
\usepackage{longtable}
\usepackage{array}
\usepackage{multirow}
\usepackage{wrapfig}
\usepackage{colortbl}
\usepackage{pdflscape}
\usepackage{tabu}
\usepackage{threeparttable}
\usepackage{threeparttablex}
\usepackage[normalem]{ulem}
\usepackage{makecell}
\usepackage{xcolor}
\usepackage[]{natbib}
\bibliographystyle{plainnat}

\author{}
\date{\vspace{-2.5em}}

\begin{document}
\frontmatter

\mainmatter
\section{Introduction}

Data visualisation is a method of conveying data in an easily digestible
manner through graphics. It is an important aspect of data presentation
and allows key information to be quickly identified by the observer.
Very many subject areas rely on such visualisations to relay messages
that may get lost or have less impact when presented as written word or
raw numbers.

The main objective of visualisation is to create figures that display
the data accurately in an aesthetic manner, giving non-misleading
messages in a format that is pleasing to the eye. A good visualisation
strikes a balance between aesthetics and information, where the
aesthetic features are designed such that they `enhance the message of
the visualisation' \citep{wilke2019}.

An incorrect balance of aesthetics to information can lead to figures
that are misleading, confusing, or unengaging. Wilke discusses the way
in which, for example, a research scientist with limited design
experience may produce a visualisation displaying the data in an
informative manner, but fail to draw immediate attention to the desired
message, and on the other hand, someone with a main interest in the
aesthetic design of a visualisation could create a figure that is very
pleasing to the eye, but create a misleading visualisation in the
process.

This literature review will discuss a range of publications discussing
various aspects of data visualisation with a focus on how poor or
uninformed visualisation design can produce misleading figures, as well
as how such visualisations may be abused to deliberately deceive the
observer. Starting with publications discussing general good
visualisation practice, the discussion will then lead on to look at
studies investigating the implementation of different visualisation
practices, from which inspiration will be drawn to design the study for
this paper.

\section{Good Visualisation Practice}

The book \textit{'Fundamentals of Data Visualization'} \citep{wilke2019}
is renowned as
\textit{'an excellent reference about producing and understanding static figures, figures'}
\citep[see][]{wilkerev} and described as being
\textit{'suitable to be used as a reference manual'}
\citep[see][]{hwang2020}. Thus, this book provides a good basis to
understanding the principles behind data visualisation, and how to
create effective, informative and aesthetic figures.

In the book, Wilke discusses a variety of topics under the data
visualisation umbrella, from relatively simple but important and often
overlooked ideas such as deciding on coordinate systems, axis scales and
colouring, to how to visualise distributions, trends and geospatial
data. This literature review will focus on the areas being investigated
in the `Empirical Study of Data Visualisation' survey; namely coordinate
systems, axis scaling, colouring for bar charts, alongside stacked and
grouped bar charts, as well as axis scaling and formats for time series
plots.

In discussing coordinate systems and axis scaling, Wilke highlights
that, prior to deciding on a coordinate system, it is important to
consider the form the data will take, and where the data will be
positioned, as well as how many dimensions this data takes. The example
used is a classic two-dimensional scatter plot, in which each data value
is represented by a point positioned in a distinct location on the 2d
plane, and thus two scales are required to define where this location
falls, traditionally with a linear scale and horizontal x-axis with the
y-axis perpendicular to this.

Alternative coordinate systems can include the perpendicular model with
non-linear axes, or circular or `curved' models such as polar
coordinates in addition to flipped axes, where the dependent variable is
represented by the x-axis as opposed to the y.

`An Empirical Study of Data Visualisation' will be mainly analysing
perception of categorical bar charts, for which the `locations' are the
category as defined by the x-axis and the bar height. It will be
discussed how the perception of these locations could be altered to
stray from `good practice', and how these alterations may mislead the
observer.

In discussion of the linear, two-dimensional cartesian system, the
author describes the various formulations that this system can take, in
terms of variables with the same or different units. For example, if two
variables with different units are represented perpendicular to one
another on a cartesian system, the designer has the freedom to stretch
or compress the data in a way to best represent the data and, as Wilke
states, `maintain a valid visualisation of the data'.

Another point of interest mentioned by Wilke here is that the ratio of x
to y- axis should be such that `important differences in position are
noticeable'. This is regarded as good practice by Wilke, but could
potentially be exploited as discussed by \citet{Few2016}; the aspect
ratio can be manipulated to make differences appear larger depending on
the story that the creator wishes to sell. On the other hand, Wilke does
state that it is `important differences' that should be noticeable, and
so may relate to differences that are already significant and crucial to
see, and which may be minimised by an inappropriate aspect ratio.

For example, consider a company facing a drop in profits from one time
step to another. An aspect ratio minimising the height in comparison to
the width can allow this difference to appear smaller. On the other side
of this coin, a company may have marginal profit gain between two time
steps, and can abuse principles of perception to lengthen the y-axis as
compared to the x, potentially making the difference seem larger.

This will be considered when writing the survey, as the perceived
differences in position will be tested when changing features such as
y-axis scaling or aspect ratio. A standard practice as laid out by Wilke
is that, for two variables with the same unit, the aspect ratio should
ensure that the space between ticks for each variable are equal in size.

\hypertarget{log-scaling}{%
\paragraph{LOG SCALING}\label{log-scaling}}

After this, Wilke goes on to discuss logarithmic scaling, which will be
investigated in this study alongside axis truncation. Conversely to what
will be investigated in this study, he talks about both logarithmic
scaling and log-transformed data whereas this study will consider
logarithmic scaling alone. \#\#\#\#\# LOG SCALING

\hypertarget{colour}{%
\paragraph{COLOUR}\label{colour}}

\hypertarget{colour-1}{%
\paragraph{COLOUR}\label{colour-1}}

\hypertarget{visualising-amounts}{%
\paragraph{VISUALISING AMOUNTS}\label{visualising-amounts}}

\citet{wilke2910} sections 6.1, 6.2, 10.2, 10.3 \#\#\#\#\# VISUALISING
AMOUNTS

\hypertarget{why-pie-charts-are-a-piece-of-shit}{%
\paragraph{WHY PIE CHARTS ARE A PIECE OF
SHIT}\label{why-pie-charts-are-a-piece-of-shit}}

\citet{wilke2910} section 10.1 \citet{pie} \#\#\#\#\# WHY PIE CHARTS ARE
A PIECE OF SHIT

\hypertarget{time-series}{%
\paragraph{TIME SERIES}\label{time-series}}

section 13.1 \#\#\#\#\# TIME SERIES

\hypertarget{briefly-3d-and-why-its-bad}{%
\paragraph{BRIEFLY: 3D AND WHY IT'S
BAD}\label{briefly-3d-and-why-its-bad}}

section 26.1 \#\#\#\#\# BRIEFLY: 3D AND WHY IT'S BAD

\hypertarget{taxonomy-paper}{%
\paragraph{TAXONOMY PAPER}\label{taxonomy-paper}}

\citet{taxonomy} \#\#\#\#\# TAXONOMY PAPER

\section{Studies in Visualisation}

There is a large amount of research and literature surrounding the topic
of misleading visualisations, looking into how various techniques can
either deliberately or unintentionally deceive an observer in the
message of the data. Results from some of these papers will be
replicated, as well as used to form hypotheses which this survey will
investigate. A large amount of the literature exploring misleading
tactics in data visualisation focuses mainly on bar plots and line plots
for categorical and time series data, and so this is what the study and
literature review will focus on.

The 2020 paper ``The Deceptive Potential of Common Design Tactics Used
in Data Visualizations'' \citep{claire-obrian}, as the title suggests,
explores how using different design tactics may mislead the person
seeing the visualisation. Similarly to ``An Empirical Study of Data
Visualisation'', the Claire and O'Brian paper uses a survey to explore
how deceptive visualisation techniques can be employed as well as their
impact on perception of the data. The survey discussed in this paper
presents the participant with four plots; a bar plot, a line plot, a pie
chart and a bubble plot. Additionally to changing aesthetic features of
the plots themselves, the study investigates the use of exaggerated,
leading titles, for example one control plot has the title " Home Sales
Show Increase From 2015 - 2016``, which is altered to''Huge Increase in
Home Sales From 2015 -- 2016!``. The control plots consist of using a
y-axis scaling beginning at 0 for the bar and line plots, a standard pie
chart, and a bubble plot with proportionally sized bubbles, all
alongside the non-exaggerated titles. The altered plots involve
truncating the y-scale for the bar and line plots, making the pie chart
in 3D, and arbitrarily altering the sizes of the bubbles on the bubble
plot. The altered plots are referred to as the''deceptive" plots. The
survey used sets of plots as crossed between deceptive aesthetics and
deceptive titles; two had control aesthetics, one with the control title
and one for the exaggerated title, and two had deceptive aesthetics with
one having the exaggerated titling.

With regard to truncated axes, Claire and O'Brian asked participants to
subjectively judge the difference between two data points using a 6
point scale ranging from ``a little'' to ``a lot''. For both the bar
plot and line plot it was found the the use of a truncated scale
increases the perceived difference between the data points. The use of a
truncated scale is also discussed by \citet{YANG2021}, whereby 5
empirical studies were performed in order to assess the effect of
altering the scale in this way. The first of the 5 studies once again
assessed how large the difference between data points is perceived to be
in the truncated plot as compared to a control, again using a subjective
scale from ``Not at all different'' to ``Extremely different'' on a 7
point scale. This scale differed, however, in the way that a midpoint
label of ``Moderately different'' was provided. The 7 point scale may be
preferable to the 6 point scale as the 7 point has a defined midpoint at
4, whereas the 6 point does not. This study once again concludes that
the differences in data points tended to be perceived as larger than for
the control plot. Alongside these studies, a 2014 blog post
\citep{parikh_2014} discusses axis truncation and its effect on
perceived data point difference for bar plots alongside other aesthetic
features. The first example shows how truncating the y-axis of a bar
plot can over-exaggerate differences in the heights of the bars, perhaps
leading to incorrect observations regarding comparisons of values within
the data.

The paper \citet{stackscale} performs a similar study, but instead
investigates the use of `stack-scale', or `stacked' bar charts and
logarithmic scaling. The aim of the study was to explore whether
stack-scale bar charts are an effective way to visualise large value
data, which is less relevant to since the Ninja Warrior and Sales data
are relatively low-valued data compared to the paper, but nevertheless
provides a framework for exploring the use of logarithmic scaling and
stacked bars in a respondent study. Participants were shown three plots;
a control with a linear scale, a bar plot using a stack-scale, and one
with logarithmic scaling. The questions asked determined how the
different scaling affected accuracy in reading individual values,
interpreting differences in values and determining which time-step
exhibits the largest difference in values. \citet{logax} additionally
discusses the use of a logarithmic axis in bar plots, explaining how it
is impossible for a zero value to be displayed on this axis, and thus
the bar start points are arbitrary and produce an inaccurate
representation of the bar height with relation to the true value. To
quote the paper,
\textit{"Don’t create bar graphs using a logarithmic axis if your goal is to honestly show the data"}.
It can be observed that the logarithmic scale makes the perceived
difference appear smaller than in the control.

As well as scaling, another aspect of visualisation design that could
potentially mislead the observer is bar width and aspect ratios. When
adding a visualisation into a publication, re-sizing the visualisation
to fit a specific gap may include altering the aspect ratio, in turn
affecting the length to width ratio of the bars in a bar plot. As
explored by Steven Few in a 2016 article for the
\textit{'Visual Business Intelligence Newsletter'} \citep{Few2016},
altering this ratio can affect viewer perception in the way of a
narrower and taller image distorting bars to appear longer, and vice
versa, meaning that perceived differences between bar heights may be
affected.

Part 2 of the survey will be based around investigating this idea,
alongside how the reading of exact values is affected. The second
section of the survey tests whether altering aspect ratio of plots
affects interpretation. The purpose of this is to mirror what my occur
when visualisations are published, and may be resized to fit the section
of the page they sit on. As in \citep{Few2016}, it will be hypothesised
that an aspect ratio that effectively narrows the bars may cause
overestimation in values, and vice versa, using a ratio that widens bars
could lead to underestimation. In the paper, the author discusses how
increasing the widths of bars could distract from the bar height as well
as take up excessive space on a page. It is also mentioned that wider
bars may be ``aesthetically displeasing''. This section of the survey
will test both how bar width alters perceived difference between bars as
well as opinions on the aesthetics. The method in the paper also
involves altering spaces between bars, including bar plots with spaces
at 50\% of the bar widths and then reducing the width of the space by a
third. Conversely to this, width of spaces between bars will not be
considered, only the effective widths of the bars themselves. The author
concludes that a length-to-width ratio of 10:1 appears to suffer from
perceptual imbalance, but increasing this such that the bars become
narrower and longer does not appear to have as much of an impact; the
ratio can be increased relatively far with out causing much perceptual
imbalance.

An article from the University of Stuttgart \citep{HuynhHaiDang2017}
gives an overview of many types of bar chart, including stacked and
grouped bars. The author remarks that grouped bar charts may make the
comparison of bars in the same category more difficult, while the
stacked bar chart sacrifices ease of comparison of values in the bars
for increased spacial efficiency. A 2018 work from the journal of
\textit{'Visual Informatics'} \citep{INDRATMO2018155} also provides a
discussion on the use of various forms of stacked and grouped bar charts
and their efficacy. The paper notes how a classical stacked bar chart
can be useful for overall comparisons as the height of the bar
represents the value of the item, with the different attributes depicted
as a segmentation of this single bar into different colours. When
discussing grouped bar charts it is mentioned that stacked bar charts
may be less useful when performing attribute comparisons, in other words
comparisons between different categories on the same bar, as a result of
the bar segments being non-aligned. This results in comparison taking
the form of length judgment as opposed to position judgment. Cleveland
and McGill in their 1984 article in the
\textit{'Journal of the American Statistical Association'}
\citep{clevelandmcgill} discuss how judgments based on length are likely
to be less accurate than those based on position. A grouped bar chart is
a way to allow for easy comparison between individual categories, but is
discussed to be less effective in overall comparison.

\backmatter
  \bibliography{main.bib}

\end{document}
