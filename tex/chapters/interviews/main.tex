% TODO decide which bits of code to show - whole code is lengthy. end in advance?
% TODO maybe start by showing code for basic barplots (I have this in all three languages) then go more in depth for R and Py
% TODO CITE: [1] https://science-education-research.com/research-methodology/research-techniques/interviews/semi-structured-interviews/
Notes on semi-structured interviews:
\begin{itemize}
    \item Let questions guide discussion, but keep it open
    \item Mainly open ended questions
    \item Won't necesarily touch on all questions
    \item Potentially just a list of topics as opposed to questions
    \item 
\end{itemize}

Structured - Follows pre-determined questions in specific order
Unstructured  - Conversation led by interviewee

Potential Questions: 
\begin{itemize}
    \item Do you have a personal bias towards any language?
    \item Do you have any initial comments on the code?
    \item Which do you feel would be more efficient? (bias here - I am more efficient in R than Py or js)
    \item Which do you feel would be more labour intensive? (bias - I prefer R, but others prefer Py)
    \item How much freedom do you feel each language allows for customisation of features? (ie. colours, axes etc)
    \item Which do you feel seems more intuitive? (bias)
    \item How do you feel each code could be improved, in your opinion?
    \item How reproducible/replicable do you feel these codes are?
    \item How well suited to visualisation to you think each language seems to be, based on these codes as well as your own knowledge?
    \item How intuitive do you feel each code is? (ggplot vs matplotlib)
    \item Which library, ggplot or matplotlib, do you feel would be easier for a beginner with an equal amount of R and Python experience to learn? Why?
    \item Which language do you feel provides a more publication-ready output?
    \item 
    \item Any other comments?
\end{itemize}