\chapter{Data collection}

\section{Background on survey design}

% TODO cite: https://books.google.co.uk/books?id=mSRTDwAAQBAJ&lpg=PP1&ots=nynKGurXVG&dq=how%20to%20write%20a%20survey&lr&pg=PP1#v=onepage&q&f=false
% TODO cite: https://www.idsurvey.com/en/advantages-and-disadvantages-of-phone-survey/

A survey is a method of data collection with the purpose of obtaining quantitative information regarding opinions and 
experiences of the respondents. Surveys could be used to investigate, for example, a product for which a company would like 
consumer feedback, or how a disease impacts patients in their daily lives. This form of investigation is useful for gathering a
large amount of data that can be easily quantified and analysed using statistical methods and tests. 

Survey methodology involves asking a sample of the target population a series of standardised questions, which may be given in 
the form of written questionaires or spoken interviews. Before the type of survey is decided, it is important to consider the 
aims and objectives of the research. 

For example, phone calls and other forms of interview-based survey allow the interviewer to form a personal connection with 
the survey participant, which can be especially helpful for a company's image if the interviewer is particularly professional 
or charismatic. This type of survey also provides an instant response, which is beneficial if there is only a short time frame 
available in which to gather information. Furthermore, any error as a result of participants misinterpreting questions can be 
negated due to the interviewer having the ability to immediately clarify on any misunderstandings. 

However, there are also shortfalls to an interview-based survey method.




% About Questionaires
A questionaire is composed of two types of questions; closed-ended questions, where participants select answers from a list,
and open-ended questions, in which the participants are able to write their own answers. 



\section{Specific goals of survey for this chapter}


\section{The survey}

\textbf{Cohorts}
\begin{itemize}
    \item Students
    \item People who work in data visualisation and/or statistical programming
    \item General population
\end{itemize}

\textbf{Information about participants}

\begin{itemize}
    \item Please select your age category. (Checkbox: Under 18, 18-25, 26-35, 36-45, Above 45)
    
    \item If you are a university student or past university graduate please specify your area of study. (maybe drop down box: Science, Technology, Engineering, Mathematics, Arts, Social Sciences, Humanities, Business, Other, N/A)
    
    \item How would you rate your spatial awareness skills? (Scale/slider: Poor, Okay, Good, Very Good)
    
    \item How would you rate your observation skills? (Scale/slider: Poor, Okay, Good, Very Good)
    
    \item How would you rate your mathematics skills? (Scale/slider: Poor, Okay, Good, Very Good)
    
    \item Do you consider yourself to be colourblind? (Binary checkbox: Y/N)
    
    \item Do you consider yourself to have a visual processing issue (ie. dyslexia)? (Binary checkbox: Y/N)
\end{itemize}

\noindent \textbf{Type of questions}

\begin{itemize}
    \item Take categorical data and show different types of plot with no numbers and as questions such as "which would you say is the largest" etc - Pie chart, bar chart, pictogram.
\end{itemize}


\section{Conclusion}
