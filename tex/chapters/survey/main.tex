\chapter{Data collection}

\section{Background on survey design}

%TODO cite: https://books.google.co.uk/books?id=mSRTDwAAQBAJ&lpg=PP1&ots=nynKGurXVG&dq=how%20to%20write%20a%20survey&lr&pg=PP1#v=onepage&q&f=false
%TODO cite: https://www.idsurvey.com/en/advantages-and-disadvantages-of-phone-survey/
%TODO cite: https://books.google.co.uk/books?hl=en&lr=&id=ctow8zWdyFgC&oi=fnd&pg=PR15&dq=survey+methodology&ots=fgcIbBXkS9&sig=LvdOfWOBGrtHFrcVbdfi1ahV970#v=onepage&q=survey%20methodology&f=true

A survey is a method of data collection with the purpose of obtaining quantitative information regarding opinions and 
experiences of the respondents in order to try and describe the views of the target population as a whole, as briefly explained
in %TODO cite third source.
The book also also mentions the systematic nature of surveys, a factor that diffentiates it from other methods of collecting
information. While the author does not go into detail about this, it can be inferred that the systematic format of a survey is part of 
what lends it so well to quantitative analysis; %TODO expand on this

There are many occasions in which a survey may be used for an investigation. A common example is market research; many 
companies use surveys as a way of gather consumer feedback regarding their products. Perfoming this kind of research and analysis
can be highly beneficial for a company. %TODO Currently trying to find a case study

The methodology involves asking a sample of the target population a series of standardised questions, which may be given in 
the form of written questionaires or spoken interviews. Each type of survey has a wide range of advantages and disadvantages, 
and each may serve a different purpose depending on the target audience and aims of the study. 

For example, phone calls and other forms of interview-based survey allow the interviewer to form a personal connection with 
the survey participant, which can be especially helpful for a company's image if the interviewer is particularly professional 
or charismatic. This type of survey also provides an instant response, which is beneficial if there is only a short time frame 
available in which to gather information. Furthermore, any error as a result of participants misinterpreting questions can be 
negated due to the interviewer having the ability to immediately clarify on any misunderstandings. 

However, there are also shortfalls to an interview-based survey method. For instance, although a charismatic interviewer can 
positively impact the image of whoever is conducting the survey, this could also lead to biases, such as the respondent 
answering in a way they feel will please the interviwer.
Additionally, the image of the organisation could potentially be tainted if the interviwer appears rude or unprofessional, 
alongside potentially providing bias in the opposite direction. As well as this, telephone surveys are likely to be 
interpreted as a telemarketing scheme, and thus potentially have a negative impact on the number of willing respondents.
The reduced anonymity of this type of survey may also create bias in the way of participants avoiding making statements
that could be deemed socially unacceptable, or that they feel they may be judged for, and therefore may not provide answers
accurate to their true line of thought.





% About Questionaires
A questionaire is composed of two types of questions; closed-ended questions, where participants select answers from a list,
and open-ended questions, in which the participants are able to write their own answers. 



\section{Specific goals of survey tool for this paper}


\section{The survey}

\textbf{Cohorts}
\begin{itemize}
    \item Students
    \item People who work in data visualisation and/or statistical programming
    \item General population
\end{itemize}

%TODO Specify further


\textbf{Information about participants}

\begin{itemize}
    \item Please enter your age
    
    \item If you are a university student or past university graduate please specify your area of study. (maybe drop down box: Science, Technology, Engineering, Mathematics, Arts, Social Sciences, Humanities, Business, Other (please specify), N/A)

    \item How strongly do you agree with each of the following statements?

    \item I have good spatial awareness skills (Scale/slider)
    
    \item I have good observational skills (Scale/slider)
    
    \item I have good numerical skills (Scale/slider)
    
    \item Are you colourblind? (Binary checkbox: Y/N)
    
    \item Do you have any disorders that may affect visual processing? (this could be a general visual processing disorder 
    or dyslexia, dyscalculia etc)
    (checkbox and open answer box: No, Yes (please specify))
\end{itemize}

\noindent \textbf{Type of questions}

\begin{itemize}
    \item Take categorical data and show different types of plot with no numbers and as questions such as "which would you 
    say is the largest" etc - Pie chart, bar chart, pictogram.
\end{itemize}


\section{Conclusion}
