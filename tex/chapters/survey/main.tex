\chapter{Data collection}

\section{Background on survey design}

%TODO 1 cite: https://books.google.co.uk/books?id=mSRTDwAAQBAJ&lpg=PP1&ots=nynKGurXVG&dq=how%20to%20write%20a%20survey&lr&pg=PP1#v=onepage&q&f=false
%TODO 2 cite: https://www.idsurvey.com/en/advantages-and-disadvantages-of-phone-survey/
%TODO 3 cite: https://books.google.co.uk/books?hl=en&lr=&id=ctow8zWdyFgC&oi=fnd&pg=PR15&dq=survey+methodology&ots=fgcIbBXkS9&sig=LvdOfWOBGrtHFrcVbdfi1ahV970#v=onepage&q=survey%20methodology&f=true
%TODO 4 cite: https://en.wikipedia.org/wiki/Census
%TODO 5 cite: https://www.understandingsociety.ac.uk/
%TODO 6 cite: https://ora.ox.ac.uk/objects/uuid:526114c2-8266-4dee-b663-351119249fd5
%TODO 7 cite: 



As described in [CITE 3 HERE], a survey is a means of obtaining quantitative information regarding opinions and 
experiences of the respondents in order to explore the views of the target population as a whole. This is a systematic
method of collecting data, where the word ``systematic" is deliberately used by the author of [CITE 3 HERE] in order to separate
surveys from other methods, such as focus groups and unstructured interviews. It can be remarked that the systematic nature of
a survey is part of what lends it so well to quantitative analysis; a series of standardised questions can be asked to many
cohorts with little room for interpretation, meaning that large amounts of data can be gathered and easily compiled to
form a coherent rolling analysis. Focus groups and other less structured formats, on the other hand, are much more difficult 
to analyse statistically, and are better suited for gaining general insights rather than to undergo rigorous statistical 
testing. Both surveys and focus groups will be used to gather information for this study, and focus groups will be discussed 
more in depth in a later chapter.

As mentioned above, survey methodology involves asking a sample of the target population a series of standardised questions, 
and these may be asked in the form of written questionaires or structured interviews. Depending on the aims of the study, there
will be many benefits and downfalls of each method, and there may be times when a combined approach is required to
gather the necessary information. 

In particular, The UK Household Longitudinal Study is an example of the use of a combined approach to gather information. 
This is an ongoing study, and began in 2009, when a sample of 40,000 households in the UK were selected to be surveyed on a 
yearly basis. This was considered 'wave 1' of the study, and involved all members of each household, overall comprising of
around 100,000 individials,
%TODO Expand on this


Phone calls and other forms of interview-based survey allow the interviewer to form a personal connection with 
the survey participant, which can be especially helpful for a company's image if the interviewer is particularly professional 
or charismatic. Additionally, while the interviewer will still be limited to asking the pre-set questions, the format of such a survey can 
be considered semi-structured and with much more room for interpretation. This can lend itself to gaining additional insights
that may not have otherwise been gathered from a more closed-form paper or online survey. Additionally, the more open format
can negate any error as a result of participants misinterpreting questions due to the interviewer's ability to immediately 
clarify on any misunderstandings. This type of survey also provides an instant response, which is beneficial if there is only 
a short time frame available in which to gather information. 

However, there are also shortfalls to an interview-based survey method. For instance, although a charismatic interviewer can 
positively impact the image of whoever is conducting the survey, this could also lead to biases, such as the respondent 
answering in a way they feel will please the interviwer.
Additionally, the image of the organisation could potentially be tainted if the interviwer appears rude or unprofessional, 
alongside potentially providing bias in the opposite direction. As well as this, telephone surveys are likely to be 
interpreted as a telemarketing scheme, and thus potentially have a negative impact on the number of willing respondents.
The reduced anonymity of this type of survey may also create bias in the way of participants avoiding making statements
that could be deemed socially unacceptable, or that they feel they may be judged for, and therefore may not provide answers
accurate to their true line of thought.

A second way of presenting survey questions, and the way that will be used to present them for this study, is a written questionaire.
A questionaire may either consist of physical paper forms mailed or handed out to people within the target population or may
exist in an online format. A questionaire is composed of two types of questions; closed-ended questions, where participants 
select answers from a list, and open-ended questions, in which the participants are able to write their own answers. 


\section{Specific goals of survey tool for this study}
This study will explore how best to represent various data types visually. One way in which this will be investigated is to create 
a series of online questionaires in order to determine how visualisations can impact the viewer's perception of the plotted data. 
While visualisations can be a very useful tool in understanding data, they also have the potential to mislead the observer, which 
could be either done accidentally or on purpose. As discussed by Gatto in [CITE 6 HERE], altering properties such as the range of 
the axes, size of bars on a bar plot, and colouring of the plot can all influence what is inferred about the data when observing 
a visualisation. Furthering Dr. Gatto's argument, she goes on to reference a 2014 blog post [CITE 7 HERE] entitled 'How to Lie with 
Data Visualisation'. This post gives a series of examples depicting how the afformentioned properties affect interpretation. The first 
example shows how truncating the y-axis of a bar plot can overexaggerate differences in the height of bars, perhaps leading incorrect 
observations about the data. For example, if the data consists of values, such as profits, at different time points, then truncating 
the y-axis can at first make the profit increase each year seem much greater than it actually is. Only when one reads the scale on the 
axis will it be observed that there may only be a very marginal, and potentially almost negligible, difference. Of course if someone 
were to read the y-axis scaling they would determine that the change was in fact small, but the human brain thrives on visual and 
pictorial feedback, and thus it is likely that a person would look at this plot and simply draw conclusions based on the plot itself
without reference to the truncated axis. Therefore, a first hypothesis that will be tested in this survey is whether truncating the y-axis
does indeed impact how the data is interpreted. Additionally, the use of a logarithmic scale will be investigated, and it will
be hypothesised that this will also impact interpretation.

%%%%% TODO:

% Perform meta-analysis of different sources to find optimal number of survey questions?
% "20 minute rule" -> Don't go beyond 20 mins
% Work out how long each question will take... Guessing 1-2 mins per question => ~10-20 questions?
% Long surveys can increase measurement error
% 

%%%%% Potential sources about survey length %%%%%
% https://journals.sagepub.com/doi/pdf/10.2501/IJMR-2017-039 -> Study about ideal length of survey in terms of time
% https://ojs.ub.uni-konstanz.de/srm/article/view/7145 -> Study about reducing measurement error due to survey length
% https://corescholar.libraries.wright.edu/etd_all/1918/ -> About 'careless responding' and 'insufficient effort responding'
% https://journals.sagepub.com/doi/abs/10.1177/089443930101900202 
 
%%%%% Potential questions: %%%%%

% Would you say there is a large difference between the weights of these dogs? (Maybe too subjective)
% -> Show plot with logarithmic and then normal scale (ie in a "and how about now?" format)
 
% Which other breed has the same weight as the [BREED]?
% -> Show with truncated y-axis
% -> Show with truncated y-axis but in order
 
% Which other breed has the same height as the [BREED]?
% -> Show with truncated y-axis
% -> Show with normal y-axis

% Which obstacle was used the least?
% -> Pie vs Bar vs Scatter

% Which variable varies most over time?/Which shows the most dramatic change over time?
% -> Two time-series line plots, one with both variables on same plot, one on different plots.
% -> Use data with very different scalings

% Not sure on exqct question yet, but take lots of data points (ie 20 obstacles) and use large legend and lots of colour ("bad" plot)
% Then show same plot but with labels on points and no legend ("Good" plot)
% -> Hypothesise that plot with labels will be more accurately interpreted.
% -> Source: https://towardsdatascience.com/color-in-data-visualization-less-how-more-why-348514a3c4d8

% Which obstacles were used [AMOUNT] times?
% -> used in conujunction with differently scales plots/barplots?

%%%%%

Two types of questionaire will be written for this study. The first type will present a series of fairly basic visualisations accompanied
by questions created to guage whether each presented visualisation has an impact on how the data is interpreted by the survey participant.


The set will consist of three separate surveys, which will be identical up to the visualisation package used. Particularly, one will contain 
visualisations made with R's ggplot2, the next with matplotlib from Python, and the last with the JavaScript library D3. These surveys will 
be distributed to the general public by sharing links on social media platforms such as Facebook. The reasoning behind creating three separate 
surveys in a variety of languages is to ascertain whether 






%TODO explain the second type of survey further
The second type will be a single survey comparing the 
implementation of the different languages and will be distributed to a group specialising in visual analytics for pharmaceutical 
research. This second survey aims to explore opinions on the coding languages themselves, in terms of features such as 
readability, reproducibility and ease of implementation.

%TODO explain that I'm using questionaires and why


In the set of three surveys we aim to answer questions such as: 

\begin{itemize}
    \item In which form is the given information of interest most accurately interpreted by the viewer?
    \item What factors of a plot can bias interpretation? 
    \item Does the visualisation tool used have an impact on interpretation?
    \item Does the tool used have an impact on opinions regarding aesthetic features?
\end{itemize}


% Idea: Since I have had to extract things from ninja warrior data (lots of data), could show data and then a plot to show how
%       visualisations are useful to find things you can't immediately see in the data itself.
% Idea: Axes in time series plots - Use short and tall axes and ask which varies most? (Maybe use HR data from http://ecg.mit.edu/time-series/)
% Idea: Distance between bars in barplot - further apart harder to interpret?
% Idea: Does uning a logarithmic scale impact interpretation?

\section{The survey}

Potentially will use google sheets and post all three, and ask people to complete one of them.
%TODO Specify further


\textbf{Information about participants}

%These questions are pretty much finalised

\begin{itemize}
    \item Please enter your age
    
    \item (maybe drop down box: Science, Technology, Engineering, Mathematics, Arts, Social Sciences, Humanities, Business, Other (please specify), N/A)

    \item How strongly do you agree with each of the following statements?

    \item I have good spatial awareness skills (Scale/slider)
    
    \item I have good observational skills (Scale/slider)
    
    \item I have good numerical skills (Scale/slider)
    
    \item Are you colourblind? (Binary checkbox: Y/N)
    
    \item Do you have any disorders that may affect visual processing? (this could be a general visual processing disorder 
    or dyslexia, dyscalculia etc)
    (checkbox and open answer box: No, Yes (please specify))
\end{itemize}

%Note: Deliberately design 'bad' as well as good plots?

Potential Questions






\section{Conclusion}
