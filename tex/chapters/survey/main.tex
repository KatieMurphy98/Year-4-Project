\chapter{Data collection}

\section{Background on survey design}

%TODO 1 cite: https://books.google.co.uk/books?id=mSRTDwAAQBAJ&lpg=PP1&ots=nynKGurXVG&dq=how%20to%20write%20a%20survey&lr&pg=PP1#v=onepage&q&f=false
%TODO 2 cite: https://www.idsurvey.com/en/advantages-and-disadvantages-of-phone-survey/
%TODO 3 cite: https://books.google.co.uk/books?hl=en&lr=&id=ctow8zWdyFgC&oi=fnd&pg=PR15&dq=survey+methodology&ots=fgcIbBXkS9&sig=LvdOfWOBGrtHFrcVbdfi1ahV970#v=onepage&q=survey%20methodology&f=true
%TODO 4 cite: https://en.wikipedia.org/wiki/Census


As described in [CITE 3 HERE], surveys are used for obtaining quantitative information regarding opinions and 
experiences of the respondents in order to explore the views of the target population as a whole. A survey is a systematic
method of collecting data, where the word "systematic" is deliberately used by the author of [CITE 3 HERE] in order to separate
surveys from other methods, such as focus groups and unstructured interviews. It can be remarked that the systematic nature of
a survey is part of what lends it so well to quantitative analysis; a series of standardised questions can be asked to many
cohorts with little room for interpretation, meaning that large amounts of data can be gathered and easily compiled to
form a coherent rolling analysis. Focus groups and other less structured formats are much more difficult to analyse, and are
better suited for gaining general insights rather than to undergo rigorous statistical analysis. Both surveys and focus groups
will be used to gather information for this study, and focus groups will be discussed more in depth in a later section.

A census is a well known use of survey methodology and is defined by [CITE 4 HERE] to be \textit{"the procedure of systematically 
enumerating, and acquiring and recording information about the members of a given population"}. Unlike most other surveys however, 
where only a sample of the target population is surveyed, a census attempts to survey all members of the population of 
interest. %TODO Expand this section
%TODO Write a paragraph joining this to the section below

The methodology involves asking a sample of the target population a series of standardised questions, which may be given in 
the form of written questionaires or structured interviews. Each type of survey has a wide range of advantages and disadvantages, 
and each may serve a different purpose depending on the target audience and aims of the study. 

For example, phone calls and other forms of interview-based survey allow the interviewer to form a personal connection with 
the survey participant, which can be especially helpful for a company's image if the interviewer is particularly professional 
or charismatic. Additionally, while the interviewer will still be limited to asking the pre-set questions, the format of such a survey can 
be considered semi-structured and with much more room for interpretation. This can lend itself to gaining additional insights
that may not have otherwise been gathered from a more closed-form paper or online survey. Additionally, the more open format
can negate any error as a result of participants misinterpreting questions due to the interviewer's ability to immediately 
clarify on any misunderstandings. This type of survey also provides an instant response, which is beneficial if there is only 
a short time frame available in which to gather information. 

However, there are also shortfalls to an interview-based survey method. For instance, although a charismatic interviewer can 
positively impact the image of whoever is conducting the survey, this could also lead to biases, such as the respondent 
answering in a way they feel will please the interviwer.
Additionally, the image of the organisation could potentially be tainted if the interviwer appears rude or unprofessional, 
alongside potentially providing bias in the opposite direction. As well as this, telephone surveys are likely to be 
interpreted as a telemarketing scheme, and thus potentially have a negative impact on the number of willing respondents.
The reduced anonymity of this type of survey may also create bias in the way of participants avoiding making statements
that could be deemed socially unacceptable, or that they feel they may be judged for, and therefore may not provide answers
accurate to their true line of thought.

A second way of presenting survey questions, and the way that will be used to present them for this study, is a written questionaire.
This may consist of paper forms mailed or handed out to people within the target population, but in the modern day is more 
commonly online. A questionaire is composed of two types of questions; closed-ended questions, where participants select answers 
from a list, and open-ended questions, in which the participants are able to write their own answers. 



\section{Specific goals of survey tool for this paper}


\section{The survey}

\textbf{Cohorts}
\begin{itemize}
    \item Students
    \item People who work in data visualisation and/or statistical programming
    \item General population
\end{itemize}

%TODO Specify further


\textbf{Information about participants}

%These questions are pretty much finalised

\begin{itemize}
    \item Please enter your age
    
    \item If you are a university student or past university graduate please specify your area of study. (maybe drop down box: Science, Technology, Engineering, Mathematics, Arts, Social Sciences, Humanities, Business, Other (please specify), N/A)

    \item How strongly do you agree with each of the following statements?

    \item I have good spatial awareness skills (Scale/slider)
    
    \item I have good observational skills (Scale/slider)
    
    \item I have good numerical skills (Scale/slider)
    
    \item Are you colourblind? (Binary checkbox: Y/N)
    
    \item Do you have any disorders that may affect visual processing? (this could be a general visual processing disorder 
    or dyslexia, dyscalculia etc)
    (checkbox and open answer box: No, Yes (please specify))
\end{itemize}


%Data from: https://data.world/ninja/anw-obstacle-history

The following data shows data relating to 5 of the most used obstacles from the TV show American Ninja Warrior. The questions
in this section are related to visualisations of this data.

%Note: Deliberately design 'bad' as well as good plots?
%TODO write these questions up in a paragraph and explain how I'll get answers. Also give some hypotheses and later perform tests?
Questions I want answers to:
- In which form is the information of interest most accurately interpreted by the viewer?
- What factors can bias interpretation? (ie. bolder colours/larger sized points/different widths of bars)
- Does ordering of bars make a difference? ie. For data close in values, having two similarly valued bars next to each other
  rather than far apart affects interpretation)
      - Could lead to a discussion about how this could cause siginificant trouble when scales are, for example, in millions etc.
- Does the visualisation tool used have an impact on interpretation?
- Does the tool used have an impact on opinions regarding aesthetic features?



\section{Conclusion}
