\chapter{Data collection}

\section{Background on survey design}

%TODO 1 cite: https://books.google.co.uk/books?id=mSRTDwAAQBAJ&lpg=PP1&ots=nynKGurXVG&dq=how%20to%20write%20a%20survey&lr&pg=PP1#v=onepage&q&f=false
%TODO 2 cite: https://www.idsurvey.com/en/advantages-and-disadvantages-of-phone-survey/
%TODO 3 cite: https://books.google.co.uk/books?hl=en&lr=&id=ctow8zWdyFgC&oi=fnd&pg=PR15&dq=survey+methodology&ots=fgcIbBXkS9&sig=LvdOfWOBGrtHFrcVbdfi1ahV970#v=onepage&q=survey%20methodology&f=true
%TODO 4 cite: https://en.wikipedia.org/wiki/Census
%TODO 5 cite: https://www.understandingsociety.ac.uk/


As described in [CITE 3 HERE], a survey is a means of obtaining quantitative information regarding opinions and 
experiences of the respondents in order to explore the views of the target population as a whole. This is a systematic
method of collecting data, where the word "systematic" is deliberately used by the author of [CITE 3 HERE] in order to separate
surveys from other methods, such as focus groups and unstructured interviews. It can be remarked that the systematic nature of
a survey is part of what lends it so well to quantitative analysis; a series of standardised questions can be asked to many
cohorts with little room for interpretation, meaning that large amounts of data can be gathered and easily compiled to
form a coherent rolling analysis. Focus groups and other less structured formats, on the other hand, are much more difficult 
to analyse statistically, and are better suited for gaining general insights rather than to undergo rigorous statistical 
testing. Both surveys and focus groups will be used to gather information for this study, and focus groups will be discussed 
more in depth in a later chapter.

As mentioned above, survey methodology involves asking a sample of the target population a series of standardised questions, 
and these may be asked in the form of written questionaires or structured interviews. Depending on the aims of the study, there
will be many benefits and downfalls of each method, and there may be times when a combined approach is required to
gather the necessary information. 

In particular, The UK Household Longitudinal Study is an example of the use of a combined approach to gather information. 
This is an ongoing study, and began in 2009, when a sample of 40,000 households in the UK were selected to be surveyed on a 
yearly basis. This was considered 'wave 1' of the study, and involved all members of each household, overall comprising of
around 100,000 individials,
%TODO Expand on this


Phone calls and other forms of interview-based survey allow the interviewer to form a personal connection with 
the survey participant, which can be especially helpful for a company's image if the interviewer is particularly professional 
or charismatic. Additionally, while the interviewer will still be limited to asking the pre-set questions, the format of such a survey can 
be considered semi-structured and with much more room for interpretation. This can lend itself to gaining additional insights
that may not have otherwise been gathered from a more closed-form paper or online survey. Additionally, the more open format
can negate any error as a result of participants misinterpreting questions due to the interviewer's ability to immediately 
clarify on any misunderstandings. This type of survey also provides an instant response, which is beneficial if there is only 
a short time frame available in which to gather information. 

However, there are also shortfalls to an interview-based survey method. For instance, although a charismatic interviewer can 
positively impact the image of whoever is conducting the survey, this could also lead to biases, such as the respondent 
answering in a way they feel will please the interviwer.
Additionally, the image of the organisation could potentially be tainted if the interviwer appears rude or unprofessional, 
alongside potentially providing bias in the opposite direction. As well as this, telephone surveys are likely to be 
interpreted as a telemarketing scheme, and thus potentially have a negative impact on the number of willing respondents.
The reduced anonymity of this type of survey may also create bias in the way of participants avoiding making statements
that could be deemed socially unacceptable, or that they feel they may be judged for, and therefore may not provide answers
accurate to their true line of thought.

A second way of presenting survey questions, and the way that will be used to present them for this study, is a written questionaire.
A questionaire may either consist of physical paper forms mailed or handed out to people within the target population or may
exist in an online format. A questionaire is composed of two types of questions; closed-ended questions, where participants 
select answers from a list, and open-ended questions, in which the participants are able to write their own answers. 


\section{Specific goals of survey tool for this study}
In this study, we aim to explore the best ways in which to represent various types of data. Two types of survey will be written;
one type containing questions on basic visualisations to gather information regarding various features, such as how changing various
aesthetic features can impact visual interpretation of the given data, and the other containing some more complex visualisations
along with snippets from the code used to create them. The first type will consist of three separate surveys, where each uses 
similar visualisations, but created in a different language. Particularly, one will contain visualisations made in R's ggplot2, 
one made with matplotlib from Python, and one with the JavaScript library D3. These surveys will be distributed to the general 
public by sharing links on social media platforms such as Facebook. The second type will be a single survey comparing the 
implementation of the different languages and will be distributed to a group of visual analytics specialists. This second survey 
aims to explore opinions on the coding languages themselves, in terms of features such as readability, reproducibility and 
ease of implementation.

In the set of three surveys we aim to answer questions such as: 

\begin{itemize}
    \item In which form is the given information of interest most accurately interpreted by the viewer?
    \item What factors of a plot can bias interpretation? 
    \item Does the visualisation tool used have an impact on interpretation?
    \item Does the tool used have an impact on opinions regarding aesthetic features?
\end{itemize}



\section{The survey}

Potentially will use google sheets and post all three, and ask people to complete one of them.



\textbf{Cohorts}
\begin{itemize}
    \item Students
    \item People who work in data visualisation and/or statistical programming
    \item General population
\end{itemize}

%TODO Specify further


\textbf{Information about participants}

%These questions are pretty much finalised

\begin{itemize}
    \item Please enter your age
    
    \item (maybe drop down box: Science, Technology, Engineering, Mathematics, Arts, Social Sciences, Humanities, Business, Other (please specify), N/A)

    \item How strongly do you agree with each of the following statements?

    \item I have good spatial awareness skills (Scale/slider)
    
    \item I have good observational skills (Scale/slider)
    
    \item I have good numerical skills (Scale/slider)
    
    \item Are you colourblind? (Binary checkbox: Y/N)
    
    \item Do you have any disorders that may affect visual processing? (this could be a general visual processing disorder 
    or dyslexia, dyscalculia etc)
    (checkbox and open answer box: No, Yes (please specify))
\end{itemize}



%Note: Deliberately design 'bad' as well as good plots?



- Axes in time series plots - Used short and tall axes and ask which varies most? (Maybe use HR data from http://ecg.mit.edu/time-series/)





\section{Conclusion}
