\documentclass[12pt]{book}
\usepackage[utf8]{inputenc}
\usepackage[english]{babel}
\usepackage{setspace}
\usepackage[]{pdfpages}
\doublespacing

\title{\textbf{An Empirical Study of Visualisation of Data}}
\author{Katie Murphy \\ 1632254}

\begin{document}
\maketitle

% \begin{abstract}
%     \begin{center}
%         \noindent \textbf{A study into the theory behind data visualisation.} 
%       
%          \vspace{0.3cm}
%       
%         \noindent This paper will investigate various ways in which data can be interpreted visually to portray information effectively in an elegant and concise manner. 
%       
%         \vspace{0.3cm}
%       
%         \noindent The investigation will include analysing several types of visualisation alongside discussing the benefits and downfalls of different software packages, namely the visualisation tools contained within R, Python and JavaScript.
%       
%         \vspace{0.3cm}
%       
%         An additional exploration will be made into discovering how understanding human psychology and visual processing can assist in creating effective visualisations.
%       
%   
%     \end{center}    
%    
% \end{abstract}
%
%
   \newpage

   \section{Introduction}
   
   \begin{itemize}
   \item What is visualisation?
   \item Understanding theory of visualisation and implementation in Python, R and D3.
   \item Types of visualisation (ie. static, dynamic and interactive)
   \end{itemize}
   
   Principles of data vis:
   \begin{enumerate}
       \item The audience (who?)
       \item The message or questions (what?)
       \item The circumstances under which the audience interacts with the data (smartphone/laptop etc) (how?)
   \end{enumerate}
   
   \section{Literature reviews}
   Look at advances in visualisation practices based on when papers were written.
   
   \noindent (ie. in an older paper they preferred this, but in a more recent paper they preferred this)
   
   \noindent Is this down to the author's personal preference or do multiple papers back this up?
   
   \vspace{1.2cm}
   \noindent \textbf{The Eyes Have It: A Task by Data Type Taxonomy for Information Visualizations}
   
   \section{Implementation of Visualisation}
   \begin{itemize}
   \item In Plotly, ggplot and D3
   \item Ease of use
   \item Consider data analysis tools available
   \item Interactivity of visualisations (pros/cons) (using Shiny, plotly and D3) 
   \item Reproducibility of visualisations (ie. use of out of software editing such as word or powerpoint to change labels etc)
   \item Publication ready output?
   \item Concept of storytelling
   \end{itemize}
   
   \section{Respondent Study}
   
   \begin{itemize}
   \item How the human mind interprets visuals
   \item Focus groups and/or survey
   \item Replicate research
   \item Intuitive visualisations
   \end{itemize}


    % Options for packages loaded elsewhere
\PassOptionsToPackage{unicode}{hyperref}
\PassOptionsToPackage{hyphens}{url}
%
\documentclass[
]{article}
\usepackage{lmodern}
\usepackage{amssymb,amsmath}
\usepackage{ifxetex,ifluatex}
\ifnum 0\ifxetex 1\fi\ifluatex 1\fi=0 % if pdftex
  \usepackage[T1]{fontenc}
  \usepackage[utf8]{inputenc}
  \usepackage{textcomp} % provide euro and other symbols
\else % if luatex or xetex
  \usepackage{unicode-math}
  \defaultfontfeatures{Scale=MatchLowercase}
  \defaultfontfeatures[\rmfamily]{Ligatures=TeX,Scale=1}
\fi
% Use upquote if available, for straight quotes in verbatim environments
\IfFileExists{upquote.sty}{\usepackage{upquote}}{}
\IfFileExists{microtype.sty}{% use microtype if available
  \usepackage[]{microtype}
  \UseMicrotypeSet[protrusion]{basicmath} % disable protrusion for tt fonts
}{}
\makeatletter
\@ifundefined{KOMAClassName}{% if non-KOMA class
  \IfFileExists{parskip.sty}{%
    \usepackage{parskip}
  }{% else
    \setlength{\parindent}{0pt}
    \setlength{\parskip}{6pt plus 2pt minus 1pt}}
}{% if KOMA class
  \KOMAoptions{parskip=half}}
\makeatother
\usepackage{xcolor}
\IfFileExists{xurl.sty}{\usepackage{xurl}}{} % add URL line breaks if available
\IfFileExists{bookmark.sty}{\usepackage{bookmark}}{\usepackage{hyperref}}
\hypersetup{
  pdftitle={Univariate Analysis - Whole Pop},
  pdfauthor={Katie Murphy},
  hidelinks,
  pdfcreator={LaTeX via pandoc}}
\urlstyle{same} % disable monospaced font for URLs
\usepackage[margin=1in]{geometry}
\usepackage{graphicx,grffile}
\makeatletter
\def\maxwidth{\ifdim\Gin@nat@width>\linewidth\linewidth\else\Gin@nat@width\fi}
\def\maxheight{\ifdim\Gin@nat@height>\textheight\textheight\else\Gin@nat@height\fi}
\makeatother
% Scale images if necessary, so that they will not overflow the page
% margins by default, and it is still possible to overwrite the defaults
% using explicit options in \includegraphics[width, height, ...]{}
\setkeys{Gin}{width=\maxwidth,height=\maxheight,keepaspectratio}
% Set default figure placement to htbp
\makeatletter
\def\fps@figure{htbp}
\makeatother
\setlength{\emergencystretch}{3em} % prevent overfull lines
\providecommand{\tightlist}{%
  \setlength{\itemsep}{0pt}\setlength{\parskip}{0pt}}
\setcounter{secnumdepth}{-\maxdimen} % remove section numbering

\title{Univariate Analysis - Whole Pop}
\author{Katie Murphy}
\date{27/02/2021}

\begin{document}
\maketitle

\hypertarget{q1---approximately-many-times-would-you-say-the-salmon-ladder-was-used}{%
\paragraph{\texorpdfstring{\textbf{Q1 - Approximately many times would
you say the `Salmon Ladder' was
used?}}{Q1 - Approximately many times would you say the `Salmon Ladder' was used?}}\label{q1---approximately-many-times-would-you-say-the-salmon-ladder-was-used}}

\begin{verbatim}
## [1] n = 70
\end{verbatim}

\begin{verbatim}
##    con_1_all       trn_1_all       log_1_all        
##  Min.   :40.00   Min.   :40.00   Min.   :9.000e+00  
##  1st Qu.:41.00   1st Qu.:41.00   1st Qu.:3.000e+01  
##  Median :41.00   Median :41.00   Median :3.500e+01  
##  Mean   :41.21   Mean   :41.35   Mean   :1.493e+13  
##  3rd Qu.:42.00   3rd Qu.:42.00   3rd Qu.:4.000e+01  
##  Max.   :45.00   Max.   :45.00   Max.   :1.000e+15  
##                                  NA's   :3
\end{verbatim}

\hypertarget{nas}{%
\paragraph{NAs}\label{nas}}

\begin{verbatim}
##      index value          
## [1,] "23"  "Don't know"   
## [2,] "25"  "Next to none."
## [3,] "68"  NA
\end{verbatim}

\begin{verbatim}
##                                   uni sp_aware obs_skl num_skl cblind vis_pro
## 101                        Technology        4       4       3     No      No
## 121                              <NA>        4       3       3     No      No
## 105 Sustainability/geological science        3       4       3     No    ADHD
\end{verbatim}

\hypertarget{q2---approximately-how-much-more-than-log-grip-would-you-say-salmon-ladder-was-was-used}{%
\paragraph{\texorpdfstring{\textbf{Q2 - Approximately how much more than
`Log Grip' would you say `Salmon Ladder' was was
used?}}{Q2 - Approximately how much more than `Log Grip' would you say `Salmon Ladder' was was used?}}\label{q2---approximately-how-much-more-than-log-grip-would-you-say-salmon-ladder-was-was-used}}

\begin{verbatim}
## [1] n = 70
\end{verbatim}

\begin{verbatim}
##    con_2_all       log_2_all       trn_2_all    
##  Min.   :3.000   Min.   :1.000   Min.   :1.000  
##  1st Qu.:4.250   1st Qu.:2.250   1st Qu.:5.000  
##  Median :5.000   Median :3.500   Median :6.000  
##  Mean   :5.357   Mean   :3.671   Mean   :5.871  
##  3rd Qu.:6.000   3rd Qu.:5.000   3rd Qu.:7.000  
##  Max.   :7.000   Max.   :7.000   Max.   :7.000
\end{verbatim}

\hypertarget{q3---approximately-how-much-more-than-quintuple-steps-would-you-say-salmon-ladder-was-used}{%
\paragraph{\texorpdfstring{\textbf{Q3 - Approximately how much more than
`Quintuple Steps' would you say `Salmon Ladder' was
used?}}{Q3 - Approximately how much more than `Quintuple Steps' would you say `Salmon Ladder' was used?}}\label{q3---approximately-how-much-more-than-quintuple-steps-would-you-say-salmon-ladder-was-used}}

\begin{verbatim}
## [1] n = 70
\end{verbatim}

\begin{verbatim}
##    con_3_all       log_3_all       trn_3_all    
##  Min.   :3.000   Min.   :1.000   Min.   :1.000  
##  1st Qu.:4.250   1st Qu.:2.250   1st Qu.:5.000  
##  Median :5.000   Median :3.500   Median :6.000  
##  Mean   :5.357   Mean   :3.671   Mean   :5.871  
##  3rd Qu.:6.000   3rd Qu.:5.000   3rd Qu.:7.000  
##  Max.   :7.000   Max.   :7.000   Max.   :7.000
\end{verbatim}

\hypertarget{q4---in-your-opinion-approximately-how-many-times-would-you-say-log-grip-was-used-as-a-percentage-of-the-number-of-times-salmon-ladder-was-used}{%
\paragraph{\texorpdfstring{\textbf{Q4 - In your opinion, approximately
how many times would you say `Log Grip' was used, as a percentage of the
number of times `Salmon Ladder' was
used?}}{Q4 - In your opinion, approximately how many times would you say `Log Grip' was used, as a percentage of the number of times `Salmon Ladder' was used?}}\label{q4---in-your-opinion-approximately-how-many-times-would-you-say-log-grip-was-used-as-a-percentage-of-the-number-of-times-salmon-ladder-was-used}}

\begin{verbatim}
## [1] n = 70
\end{verbatim}

\begin{verbatim}
##    con_4_all       log_4_all       trn_4_all     
##  Min.   : 5.00   Min.   : 0.10   Min.   :  1.00  
##  1st Qu.:50.00   1st Qu.: 0.50   1st Qu.: 14.38  
##  Median :50.00   Median : 0.75   Median : 50.00  
##  Mean   :47.66   Mean   :12.86   Mean   : 39.81  
##  3rd Qu.:50.00   3rd Qu.: 0.90   3rd Qu.: 50.00  
##  Max.   :75.00   Max.   :90.00   Max.   :100.00  
##  NA's   :3       NA's   :4       NA's   :2
\end{verbatim}

\hypertarget{nas-1}{%
\paragraph{NAs}\label{nas-1}}

\begin{verbatim}
##      index con_4_all log_4_all trn_4_all
## [1,]    11        NA        NA        NA
## [2,]    48        NA        NA        48
## [3,]    60        NA        NA        NA
## [4,]    68        50        NA        50
\end{verbatim}

\begin{verbatim}
##                                   uni sp_aware obs_skl num_skl cblind vis_pro
## 12                            Science        2       3       2     No      No
## 17                        Engineering        4       4       4     No     Yes
## 25                          Geography        4       4       4   <NA>      No
## 105 Sustainability/geological science        3       4       3     No    ADHD
\end{verbatim}

\end{document}

    
    \chapter{Univariate Analysis}
    % Options for packages loaded elsewhere
\PassOptionsToPackage{unicode}{hyperref}
\PassOptionsToPackage{hyphens}{url}
%
\documentclass[
]{article}
\usepackage{lmodern}
\usepackage{amssymb,amsmath}
\usepackage{ifxetex,ifluatex}
\ifnum 0\ifxetex 1\fi\ifluatex 1\fi=0 % if pdftex
  \usepackage[T1]{fontenc}
  \usepackage[utf8]{inputenc}
  \usepackage{textcomp} % provide euro and other symbols
\else % if luatex or xetex
  \usepackage{unicode-math}
  \defaultfontfeatures{Scale=MatchLowercase}
  \defaultfontfeatures[\rmfamily]{Ligatures=TeX,Scale=1}
\fi
% Use upquote if available, for straight quotes in verbatim environments
\IfFileExists{upquote.sty}{\usepackage{upquote}}{}
\IfFileExists{microtype.sty}{% use microtype if available
  \usepackage[]{microtype}
  \UseMicrotypeSet[protrusion]{basicmath} % disable protrusion for tt fonts
}{}
\makeatletter
\@ifundefined{KOMAClassName}{% if non-KOMA class
  \IfFileExists{parskip.sty}{%
    \usepackage{parskip}
  }{% else
    \setlength{\parindent}{0pt}
    \setlength{\parskip}{6pt plus 2pt minus 1pt}}
}{% if KOMA class
  \KOMAoptions{parskip=half}}
\makeatother
\usepackage{xcolor}
\IfFileExists{xurl.sty}{\usepackage{xurl}}{} % add URL line breaks if available
\IfFileExists{bookmark.sty}{\usepackage{bookmark}}{\usepackage{hyperref}}
\hypersetup{
  pdftitle={Univariate Analysis - Whole Pop},
  pdfauthor={Katie Murphy},
  hidelinks,
  pdfcreator={LaTeX via pandoc}}
\urlstyle{same} % disable monospaced font for URLs
\usepackage[margin=1in]{geometry}
\usepackage{graphicx,grffile}
\makeatletter
\def\maxwidth{\ifdim\Gin@nat@width>\linewidth\linewidth\else\Gin@nat@width\fi}
\def\maxheight{\ifdim\Gin@nat@height>\textheight\textheight\else\Gin@nat@height\fi}
\makeatother
% Scale images if necessary, so that they will not overflow the page
% margins by default, and it is still possible to overwrite the defaults
% using explicit options in \includegraphics[width, height, ...]{}
\setkeys{Gin}{width=\maxwidth,height=\maxheight,keepaspectratio}
% Set default figure placement to htbp
\makeatletter
\def\fps@figure{htbp}
\makeatother
\setlength{\emergencystretch}{3em} % prevent overfull lines
\providecommand{\tightlist}{%
  \setlength{\itemsep}{0pt}\setlength{\parskip}{0pt}}
\setcounter{secnumdepth}{-\maxdimen} % remove section numbering

\title{Univariate Analysis - Whole Pop}
\author{Katie Murphy}
\date{27/02/2021}

\begin{document}
\maketitle

\hypertarget{q1---approximately-many-times-would-you-say-the-salmon-ladder-was-used}{%
\paragraph{\texorpdfstring{\textbf{Q1 - Approximately many times would
you say the `Salmon Ladder' was
used?}}{Q1 - Approximately many times would you say the `Salmon Ladder' was used?}}\label{q1---approximately-many-times-would-you-say-the-salmon-ladder-was-used}}

\begin{verbatim}
## [1] n = 70
\end{verbatim}

\begin{verbatim}
##    con_1_all       trn_1_all       log_1_all        
##  Min.   :40.00   Min.   :40.00   Min.   :9.000e+00  
##  1st Qu.:41.00   1st Qu.:41.00   1st Qu.:3.000e+01  
##  Median :41.00   Median :41.00   Median :3.500e+01  
##  Mean   :41.21   Mean   :41.35   Mean   :1.493e+13  
##  3rd Qu.:42.00   3rd Qu.:42.00   3rd Qu.:4.000e+01  
##  Max.   :45.00   Max.   :45.00   Max.   :1.000e+15  
##                                  NA's   :3
\end{verbatim}

\hypertarget{nas}{%
\paragraph{NAs}\label{nas}}

\begin{verbatim}
##      index value          
## [1,] "23"  "Don't know"   
## [2,] "25"  "Next to none."
## [3,] "68"  NA
\end{verbatim}

\begin{verbatim}
##                                   uni sp_aware obs_skl num_skl cblind vis_pro
## 101                        Technology        4       4       3     No      No
## 121                              <NA>        4       3       3     No      No
## 105 Sustainability/geological science        3       4       3     No    ADHD
\end{verbatim}

\hypertarget{q2---approximately-how-much-more-than-log-grip-would-you-say-salmon-ladder-was-was-used}{%
\paragraph{\texorpdfstring{\textbf{Q2 - Approximately how much more than
`Log Grip' would you say `Salmon Ladder' was was
used?}}{Q2 - Approximately how much more than `Log Grip' would you say `Salmon Ladder' was was used?}}\label{q2---approximately-how-much-more-than-log-grip-would-you-say-salmon-ladder-was-was-used}}

\begin{verbatim}
## [1] n = 70
\end{verbatim}

\begin{verbatim}
##    con_2_all       log_2_all       trn_2_all    
##  Min.   :3.000   Min.   :1.000   Min.   :1.000  
##  1st Qu.:4.250   1st Qu.:2.250   1st Qu.:5.000  
##  Median :5.000   Median :3.500   Median :6.000  
##  Mean   :5.357   Mean   :3.671   Mean   :5.871  
##  3rd Qu.:6.000   3rd Qu.:5.000   3rd Qu.:7.000  
##  Max.   :7.000   Max.   :7.000   Max.   :7.000
\end{verbatim}

\hypertarget{q3---approximately-how-much-more-than-quintuple-steps-would-you-say-salmon-ladder-was-used}{%
\paragraph{\texorpdfstring{\textbf{Q3 - Approximately how much more than
`Quintuple Steps' would you say `Salmon Ladder' was
used?}}{Q3 - Approximately how much more than `Quintuple Steps' would you say `Salmon Ladder' was used?}}\label{q3---approximately-how-much-more-than-quintuple-steps-would-you-say-salmon-ladder-was-used}}

\begin{verbatim}
## [1] n = 70
\end{verbatim}

\begin{verbatim}
##    con_3_all       log_3_all       trn_3_all    
##  Min.   :3.000   Min.   :1.000   Min.   :1.000  
##  1st Qu.:4.250   1st Qu.:2.250   1st Qu.:5.000  
##  Median :5.000   Median :3.500   Median :6.000  
##  Mean   :5.357   Mean   :3.671   Mean   :5.871  
##  3rd Qu.:6.000   3rd Qu.:5.000   3rd Qu.:7.000  
##  Max.   :7.000   Max.   :7.000   Max.   :7.000
\end{verbatim}

\hypertarget{q4---in-your-opinion-approximately-how-many-times-would-you-say-log-grip-was-used-as-a-percentage-of-the-number-of-times-salmon-ladder-was-used}{%
\paragraph{\texorpdfstring{\textbf{Q4 - In your opinion, approximately
how many times would you say `Log Grip' was used, as a percentage of the
number of times `Salmon Ladder' was
used?}}{Q4 - In your opinion, approximately how many times would you say `Log Grip' was used, as a percentage of the number of times `Salmon Ladder' was used?}}\label{q4---in-your-opinion-approximately-how-many-times-would-you-say-log-grip-was-used-as-a-percentage-of-the-number-of-times-salmon-ladder-was-used}}

\begin{verbatim}
## [1] n = 70
\end{verbatim}

\begin{verbatim}
##    con_4_all       log_4_all       trn_4_all     
##  Min.   : 5.00   Min.   : 0.10   Min.   :  1.00  
##  1st Qu.:50.00   1st Qu.: 0.50   1st Qu.: 14.38  
##  Median :50.00   Median : 0.75   Median : 50.00  
##  Mean   :47.66   Mean   :12.86   Mean   : 39.81  
##  3rd Qu.:50.00   3rd Qu.: 0.90   3rd Qu.: 50.00  
##  Max.   :75.00   Max.   :90.00   Max.   :100.00  
##  NA's   :3       NA's   :4       NA's   :2
\end{verbatim}

\hypertarget{nas-1}{%
\paragraph{NAs}\label{nas-1}}

\begin{verbatim}
##      index con_4_all log_4_all trn_4_all
## [1,]    11        NA        NA        NA
## [2,]    48        NA        NA        48
## [3,]    60        NA        NA        NA
## [4,]    68        50        NA        50
\end{verbatim}

\begin{verbatim}
##                                   uni sp_aware obs_skl num_skl cblind vis_pro
## 12                            Science        2       3       2     No      No
## 17                        Engineering        4       4       4     No     Yes
## 25                          Geography        4       4       4   <NA>      No
## 105 Sustainability/geological science        3       4       3     No    ADHD
\end{verbatim}

\end{document}
    

% Options for packages loaded elsewhere
\PassOptionsToPackage{unicode}{hyperref}
\PassOptionsToPackage{hyphens}{url}
%
\documentclass[
]{article}
\usepackage{lmodern}
\usepackage{amssymb,amsmath}
\usepackage{ifxetex,ifluatex}
\ifnum 0\ifxetex 1\fi\ifluatex 1\fi=0 % if pdftex
  \usepackage[T1]{fontenc}
  \usepackage[utf8]{inputenc}
  \usepackage{textcomp} % provide euro and other symbols
\else % if luatex or xetex
  \usepackage{unicode-math}
  \defaultfontfeatures{Scale=MatchLowercase}
  \defaultfontfeatures[\rmfamily]{Ligatures=TeX,Scale=1}
\fi
% Use upquote if available, for straight quotes in verbatim environments
\IfFileExists{upquote.sty}{\usepackage{upquote}}{}
\IfFileExists{microtype.sty}{% use microtype if available
  \usepackage[]{microtype}
  \UseMicrotypeSet[protrusion]{basicmath} % disable protrusion for tt fonts
}{}
\makeatletter
\@ifundefined{KOMAClassName}{% if non-KOMA class
  \IfFileExists{parskip.sty}{%
    \usepackage{parskip}
  }{% else
    \setlength{\parindent}{0pt}
    \setlength{\parskip}{6pt plus 2pt minus 1pt}}
}{% if KOMA class
  \KOMAoptions{parskip=half}}
\makeatother
\usepackage{xcolor}
\IfFileExists{xurl.sty}{\usepackage{xurl}}{} % add URL line breaks if available
\IfFileExists{bookmark.sty}{\usepackage{bookmark}}{\usepackage{hyperref}}
\hypersetup{
  pdftitle={Univariate Analysis - Whole Pop},
  pdfauthor={Katie Murphy},
  hidelinks,
  pdfcreator={LaTeX via pandoc}}
\urlstyle{same} % disable monospaced font for URLs
\usepackage[margin=1in]{geometry}
\usepackage{graphicx,grffile}
\makeatletter
\def\maxwidth{\ifdim\Gin@nat@width>\linewidth\linewidth\else\Gin@nat@width\fi}
\def\maxheight{\ifdim\Gin@nat@height>\textheight\textheight\else\Gin@nat@height\fi}
\makeatother
% Scale images if necessary, so that they will not overflow the page
% margins by default, and it is still possible to overwrite the defaults
% using explicit options in \includegraphics[width, height, ...]{}
\setkeys{Gin}{width=\maxwidth,height=\maxheight,keepaspectratio}
% Set default figure placement to htbp
\makeatletter
\def\fps@figure{htbp}
\makeatother
\setlength{\emergencystretch}{3em} % prevent overfull lines
\providecommand{\tightlist}{%
  \setlength{\itemsep}{0pt}\setlength{\parskip}{0pt}}
\setcounter{secnumdepth}{-\maxdimen} % remove section numbering

\title{Univariate Analysis - Whole Pop}
\author{Katie Murphy}
\date{27/02/2021}

\begin{document}
\maketitle

\hypertarget{q1---approximately-many-times-would-you-say-the-salmon-ladder-was-used}{%
\paragraph{\texorpdfstring{\textbf{Q1 - Approximately many times would
you say the `Salmon Ladder' was
used?}}{Q1 - Approximately many times would you say the `Salmon Ladder' was used?}}\label{q1---approximately-many-times-would-you-say-the-salmon-ladder-was-used}}

\begin{verbatim}
## [1] n = 70
\end{verbatim}

\begin{verbatim}
##    con_1_all       trn_1_all       log_1_all        
##  Min.   :40.00   Min.   :40.00   Min.   :9.000e+00  
##  1st Qu.:41.00   1st Qu.:41.00   1st Qu.:3.000e+01  
##  Median :41.00   Median :41.00   Median :3.500e+01  
##  Mean   :41.21   Mean   :41.35   Mean   :1.493e+13  
##  3rd Qu.:42.00   3rd Qu.:42.00   3rd Qu.:4.000e+01  
##  Max.   :45.00   Max.   :45.00   Max.   :1.000e+15  
##                                  NA's   :3
\end{verbatim}

\hypertarget{nas}{%
\paragraph{NAs}\label{nas}}

\begin{verbatim}
##      index value          
## [1,] "23"  "Don't know"   
## [2,] "25"  "Next to none."
## [3,] "68"  NA
\end{verbatim}

\begin{verbatim}
##                                   uni sp_aware obs_skl num_skl cblind vis_pro
## 101                        Technology        4       4       3     No      No
## 121                              <NA>        4       3       3     No      No
## 105 Sustainability/geological science        3       4       3     No    ADHD
\end{verbatim}

\hypertarget{q2---approximately-how-much-more-than-log-grip-would-you-say-salmon-ladder-was-was-used}{%
\paragraph{\texorpdfstring{\textbf{Q2 - Approximately how much more than
`Log Grip' would you say `Salmon Ladder' was was
used?}}{Q2 - Approximately how much more than `Log Grip' would you say `Salmon Ladder' was was used?}}\label{q2---approximately-how-much-more-than-log-grip-would-you-say-salmon-ladder-was-was-used}}

\begin{verbatim}
## [1] n = 70
\end{verbatim}

\begin{verbatim}
##    con_2_all       log_2_all       trn_2_all    
##  Min.   :3.000   Min.   :1.000   Min.   :1.000  
##  1st Qu.:4.250   1st Qu.:2.250   1st Qu.:5.000  
##  Median :5.000   Median :3.500   Median :6.000  
##  Mean   :5.357   Mean   :3.671   Mean   :5.871  
##  3rd Qu.:6.000   3rd Qu.:5.000   3rd Qu.:7.000  
##  Max.   :7.000   Max.   :7.000   Max.   :7.000
\end{verbatim}

\hypertarget{q3---approximately-how-much-more-than-quintuple-steps-would-you-say-salmon-ladder-was-used}{%
\paragraph{\texorpdfstring{\textbf{Q3 - Approximately how much more than
`Quintuple Steps' would you say `Salmon Ladder' was
used?}}{Q3 - Approximately how much more than `Quintuple Steps' would you say `Salmon Ladder' was used?}}\label{q3---approximately-how-much-more-than-quintuple-steps-would-you-say-salmon-ladder-was-used}}

\begin{verbatim}
## [1] n = 70
\end{verbatim}

\begin{verbatim}
##    con_3_all       log_3_all       trn_3_all    
##  Min.   :3.000   Min.   :1.000   Min.   :1.000  
##  1st Qu.:4.250   1st Qu.:2.250   1st Qu.:5.000  
##  Median :5.000   Median :3.500   Median :6.000  
##  Mean   :5.357   Mean   :3.671   Mean   :5.871  
##  3rd Qu.:6.000   3rd Qu.:5.000   3rd Qu.:7.000  
##  Max.   :7.000   Max.   :7.000   Max.   :7.000
\end{verbatim}

\hypertarget{q4---in-your-opinion-approximately-how-many-times-would-you-say-log-grip-was-used-as-a-percentage-of-the-number-of-times-salmon-ladder-was-used}{%
\paragraph{\texorpdfstring{\textbf{Q4 - In your opinion, approximately
how many times would you say `Log Grip' was used, as a percentage of the
number of times `Salmon Ladder' was
used?}}{Q4 - In your opinion, approximately how many times would you say `Log Grip' was used, as a percentage of the number of times `Salmon Ladder' was used?}}\label{q4---in-your-opinion-approximately-how-many-times-would-you-say-log-grip-was-used-as-a-percentage-of-the-number-of-times-salmon-ladder-was-used}}

\begin{verbatim}
## [1] n = 70
\end{verbatim}

\begin{verbatim}
##    con_4_all       log_4_all       trn_4_all     
##  Min.   : 5.00   Min.   : 0.10   Min.   :  1.00  
##  1st Qu.:50.00   1st Qu.: 0.50   1st Qu.: 14.38  
##  Median :50.00   Median : 0.75   Median : 50.00  
##  Mean   :47.66   Mean   :12.86   Mean   : 39.81  
##  3rd Qu.:50.00   3rd Qu.: 0.90   3rd Qu.: 50.00  
##  Max.   :75.00   Max.   :90.00   Max.   :100.00  
##  NA's   :3       NA's   :4       NA's   :2
\end{verbatim}

\hypertarget{nas-1}{%
\paragraph{NAs}\label{nas-1}}

\begin{verbatim}
##      index con_4_all log_4_all trn_4_all
## [1,]    11        NA        NA        NA
## [2,]    48        NA        NA        48
## [3,]    60        NA        NA        NA
## [4,]    68        50        NA        50
\end{verbatim}

\begin{verbatim}
##                                   uni sp_aware obs_skl num_skl cblind vis_pro
## 12                            Science        2       3       2     No      No
## 17                        Engineering        4       4       4     No     Yes
## 25                          Geography        4       4       4   <NA>      No
## 105 Sustainability/geological science        3       4       3     No    ADHD
\end{verbatim}

\end{document}

% Options for packages loaded elsewhere
\PassOptionsToPackage{unicode}{hyperref}
\PassOptionsToPackage{hyphens}{url}
%
\documentclass[
]{article}
\usepackage{lmodern}
\usepackage{amssymb,amsmath}
\usepackage{ifxetex,ifluatex}
\ifnum 0\ifxetex 1\fi\ifluatex 1\fi=0 % if pdftex
  \usepackage[T1]{fontenc}
  \usepackage[utf8]{inputenc}
  \usepackage{textcomp} % provide euro and other symbols
\else % if luatex or xetex
  \usepackage{unicode-math}
  \defaultfontfeatures{Scale=MatchLowercase}
  \defaultfontfeatures[\rmfamily]{Ligatures=TeX,Scale=1}
\fi
% Use upquote if available, for straight quotes in verbatim environments
\IfFileExists{upquote.sty}{\usepackage{upquote}}{}
\IfFileExists{microtype.sty}{% use microtype if available
  \usepackage[]{microtype}
  \UseMicrotypeSet[protrusion]{basicmath} % disable protrusion for tt fonts
}{}
\makeatletter
\@ifundefined{KOMAClassName}{% if non-KOMA class
  \IfFileExists{parskip.sty}{%
    \usepackage{parskip}
  }{% else
    \setlength{\parindent}{0pt}
    \setlength{\parskip}{6pt plus 2pt minus 1pt}}
}{% if KOMA class
  \KOMAoptions{parskip=half}}
\makeatother
\usepackage{xcolor}
\IfFileExists{xurl.sty}{\usepackage{xurl}}{} % add URL line breaks if available
\IfFileExists{bookmark.sty}{\usepackage{bookmark}}{\usepackage{hyperref}}
\hypersetup{
  pdftitle={Univariate Analysis - Whole Pop},
  pdfauthor={Katie Murphy},
  hidelinks,
  pdfcreator={LaTeX via pandoc}}
\urlstyle{same} % disable monospaced font for URLs
\usepackage[margin=1in]{geometry}
\usepackage{graphicx,grffile}
\makeatletter
\def\maxwidth{\ifdim\Gin@nat@width>\linewidth\linewidth\else\Gin@nat@width\fi}
\def\maxheight{\ifdim\Gin@nat@height>\textheight\textheight\else\Gin@nat@height\fi}
\makeatother
% Scale images if necessary, so that they will not overflow the page
% margins by default, and it is still possible to overwrite the defaults
% using explicit options in \includegraphics[width, height, ...]{}
\setkeys{Gin}{width=\maxwidth,height=\maxheight,keepaspectratio}
% Set default figure placement to htbp
\makeatletter
\def\fps@figure{htbp}
\makeatother
\setlength{\emergencystretch}{3em} % prevent overfull lines
\providecommand{\tightlist}{%
  \setlength{\itemsep}{0pt}\setlength{\parskip}{0pt}}
\setcounter{secnumdepth}{-\maxdimen} % remove section numbering

\title{Univariate Analysis - Whole Pop}
\author{Katie Murphy}
\date{27/02/2021}

\begin{document}
\maketitle

\hypertarget{q1---approximately-many-times-would-you-say-the-salmon-ladder-was-used}{%
\paragraph{\texorpdfstring{\textbf{Q1 - Approximately many times would
you say the `Salmon Ladder' was
used?}}{Q1 - Approximately many times would you say the `Salmon Ladder' was used?}}\label{q1---approximately-many-times-would-you-say-the-salmon-ladder-was-used}}

\begin{verbatim}
## [1] n = 70
\end{verbatim}

\begin{verbatim}
##    con_1_all       trn_1_all       log_1_all        
##  Min.   :40.00   Min.   :40.00   Min.   :9.000e+00  
##  1st Qu.:41.00   1st Qu.:41.00   1st Qu.:3.000e+01  
##  Median :41.00   Median :41.00   Median :3.500e+01  
##  Mean   :41.21   Mean   :41.35   Mean   :1.493e+13  
##  3rd Qu.:42.00   3rd Qu.:42.00   3rd Qu.:4.000e+01  
##  Max.   :45.00   Max.   :45.00   Max.   :1.000e+15  
##                                  NA's   :3
\end{verbatim}

\hypertarget{nas}{%
\paragraph{NAs}\label{nas}}

\begin{verbatim}
##      index value          
## [1,] "23"  "Don't know"   
## [2,] "25"  "Next to none."
## [3,] "68"  NA
\end{verbatim}

\begin{verbatim}
##                                   uni sp_aware obs_skl num_skl cblind vis_pro
## 101                        Technology        4       4       3     No      No
## 121                              <NA>        4       3       3     No      No
## 105 Sustainability/geological science        3       4       3     No    ADHD
\end{verbatim}

\hypertarget{q2---approximately-how-much-more-than-log-grip-would-you-say-salmon-ladder-was-was-used}{%
\paragraph{\texorpdfstring{\textbf{Q2 - Approximately how much more than
`Log Grip' would you say `Salmon Ladder' was was
used?}}{Q2 - Approximately how much more than `Log Grip' would you say `Salmon Ladder' was was used?}}\label{q2---approximately-how-much-more-than-log-grip-would-you-say-salmon-ladder-was-was-used}}

\begin{verbatim}
## [1] n = 70
\end{verbatim}

\begin{verbatim}
##    con_2_all       log_2_all       trn_2_all    
##  Min.   :3.000   Min.   :1.000   Min.   :1.000  
##  1st Qu.:4.250   1st Qu.:2.250   1st Qu.:5.000  
##  Median :5.000   Median :3.500   Median :6.000  
##  Mean   :5.357   Mean   :3.671   Mean   :5.871  
##  3rd Qu.:6.000   3rd Qu.:5.000   3rd Qu.:7.000  
##  Max.   :7.000   Max.   :7.000   Max.   :7.000
\end{verbatim}

\hypertarget{q3---approximately-how-much-more-than-quintuple-steps-would-you-say-salmon-ladder-was-used}{%
\paragraph{\texorpdfstring{\textbf{Q3 - Approximately how much more than
`Quintuple Steps' would you say `Salmon Ladder' was
used?}}{Q3 - Approximately how much more than `Quintuple Steps' would you say `Salmon Ladder' was used?}}\label{q3---approximately-how-much-more-than-quintuple-steps-would-you-say-salmon-ladder-was-used}}

\begin{verbatim}
## [1] n = 70
\end{verbatim}

\begin{verbatim}
##    con_3_all       log_3_all       trn_3_all    
##  Min.   :3.000   Min.   :1.000   Min.   :1.000  
##  1st Qu.:4.250   1st Qu.:2.250   1st Qu.:5.000  
##  Median :5.000   Median :3.500   Median :6.000  
##  Mean   :5.357   Mean   :3.671   Mean   :5.871  
##  3rd Qu.:6.000   3rd Qu.:5.000   3rd Qu.:7.000  
##  Max.   :7.000   Max.   :7.000   Max.   :7.000
\end{verbatim}

\hypertarget{q4---in-your-opinion-approximately-how-many-times-would-you-say-log-grip-was-used-as-a-percentage-of-the-number-of-times-salmon-ladder-was-used}{%
\paragraph{\texorpdfstring{\textbf{Q4 - In your opinion, approximately
how many times would you say `Log Grip' was used, as a percentage of the
number of times `Salmon Ladder' was
used?}}{Q4 - In your opinion, approximately how many times would you say `Log Grip' was used, as a percentage of the number of times `Salmon Ladder' was used?}}\label{q4---in-your-opinion-approximately-how-many-times-would-you-say-log-grip-was-used-as-a-percentage-of-the-number-of-times-salmon-ladder-was-used}}

\begin{verbatim}
## [1] n = 70
\end{verbatim}

\begin{verbatim}
##    con_4_all       log_4_all       trn_4_all     
##  Min.   : 5.00   Min.   : 0.10   Min.   :  1.00  
##  1st Qu.:50.00   1st Qu.: 0.50   1st Qu.: 14.38  
##  Median :50.00   Median : 0.75   Median : 50.00  
##  Mean   :47.66   Mean   :12.86   Mean   : 39.81  
##  3rd Qu.:50.00   3rd Qu.: 0.90   3rd Qu.: 50.00  
##  Max.   :75.00   Max.   :90.00   Max.   :100.00  
##  NA's   :3       NA's   :4       NA's   :2
\end{verbatim}

\hypertarget{nas-1}{%
\paragraph{NAs}\label{nas-1}}

\begin{verbatim}
##      index con_4_all log_4_all trn_4_all
## [1,]    11        NA        NA        NA
## [2,]    48        NA        NA        48
## [3,]    60        NA        NA        NA
## [4,]    68        50        NA        50
\end{verbatim}

\begin{verbatim}
##                                   uni sp_aware obs_skl num_skl cblind vis_pro
## 12                            Science        2       3       2     No      No
## 17                        Engineering        4       4       4     No     Yes
## 25                          Geography        4       4       4   <NA>      No
## 105 Sustainability/geological science        3       4       3     No    ADHD
\end{verbatim}

\end{document}

% Options for packages loaded elsewhere
\PassOptionsToPackage{unicode}{hyperref}
\PassOptionsToPackage{hyphens}{url}
%
\documentclass[
]{article}
\usepackage{lmodern}
\usepackage{amssymb,amsmath}
\usepackage{ifxetex,ifluatex}
\ifnum 0\ifxetex 1\fi\ifluatex 1\fi=0 % if pdftex
  \usepackage[T1]{fontenc}
  \usepackage[utf8]{inputenc}
  \usepackage{textcomp} % provide euro and other symbols
\else % if luatex or xetex
  \usepackage{unicode-math}
  \defaultfontfeatures{Scale=MatchLowercase}
  \defaultfontfeatures[\rmfamily]{Ligatures=TeX,Scale=1}
\fi
% Use upquote if available, for straight quotes in verbatim environments
\IfFileExists{upquote.sty}{\usepackage{upquote}}{}
\IfFileExists{microtype.sty}{% use microtype if available
  \usepackage[]{microtype}
  \UseMicrotypeSet[protrusion]{basicmath} % disable protrusion for tt fonts
}{}
\makeatletter
\@ifundefined{KOMAClassName}{% if non-KOMA class
  \IfFileExists{parskip.sty}{%
    \usepackage{parskip}
  }{% else
    \setlength{\parindent}{0pt}
    \setlength{\parskip}{6pt plus 2pt minus 1pt}}
}{% if KOMA class
  \KOMAoptions{parskip=half}}
\makeatother
\usepackage{xcolor}
\IfFileExists{xurl.sty}{\usepackage{xurl}}{} % add URL line breaks if available
\IfFileExists{bookmark.sty}{\usepackage{bookmark}}{\usepackage{hyperref}}
\hypersetup{
  pdftitle={Univariate Analysis - Whole Pop},
  pdfauthor={Katie Murphy},
  hidelinks,
  pdfcreator={LaTeX via pandoc}}
\urlstyle{same} % disable monospaced font for URLs
\usepackage[margin=1in]{geometry}
\usepackage{graphicx,grffile}
\makeatletter
\def\maxwidth{\ifdim\Gin@nat@width>\linewidth\linewidth\else\Gin@nat@width\fi}
\def\maxheight{\ifdim\Gin@nat@height>\textheight\textheight\else\Gin@nat@height\fi}
\makeatother
% Scale images if necessary, so that they will not overflow the page
% margins by default, and it is still possible to overwrite the defaults
% using explicit options in \includegraphics[width, height, ...]{}
\setkeys{Gin}{width=\maxwidth,height=\maxheight,keepaspectratio}
% Set default figure placement to htbp
\makeatletter
\def\fps@figure{htbp}
\makeatother
\setlength{\emergencystretch}{3em} % prevent overfull lines
\providecommand{\tightlist}{%
  \setlength{\itemsep}{0pt}\setlength{\parskip}{0pt}}
\setcounter{secnumdepth}{-\maxdimen} % remove section numbering

\title{Univariate Analysis - Whole Pop}
\author{Katie Murphy}
\date{27/02/2021}

\begin{document}
\maketitle

\hypertarget{q1---approximately-many-times-would-you-say-the-salmon-ladder-was-used}{%
\paragraph{\texorpdfstring{\textbf{Q1 - Approximately many times would
you say the `Salmon Ladder' was
used?}}{Q1 - Approximately many times would you say the `Salmon Ladder' was used?}}\label{q1---approximately-many-times-would-you-say-the-salmon-ladder-was-used}}

\begin{verbatim}
## [1] n = 70
\end{verbatim}

\begin{verbatim}
##    con_1_all       trn_1_all       log_1_all        
##  Min.   :40.00   Min.   :40.00   Min.   :9.000e+00  
##  1st Qu.:41.00   1st Qu.:41.00   1st Qu.:3.000e+01  
##  Median :41.00   Median :41.00   Median :3.500e+01  
##  Mean   :41.21   Mean   :41.35   Mean   :1.493e+13  
##  3rd Qu.:42.00   3rd Qu.:42.00   3rd Qu.:4.000e+01  
##  Max.   :45.00   Max.   :45.00   Max.   :1.000e+15  
##                                  NA's   :3
\end{verbatim}

\hypertarget{nas}{%
\paragraph{NAs}\label{nas}}

\begin{verbatim}
##      index value          
## [1,] "23"  "Don't know"   
## [2,] "25"  "Next to none."
## [3,] "68"  NA
\end{verbatim}

\begin{verbatim}
##                                   uni sp_aware obs_skl num_skl cblind vis_pro
## 101                        Technology        4       4       3     No      No
## 121                              <NA>        4       3       3     No      No
## 105 Sustainability/geological science        3       4       3     No    ADHD
\end{verbatim}

\hypertarget{q2---approximately-how-much-more-than-log-grip-would-you-say-salmon-ladder-was-was-used}{%
\paragraph{\texorpdfstring{\textbf{Q2 - Approximately how much more than
`Log Grip' would you say `Salmon Ladder' was was
used?}}{Q2 - Approximately how much more than `Log Grip' would you say `Salmon Ladder' was was used?}}\label{q2---approximately-how-much-more-than-log-grip-would-you-say-salmon-ladder-was-was-used}}

\begin{verbatim}
## [1] n = 70
\end{verbatim}

\begin{verbatim}
##    con_2_all       log_2_all       trn_2_all    
##  Min.   :3.000   Min.   :1.000   Min.   :1.000  
##  1st Qu.:4.250   1st Qu.:2.250   1st Qu.:5.000  
##  Median :5.000   Median :3.500   Median :6.000  
##  Mean   :5.357   Mean   :3.671   Mean   :5.871  
##  3rd Qu.:6.000   3rd Qu.:5.000   3rd Qu.:7.000  
##  Max.   :7.000   Max.   :7.000   Max.   :7.000
\end{verbatim}

\hypertarget{q3---approximately-how-much-more-than-quintuple-steps-would-you-say-salmon-ladder-was-used}{%
\paragraph{\texorpdfstring{\textbf{Q3 - Approximately how much more than
`Quintuple Steps' would you say `Salmon Ladder' was
used?}}{Q3 - Approximately how much more than `Quintuple Steps' would you say `Salmon Ladder' was used?}}\label{q3---approximately-how-much-more-than-quintuple-steps-would-you-say-salmon-ladder-was-used}}

\begin{verbatim}
## [1] n = 70
\end{verbatim}

\begin{verbatim}
##    con_3_all       log_3_all       trn_3_all    
##  Min.   :3.000   Min.   :1.000   Min.   :1.000  
##  1st Qu.:4.250   1st Qu.:2.250   1st Qu.:5.000  
##  Median :5.000   Median :3.500   Median :6.000  
##  Mean   :5.357   Mean   :3.671   Mean   :5.871  
##  3rd Qu.:6.000   3rd Qu.:5.000   3rd Qu.:7.000  
##  Max.   :7.000   Max.   :7.000   Max.   :7.000
\end{verbatim}

\hypertarget{q4---in-your-opinion-approximately-how-many-times-would-you-say-log-grip-was-used-as-a-percentage-of-the-number-of-times-salmon-ladder-was-used}{%
\paragraph{\texorpdfstring{\textbf{Q4 - In your opinion, approximately
how many times would you say `Log Grip' was used, as a percentage of the
number of times `Salmon Ladder' was
used?}}{Q4 - In your opinion, approximately how many times would you say `Log Grip' was used, as a percentage of the number of times `Salmon Ladder' was used?}}\label{q4---in-your-opinion-approximately-how-many-times-would-you-say-log-grip-was-used-as-a-percentage-of-the-number-of-times-salmon-ladder-was-used}}

\begin{verbatim}
## [1] n = 70
\end{verbatim}

\begin{verbatim}
##    con_4_all       log_4_all       trn_4_all     
##  Min.   : 5.00   Min.   : 0.10   Min.   :  1.00  
##  1st Qu.:50.00   1st Qu.: 0.50   1st Qu.: 14.38  
##  Median :50.00   Median : 0.75   Median : 50.00  
##  Mean   :47.66   Mean   :12.86   Mean   : 39.81  
##  3rd Qu.:50.00   3rd Qu.: 0.90   3rd Qu.: 50.00  
##  Max.   :75.00   Max.   :90.00   Max.   :100.00  
##  NA's   :3       NA's   :4       NA's   :2
\end{verbatim}

\hypertarget{nas-1}{%
\paragraph{NAs}\label{nas-1}}

\begin{verbatim}
##      index con_4_all log_4_all trn_4_all
## [1,]    11        NA        NA        NA
## [2,]    48        NA        NA        48
## [3,]    60        NA        NA        NA
## [4,]    68        50        NA        50
\end{verbatim}

\begin{verbatim}
##                                   uni sp_aware obs_skl num_skl cblind vis_pro
## 12                            Science        2       3       2     No      No
## 17                        Engineering        4       4       4     No     Yes
## 25                          Geography        4       4       4   <NA>      No
## 105 Sustainability/geological science        3       4       3     No    ADHD
\end{verbatim}

\end{document}


    
    \bibliographystyle{plain}
    \bibliography{references.bib}


\end{document}
