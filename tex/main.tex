\documentclass[12pt]{article}
\usepackage[utf8]{inputenc}

\usepackage{setspace}
\doublespacing

\title{\textbf{An Empirical Study of Visualisation of Data}}
\author{Katie Murphy \\ 1632254}

\begin{document}


\maketitle

\newpage

\begin{abstract}

\begin{center}
    \noindent \textbf{A study into the theory behind data visualisation.} 
    
     \vspace{0.3cm}
    
    \noindent This paper will investigate various ways in which data can be interpreted visually to portray information effectively in an elegant and concise manner. 
    
    \vspace{0.3cm}
    
    \noindent The investigation will include analysing several types of visualisation alongside discussing the benefits and downfalls of different software packages, namely the visualisation tools contained within R, Python and JavaScript.
    
    \vspace{0.3cm}
    
    An additional exploration will be made into discovering how understanding human psychology and visual processing can assist in creating effective visualisations.
    

\end{center}    
 
\end{abstract}



\newpage


\section{Introduction}

\begin{itemize}
\item What is visualisation?
\item Understanding theory of visualisation and implementation in Python, R and D3.
\item Types of visualisation (ie. static, dynamic and interactive)
\end{itemize}

\section{Literature reviews}
Look at advances in visualisation practices based on when papers were written.

\noindent (ie. in a 1996 paper they preferred this, but in a more recent paper they preferred this)

\noindent Is this down to the author's personal preference or do multiple papers back this up?

\section{Implementation of Visualisation}
\begin{itemize}
\item In Plotly, ggplot and D3
\item Ease of use
\item Consider data analysis tools available
\item Interactivity of visualisations (pros/cons) (using Shiny, plotly and D3) 
\item Reproducibility of visualisations (ie. use of out of software editing such as word or powerpoint to change labels etc)
\item Publication ready output?
\item Concept of storytelling
\end{itemize}

\section{Respondent Study}

\begin{itemize}
\item How the human mind interprets visuals
\item Focus groups and/or survey
\item Replicate research
\item Intuitive visualisations
\end{itemize}


\section{Visualise Study and Data Analysis}

\section{Critique of Implementation}

\bibliographystyle{plain}
\bibliography{references.bib}

\end{document}
